
%
\notePage{0}{Title and list of authors changed}

%
\notePage{iii}{Prelude added.}

%
\replacementPage{iv}{fourteen}{twenty}

%
\deletionPage{vii}{
   The Definition has evolved through a sequence of three previous versions,
   circulated as Technical Reports.  For those who have followed the
   sequence, we should point out that the treatment of {\sl equality types}
   and of {\sl admissibility} has been slightly modified in this publication 
   to meet the claim for principal signatures.  The changes are mainly
   in Sections~4.9, 5.5 and 5.13 and in the inference rules~19, 20,
   29 and 65.}

%
\replacementPage{vii}{Some}{Many}

%
\replacementPage{vii}{have also become}{became}

%
\insertionPage{vii}{1990}

%
\replacementPage{vii}{will not be}{are not}

%
\replacementPage{vii}{So we are writing
a Commentary\cite{mt91} on the Definition which will assist people in reading it, and
which will serve as a bridge between the Definition and other texts.}
{We therefore wrote a 
Commentary on the 1990
Definition to
assist people in reading it, and to
serve as a bridge between the Definition and other texts. Though in
part outdated by the present revision, the Commentary still fulfils its 
purpose.}

%
\insertionPage{vii}{There exist several textbooks on programming
with Standard ML\cite{paulson96,mcp93,Ullman94,stansifer92}. The second
edition of Pauson's book\cite{paulson96} conforms with the present revision.}

%
\deletionPage{vii}{\begin{flushright} Edinburgh\\August 1989 \end{flushright}}

%
\replacementPage{1}{just}{part of}

%
\replacementPage{2}{all predefined identifiers.}{a
small set of predefined identifiers. A richer basis is defined
in a separate document\cite{mllib96}.}

%
\deletionPage{2}{
This theory, particularly the theory of types and signatures, will
benefit from a more pedagogic treatment in other publications; the
treatment here is
probably the minimum required to understand the meaning of the rules.}

%
\deletionPage{2}{In two cases, however, they are presented as ``claims'' rather than
theorems; these are the claim of principal environments in 
Section~\ref{principal-env-sec}, and the claim of principal signatures
in Section~\ref{prinsig-sec}. We need further confirmation of our
detailed proofs of these claims, before asserting them as theorems.}

%
\replacementPage{3}{
An\index{6.2} {\sl integer constant} 
is any non-empty sequence of digits, possibly preceded
by a negation symbol (\tttilde).
A {\sl real constant} is an integer constant,
possibly followed by a point ({\tt .}) and one or
more digits, possibly followed by an exponent symbol ~{\tt E}~ and an integer
constant; at least one of the optional parts must occur, hence no integer
constant is a real constant.}{
An\index{6.2} {\sl integer constant (in
decimal notation)} is an optional negation symbol (\tttilde)
followed by a non-empty sequence of decimal digits (\boxml{0}-\boxml{9}).
An {\sl integer constant (in
hexadecimal notation)} is an optional negation symbol 
followed by \boxml{0x} followed by a non-empty sequence of
hexadecimal digits (\boxml{0}-\boxml{9}\boxml{a}-\boxml{f}\boxml{A}-\boxml{F},
where \boxml{A}-\boxml{F} are alternatives for \boxml{a}-\boxml{f},
respectively).

A {\sl word constant (in decimal notation)} is \boxml{0w}  followed
by a non-empty sequence of decimal digits. A {\sl word constant
(in hexadecimal notation)} is \boxml{0wx} followed by a non-empty
sequence of hexadecimal digits.
A {\sl real constant} is an integer constant in decimal notation,
possibly followed by a point ({\tt .}) and one or
more decimal digits, possibly followed by an exponent symbol ~({\tt E} or {\tt e})~ and an integer
constant in decimal notation; at least one of the optional parts must occur, hence no integer
constant is a real constant.}

%
\replacementPage{3}{We assume an underlying alphabet of 256 characters 
(numbered 0 to 255) such that the characters with numbers 
0 to 127 coincide with the ASCII character set.
}{We assume an underlying alphabet of $N$ characters ($N \geq 256$), numbered
     $0$ to $N-1$, which agrees with the ASCII character set on the characters
     numbered 0 to 127. The interval $[0, N-1]$ is called the {\sl ordinal range} of
     the alphabet.
%For use in string and character constants, we assume two underlying
%alphabets, of 8 bit and 16 bit characters, respectively. The {\sl ordinal
%ranges} of the two alphabets are $[0,255]$ and $[0,65535]$, respectively.
%In both cases, the characters with numbers 0 to 127 are assumed to coincide
%with the ASCII character set.
}

%
\notePage{3}{Inserted additional escape sequences in figure concerning
string constants and unicodes}

%
\insertionPage{4}{A {\sl character constant} is a sequence of the form
{\tt\#}$s$, where $s$ is a string constant denoting a string of size one character.

Libraries may provide multiple numeric types and multiple string types.
To each string type 
corresponds an alphabet with ordinal range $[0, N-1]$
for some $N\geq 256$; each alphabet must agree with the ASCII character set on 
the characters numbered 0 to 127. When multiple alphabets are supported,
all characters of a given string constant are interpreted over the same
alphabet. For each special constant, overloading
resolution is used for determining the type of the constant 
(see Appendix~\ref{overload.sec}).

%   All the escape sequences in a given string constant are interpreted
%   over the same underlying alphabet (either 8-bit characters or 16-bit 
%   characters), and for each string constant this alphabet is determined 
%   by overloading resolution (see Appendix~\ref{overload.sec}).  
%   It is a compile-time error 
% if the
%constant contains an escape sequence of the form $\uconst$ where
%$xxxx$ denotes an integer outside the ordinal range of the alphabet so
%determined. For example, within a sequence of 8-bit characters, the
%two leftmost hexadecimal digits of $\uconst$ must be {\tt 0} (zero).
}

%
\insertionPage{4}{word, character}

%
\replacementPage{4}{An 
unmatched comment bracket should be
detected by the compiler.}{No space is allowed between 
            the two characters which make up
            a comment bracket \ml{(*} or \ml{*)}.  An unmatched
            \boxml{(*} should be detected by the compiler.}

%
\notePage{4}{Figure 1 replaced by new figure (Var, Con and ExCon 
merged into VId)}

%
\replacementPage{4}{$\var$}{$\vid$}

%
\replacementPage{4}{Var}{$\VId$}

%
\deletionPage{5}{The subclass $\ImpTyVar$ of
$\TyVar$, the {\sl imperative} type variables, consists of those which
start with one or two primes followed by an underbar.  The complement
$\AppTyVar=\TyVar\setminus\ImpTyVar$\index{8.1} consists of the {\sl
applicative} type variables.
}

%
\replacementPage{5}{The other six classes ({\Var}, {\Con},
{\Exn}, {\TyCon}, {\Lab} and {\StrId}) are represented by identifiers
not starting with a prime.}{The other four classes ({\VId}, 
{\TyCon}, {\Lab} and {\StrId}) are represented by identifiers
not starting with a prime.}

%
\replacementPage{5}{six}{four}

%
\replacementPage{5}{
  \item Elsewhere, $\id$ is an exception constructor if it occurs in
        the scope of an exception binding which introduces it as such, 
        or a value constructor if it occurs in the
        scope of a datatype binding which introduced it as such;
        otherwise it is a value variable.}{
  \item Elsewhere, $\id$ is a value identifier.}

%
\deletionPage{5}{
It follows from the last rule that no value declaration can make a
``hole'' in the scope of a value or exception constructor 
by introducing the same identifier as a variable; this
is because, in the scope of the declaration which introduces $\id$ as a value
or exception constructor, any occurrence of $\id$ in a pattern 
is interpreted as the
constructor and not as the binding occurrence of a new variable.}

%
\replacementPage{6}{$\var$, a
$\con$ or an $\exn$}{$\vid$}

%
\replacementPage{6}{directive.}{directive, and in these uses it is called an
      {\sl infixed operator}.}

%
\replacementPage{6}{$\id$}{$\vid$}

%
\replacementPage{6}{``$\exp_1\ \id\ \exp_2$''}{``$\exp_1\ \vid\ \exp_2$''}

%
\replacementPage{6}{``$\pat_1\
\id\ \pat_2$''}{``$\pat_1\
\vid\ \pat_2$''}

%
\replacementPage{6}{$\id$}{$\vid$}

%
\replacementPage{6}{$\id$}{$\vid$}

%
\replacementPage{6}{$\id$}{$\vid$}

%
\replacementPage{6}{$\id$}{$\vid$}

%
\replacementPage{6}{\[ \longinfix \]
\[ \longinfixr \]
\[ \longnonfix \]}{\[ \newlonginfix \]
\[ \newlonginfixr \]
\[ \newlongnonfix \]}

%
\replacementPage{6}{$0$}{{\tt 0}}

%
\replacementPage{6}{~$\INFIX$~ and ~$\INFIXR$~ dictate left and right
associativity respectively; association is always to the left for different
operators of the same precedence.}{~$\INFIX$~ and ~$\INFIXR$~ dictate left and right
associativity respectively. In an expression of the form  $\exp_1\, \vid_1\, \exp_2\, \vid_2\, \exp_3$, where
      $\vid_1$ and $\vid_2$ are infixed operators with the same precedence, 
      either both must associate to the left or both must
associate to the right.
For example, suppose that {\tt <<} and {\tt >>} have equal precedence,
but associate to the left and right respectively; then
\medskip

\tabskip4cm
\halign to\hsize{\indent\hfil{\tt #}\tabskip1em&\hfil#\hfil\ &\ {\tt #}\hfil\cr
x << y << z&parses as&(x << y) << z\cr
x >> y >> z&parses as&x >> (y >> z)\cr
x << y >> z&is illegal\cr
x >> y << z&is illegal\cr}
\medskip}

%
\replacementPage{6}{infix}{infixed}

%
\replacementPage{7}{Figures~\ref{exp-syn}
and~\ref{pat-syn}.}{Figures~\ref{pat-syn} and~\ref{exp-syn}.}

%
\deletionPage{8}{\begin{minipage}{\textwidth}\halign{\indent#\hfil&#\hfil&#\hfil&#\hfil\cr
        &       & \opp\var      & variable\cr
        &       & \opp\longcon  & constant\cr
        &       & \opp\longexn  & exception constant\cr}\end{minipage}}

%
\insertionPage{8}{$\opp\longvid$\qquad value identifier}

%
\insertionPage{8}{::=}

%
\deletionPage{8}{\begin{minipage}{\textwidth}\halign{\indent#\hfil&#\hfil&#\hfil&#\hfil\cr
        &       & \opp\conpat   & value construction\cr
        &       & \opp\exconpat  & exception construction\cr
        &       & \infpat       & infixed value construction\cr
        &       & \infexpat     & infixed exception construction\cr}\end{minipage}}

%
\insertionPage{8}{\begin{minipage}{\textwidth}\halign{\indent#\hfil&#\hfil&#\hfil&#\hfil\cr
	& \opp\vidpat   & constructed pattern\cr
        &       & \vidinfpat       & infixed value construction\cr}\end{minipage}}

%
\replacementPage{8}{$\opp\layeredpat$}{$\opp\layeredvidpat$}

%
\deletionPage{10}{\begin{minipage}{\textwidth}\halign{\indent#\hfil&#\hfil&#\hfil&#\hfil\cr
        &       & \opp\longvar  & value variable\cr
        &       & \opp\longcon  & value constructor\cr
        &       & \opp\longexn  & exception constructor\cr}\end{minipage}}

%
\insertionPage{10}{$\opp\longvid$\quad value identifier}

%
\replacementPage{10}{\infexp}{\vidinfexp}

%
\replacementPage{10}{\valdec}{\explicitvaldec}

%
\insertionPage{10}{$\datatyperepldecb$\qquad datatype replication}

%
\replacementPage{10}{\longinfix}{\newlonginfix}

%
\replacementPage{10}{\longinfixr}{\newlonginfixr}

%
\replacementPage{10}{\longnonfix}{\newlongnonfix}

%
\replacementPage{10}{$\opp\longconstrs$}{$\opp\longvidconstrs$}

%
\replacementPage{10}{\generativeexnbind}{\generativeexnvidbind}

%
\replacementPage{10}{\eqexnbind}{\eqexnvidbind}

%
\deletionPage{8}{No pattern may contain the same $\var$ twice. }

%
\insertionPage{8}{-expression}

%
\replacementPage{8}{constructors}{identifiers}

%
\insertionPage{8}{
For each $\dec$ of the form $\datatyperepldeca$,
      the sequences $\tyvarseq$ and $\tyvarseq'$ must be equal and neither
      may contain the same type variable twice.}

%
\insertionPage{9}{\item No $\datbind$ or $\exnbind$ may bind $\TRUE$, $\FALSE$, $\IT$, 
$\NIL$, \boxml{::} or $\REF$.}

%
\replacementPage{11}{The derived forms for modules concern functors and  appear in 
Appendix~\ref{derived-forms-app}.}{The derived forms for modules  appear in 
Appendix~\ref{derived-forms-app}.}

%
\notePage{11}{Inserted new keywords  {\tt where} and {\tt :>}}

%
\replacementPage{11}{syntax}{identifier}

%
\replacementPage{11}{generative}{basic}

%
\deletionPage{11}{A more liberal scheme (which is under consideration)
would allow fixity directives to appear also as specifications, so that
fixity may be dictated by a signature expression; furthermore, it would allow an ~$\OPEN$~
or ~$\INCLUDE$~ construction to restore the fixity which prevailed
in the structures being opened, or in the signatures being included.
This scheme is not adopted at present.}

%
\deletionPage{12}{SharEq \quad sharing equations}

%
\deletionPage{12}{\begin{minipage}{\textwidth}\halign{\indent#\hfil&#\hfil&#\hfil&#\hfil\cr
FunSigExp & functor signature expressions\cr
FunSpec & functor specifications\cr
FunDesc & functor descriptions\cr}\end{minipage}}

%
\deletionPage{11}{It should be noted that functor specifications (FunSpec) cannot
occur in programs;
neither can the associated functor descriptions (FunDesc)
and functor signature expressions (FunSigExp).  The purpose of a $\funspec$
is to specify the static attributes (i.e. functor signature) of one
or more functors. This will be useful, in fact essential, for
separate compilation of functors. If, for example, a functor $g$
refers to another functor $f$ then --- in order to compile $g$ in
the absence of the declaration of $f$ --- at least the specification
of $f$ (i.e. its functor signature) must be available. At present there is no
special grammatical form for  a separately compilable ``chunk'' of text
-- which we may like to call call a {\sl module} -- containing a $\fundec$
together with a $\funspec$ specifying its global references. However, below in
the semantics for Modules it is defined when a
declared functor matches a functor signature specified for it. This determines
exactly those functor environments (containing declared functors
such as $f$) into which the separately compiled ``chunk''
containing the declaration of $g$ may be loaded.}

%
\replacementPage{13}{generative}{basic}

%
\insertionPage{13}{$\transpconstraint$\quad transparent constraint}

%
\insertionPage{13}{ $ \opaqueconstraint$\quad opaque constraint}

%
\replacementPage{13}{$\strbindera$}{$\barestrbindera$}

%
\replacementPage{13}{generative}{basic}

%
\insertionPage{13}{$\wheretypesigexp$\quad type realisation}

%
\deletionPage{13}{single}

%
\deletionPage{13}{\begin{minipage}{\textwidth}\halign{\indent#\hfil&#\hfil&#\hfil&#\hfil\cr
        &       & \emptysigdec          & empty\cr
        &       & \seqsigdec            & sequential\cr}\end{minipage}}

%
\replacementPage{12}{, $\strdesc$ or $\fundesc$}{or $\strdesc$}

%
\replacementPage{12}{constructors}{identifiers}

%
\insertionPage{12}{
No ${\it tyvarseq}$ may contain the same ${\it tyvar}$ twice.}

%
\insertionPage{12}{
For each $\spec$ of the form $\datatypereplspeca$, the sequences
$\tyvarseq$ and $\tyvarseq'$ must be equal.}

%
\insertionPage{12}{
 
      Any $\tyvar$ occurring on the right side of a $\datdesc$ of the form\linebreak
 $\tyvarseq \;\tycon\;\boxml{=}$ $\cdots$ must occur
      in the $\tyvarseq$; similarly, in signature expressions of the form $\sigexp\ \boxml{where type}\, \tyvarseq\,\longtycon\,$
$\boxml{=}\,\ty$, any $\tyvar$ occurring in $\ty$ must occur in $\tyvarseq$.}

%
\insertionPage{12}{\item No $\datdesc$ or $\exndesc$ may describe 
$\TRUE$, $\FALSE$, $\IT$, 
$\NIL$, \boxml{::} or $\REF$.}

%
\insertionPage{14}{$\datatypereplspecb$\qquad replication}

%
\deletionPage{14}{\begin{minipage}{\textwidth}\halign{\indent#\hfil&#\hfil&#\hfil&#\hfil\cr
        &	& \sharingspec	        & sharing\cr
	&	& \localspec    	& local\cr
        &       & \openspec             & open ($n\geq 1$)\cr }\end{minipage}}

%
\replacementPage{14}{$\inclspec$\qquad include ($n\geq 1$)}{
$\singleinclspec$\qquad include}

%
\insertionPage{14}{$\newsharingspec$\qquad sharing \quad $(n\geq 2)$}

%
\replacementPage{14}{\valdescription}{\valviddescription}

%
\replacementPage{14}{\condescription}{\conviddescription}

%
\replacementPage{14}{\exndescription}{\exnviddescription}

%
\deletionPage{14}{\begin{minipage}{\textwidth}\halign{\indent#\hfil&#\hfil&#\hfil&#\hfil\cr
\shareq & ::=   & \strshareq            & structure sharing\cr
        &       &                       & \qquad ($n\geq 2$) \cr
        &       & \typshareq            & type sharing \cr
        &       &                       & \qquad ($n\geq 2$) \cr
        &       & \multshareq           & multiple\cr\noalign{\vspace{6pt}}
\noalign{\vspace{6pt}}}\end{minipage}}

%
\notePage{12}{Figure 8: Functor signature expressions and
functor specifications have been removed. Sequential and empty
functor declarations and signature declarations have been removed.
The grammar for $\topdec$ now allows sequencing (without semicolon) 
of structure-level declarations, signature declarations and functor
declarations.}

%
\deletionPage{12}{
\subsection{Closure Restrictions}
\label{closure-restr-sec}
The\index{18.2} semantics presented in later sections requires no restriction on
reference to non-local identifiers. For example, it allows a signature 
expression to refer to external signature identifiers and
(via ~$\SHARING$~ or ~$\OPEN$~) to external structure identifiers; it also
allows a functor to refer to external identifiers of any kind.

However, implementers who want to provide a simple facility for
separate compilation may want to impose the following restrictions
(ignoring references to identifiers bound in the initial basis
$\B_0$, which may occur anywhere):

%However, in the present version of the language,
%apart from references to identifiers bound in the initial basis $B_0$
%(which may occur anywhere), it is required that signatures only refer
%non-locally to signature identifiers and that functors only
%refer non-locally to functor and signature identifiers.
%These restrictions ease separate
%compilation; however, they may be relaxed in a future version of the language.
%
%More precisely, the restrictions are as follows (ignoring reference to
%identifiers bound in $B_0$):
\begin{enumerate}
\item In any signature binding ~$\sigid\ \mbox{{\tt =}}\ \sigexp$~,
the only non-local
references in $\sigexp$ are to signature identifiers.
\item In any functor description ~$\funid\ \longfunsigexpa$~,
the only non-local
references in $\sigexp$ and $\sigexp'$ are to signature identifiers,
except that $\sigexp'$ may refer to $\strid$ and its components.
\item In any functor binding ~$\funstrbinder$~, the only non-local
references in $\sigexp$, $\sigexp'$ and $\strexp$ are to functor and signature
identifiers,
except that both $\sigexp'$ and $\strexp$ may refer to $\strid$ and
its components.
\end{enumerate}
In the last two cases the final qualification allows, for example, sharing
constraints to be specified between functor argument and result.
(For a completely precise definition of these closure restrictions,
see the comments to rules \ref{single-sigdec-rule} 
(page~\pageref{single-sigdec-rule}), 
\ref{singfunspec-rule} (page~\pageref{singfunspec-rule})
and \ref{singfundec-rule} (page~\pageref{singfundec-rule})
in the static semantics of modules, Section~\ref{statmod-sec}.)

The\index{19.1} 
significance of these restrictions is that they may ease separate
compilation; this may be seen as follows. If one takes a {\sl module}
to be a sequence of signature declarations, functor specifications
and functor declarations satisfying the above restrictions then the
elaboration of a module can be made to depend on the initial
static basis alone (in particular, it will not rely on
structures outside the module). Moreover, the elaboration 
of a module cannot create new free structure or type names, so 
name consistency (as defined in Section~\ref{consistency-sec}, 
page \pageref{consistency-sec}) is automatically preserved
across separately compiled modules. On the other hand,
imposing these restrictions may force the programmer to write
many more sharing equations than is needed if functors
and signature expressions can refer to free structures.
}

%
\replacementPage{15}{All semantic objects in the static semantics of the entire 
language are built from identifiers and two further kinds of simple objects: 
type constructor names and structure names.}{All semantic objects in 
the static semantics of the entire 
language are built from identifiers and two further kinds of simple objects: 
type constructor names and identifier status descriptors.}

%
\deletionPage{15}{Structure names play an active role only in
the Modules semantics; they enter the Core semantics only because
they appear in structure environments, which (in turn) are needed in the Core
semantics only to determine the values of long identifiers.}

%
\replacementPage{15}{
\begin{displaymath}
\begin{array}{rclr}
\alpha\ {\rm or}\ \tyvar & \in   & \TyVar       & \mbox{type variables}\\
\t               & \in   & \TyNames     & \mbox{type names}\\ 
\m              & \in   & \StrNames     & \mbox{structure names}
\end{array}
\end{displaymath}}{\begin{displaymath}
\begin{array}{rcll}
\alpha\ {\rm or}\ \tyvar & \in   & \TyVar       & \mbox{type variables}\\
\t               & \in   & \TyNames     & \mbox{type names}\\
\is              & \in   & \IdStatus = \{\isc,\ise,\isv\}    & \mbox{identifier status descriptors}
\end{array}
\end{displaymath}}

%
\deletionPage{15}{Independently hereof, each $\alpha$ possesses a boolean attribute,
the {\sl imperative} attribute, which determines whether it is imperative,
i.e. whether it is a member of $\ImpTyVar$ (defined on page~\pageref{etyvar-lab})
or not.}

%
\replacementPage{15}{{\INT}, {\REAL} 
 or {\STRING}}{{\INT}, {\REAL}, {\WORD}, {\CHAR}
 or {\STRING}}

%
\insertionPage{15}{(However, see Appendix~\ref{overload.sec} 
concerning types of overloaded special constants.)}

%
\replacementPage{16}{\RecType}{\RowType}

%
\replacementPage{16}{\RecType}{\RowType}

%
\deletionPage{16}{\begin{minipage}{\textwidth}\halign{\indent$#$\hfil&$#$\hfil&$#$\hfil&$#$\hfil\cr
\S\ {\rm or}\ (\m,\E)
                & \in   & \Str = \StrNames\times\Env  \cr}\end{minipage}}

%
\replacementPage{16}{\CE}{\VE}

%
\replacementPage{16}{$\TyStr = \TypeFcn\times\ConEnv$}{$\TyStr = \TypeFcn\times\ValEnv$}

%
\replacementPage{16}{$\StrEnv = \finfun{\StrId}{\Str}$}{$\StrEnv = \finfun{\StrId}{\Env}$}

%
\deletionPage{16}{$\CE\quad \in\quad \ConEnv = \finfun{\Con}{\TypeScheme}$}

%
\replacementPage{16}{$\VarEnv = \finfun{(\Var\cup\Con\cup\Exn)}{\TypeScheme}$}{$\ValEnv = \finfun{\VId}{\TypeScheme\times\IdStatus}$}

%
\deletionPage{16}{$\EE \in   \ExnEnv = \finfun{\Exn}{\Type}$}

%
\replacementPage{16}{$\longE{}$}{$\newlongE{}$}

%
\replacementPage{16}{$\Env = \StrEnv\times\TyEnv\times\VarEnv\times\ExnEnv$}{$\Env = \StrEnv\times\TyEnv\times\ValEnv$}

%
\deletionPage{16}{Moreover, $\imptyvars A$ and $\apptyvars A$ denote respectively the set
of imperative type variables and the set of applicative
type variables occurring free in $A$.}

%
\insertionPage{16}{\par Also note that a value environment maps
value identifiers to a pair of a type scheme and an identifier status.
If $\VE(\vid) = (\sigma,\is)$, we say that $\vid$ {\sl has status $\is$
in $\VE$}. An occurrence of a value identifier which is elaborated
in $\VE$ is referred to as a {\sl value variable}, a {\sl value constructor}
or an {\sl exception constructor}, depending on whether its status in $\VE$
is $\isv$, $\isc$ or $\ise$, respectively. }

%
\replacementPage{17}{variable-environment}{value-environment}

%
\replacementPage{17}{variable}{metavariable}

%
\replacementPage{17}{variable-environment}{value-environment}

%
\deletionPage{17}{ and ``$\of{\m}{\S}$'' means ``the structure name of $\S$''}

%
\replacementPage{17}{For instance $\C(\tycon)$ means
$(\of{\TE}{\C})\tycon$.

A particular case needs mention:  $\C(\con)$ is taken to stand for
$(\of{\VE}{\C})\con$; similarly, $\C(\exn)$ is taken to stand for
$(\of{\VE}{\C})\exn$.
  The type scheme of a value constructor is
held in $\VE$ as well as in $\TE$ (where it will be recorded within
a $\CE$); similarly, the type of an exception constructor is held in
$\VE$ as well as in $\EE$.
Thus the re-binding of a constructor of either kind is given proper
effect by accessing it in $\VE$, rather than in $\TE$ or in $\EE$.}{For 
instance $\C(\tycon)$ means
$(\of{\TE}{\C})\tycon$ and $\C(\vid)$ means $(\of{\VE}{(\of{E}{\C})})(\vid)$.}

%
\replacementPage{17}{For instance if $\longcon = \strid_1.\cdots.\strid_k.\con$ then
$\E(\longcon)$ means
\[ (\of{\VE}
       {(\of{\SE}
            {\cdots(\of{\SE}
                       {(\of{\SE}{\E})\strid_1}
                   )\strid_2\cdots}
        )\strid_k}
    )\con.
\]
}{For instance if $\longvid = \strid_1.\cdots.\strid_k.\vid$ then
$\E(\longvid)$ means
\[ (\of{\VE}
       {(\of{\SE}
            {\cdots(\of{\SE}
                       {(\of{\SE}{\E})\strid_1}
                   )\strid_2\cdots}
        )\strid_k}
    )\vid.
\]
}

%
\replacementPage{17}{$(\emptymap,\emptymap,\VE,\emptymap)$.}{$(\emptymap,\emptymap,\VE)$.}

%
\replacementPage{17}{$\E+(\emptymap,\emptymap,\VE,\emptymap)$.}{$\E+(\emptymap,\emptymap,\VE)$.}

%
\deletionPage{17}{Similarly, the imperative attribute has no significance 
in the bound variable of a type function.}

%
\deletionPage{18}{A type is {\sl imperative} if all type variables occurring in it are
imperative.}

%
\replacementPage{18}{if $\tau'=\tau\{\tauk/\alphak\}$ for some $\tauk$, where each member $\tau_i$
of $\tauk$ admits equality if $\alpha_i$ does,  
%poly 
and $\tau_i$ is imperative if $\alpha_i$ is imperative.}{if $\tau'=\tau\{\tauk/\alphak\}$ for some $\tauk$, where each member $\tau_i$
of $\tauk$ admits equality if $\alpha_i$ does.}

%
\deletionPage{18}{Similarly, the imperative attribute of a bound type variable of a
type scheme {\sl is} significant.}

%
\insertionPage{18}{Moreover, in a value declaration
{\tt val $\tyvarseq$ $\valbind$}, the sequence $\tyvarseq$ binds
type variables: a type variable occurs free in 
{\tt val $\tyvarseq$ $\valbind$} iff it occurs free in $\valbind$
and is not in the sequence $\tyvarseq$.}

%
\deletionPage{18}{
In the modules, a description of a value, type, or datatype
may contain explicit type variables whose scope is that
description.}

%
\insertionPage{18}{explicit binding of type
variables at {\tt val} is optional, so}

%
\insertionPage{18}{free}

%
\replacementPage{18}{First, an occurrence of $\alpha$ in a value declaration $\valdec$ is said
to be {\sl unguarded} if the occurrence is not part of a smaller value
declaration within $\valbind$.}{First, a free occurrence of $\alpha$ in a value declaration 
$\explicitvaldec$ is said
to be {\sl unguarded} if the occurrence is not part of a smaller value
declaration within $\valbind$.}

%
\replacementPage{18}{Then we say that $\alpha$ is {\sl scoped at} 
a particular occurrence
$O$ of $\valdec$ in a program if}{Then we say that $\alpha$ is {\sl implicitly scoped at} a particular value declaration
{\tt val $\tyvarseq$ $\valbind$} in a program if}

%
\replacementPage{18}{occurrence $O$.}{given one.}

%
\deletionPage{18}{
Hence, associated with every occurrence of a value declaration there is 
a set $\U$ of the explicit type variables that are implicitly
scoped at that
occurrence. One may think of each occurrence of $\VAL$ as being implicitly
decorated with such a  set, for instance:

\vspace*{3mm}
\halign{\indent$#$&$#$&$#$\cr
\mbox{$\VAL_{\{\}}$ \ml{x = }}&\mbox{\ml{(}}&
\mbox{\ml{let $\VAL_{\{\mbox{\ml{'a}}\}}$ Id1:'a->'a = fn z=>z in Id1 Id1 end,}}\cr
& &\mbox{\ml{let $\VAL_{\{\mbox{\ml{'a}}\}}$ Id2:'a->'a = fn z=>z in Id2 Id2 end)}}\cr
\noalign{\vspace*{3mm}}
\mbox{$\VAL_{\{\mbox{\ml{'a}}\}}$ \ml{x = }}&\mbox{\ml{(}}&
\mbox{\ml{let $\VAL_{\{\}}$ Id:'a->'a = fn z=>z in Id Id end,}}\cr
& &\mbox{\ml{fn z=> z:'a)}}\cr}
}

%
\insertionPage{19}{Henceforth, we assume that for every
value declaration $\boxml{val}\,\tyvarseq\cdots$ occurring in the
program, every explicit type variable implicitly scoped at the {\tt val}
has been added to $\tyvarseq$. Thus for example, in the two declarations
\begin{tabbing}
\indent\=\tt  val x =  let val id:'a->'a = fn z=>z in id id end\\
       \>\tt  val x = (let val id:'a->'a = fn z=>z in id id end; fn z=>z:'a)
\end{tabbing}
the type variable \boxml{'a} is scoped differently; they become respectively
\begin{tabbing}
\indent\=\tt val x =  let val 'a id:'a->'a = fn z=>z in id id end\\
       \>\tt val 'a x = (let val id:'a->'a = fn z=>z in id id end; fn z=>z:'a)
\end{tabbing}
}

%
\replacementPage{19}{According}{Then, according}

%
\replacementPage{19}{\ml{Id}}{\ml{id}}

%
\replacementPage{19}{\ml{Id Id}}{\ml{id id}}

%
\deletionPage{19}{
\subsection{Non-expansive Expressions}
\label{expansive-sec}
In\index{23.4} order to treat polymorphic references and exceptions,
the set Exp of expressions is partitioned into two classes, the {\sl
expansive} and the {\sl non-expansive} expressions. 
Any variable,
constructor and $\FN$ expression, possibly constrained by one or more
type expressions, is non-expansive; all other expressions are said to
be expansive.  The idea is that the dynamic evaluation of a
non-expansive expression will neither generate an exception nor extend
the domain of the memory, while the evaluation of an expansive
expression might.}

%
\insertionPage{19}{
\subsection{Non-expansive Expressions}
\label{expansive-sec}
In\index{23.4} order to treat polymorphic references and exceptions,
the set Exp of expressions is partitioned into two classes, the {\sl
expansive} and the {\sl non-expansive} expressions. 
An expression
     is {\sl non-expansive in context $\C$} if, after replacing infixed forms 
     by their equivalent prefixed forms, and derived forms by their equivalent
     forms, it can be generated  by the following grammar from the 
     non-terminal $\nexp$:
\medskip

\halign{\indent\hfil$#$\ &\ $#$\hfil\ &\ $#$\hfil&\ #\hfil\cr
\nexp&::=&\scon\cr
&&\langle\OP\rangle\longvid\cr
&&\ttlbrace\langle\nexprow\rangle\ttrbrace\cr
&&\boxml{($\nexp$)}\cr
&&\boxml{$\conexp\;\nexp$}\cr
%&&\boxml{$\longvid\;\nexp$}\qquad\qquad\hbox{restrictions:}&$\longvid\neq\REF$ and\cr
%&&&$\of{\is}{\C(\longvid)}\in\{\isc,\ise\}$\cr
&&\nexp \boxml{:} \ty\cr
%&&\boxml{$\nexp$ handle $\match$}\cr
&&\boxml{fn $\match$}\cr%\cr
\nexprow&::=&\boxml{$\lab$ = $\nexp\langle$, $\nexprow\rangle$}\cr
\conexp&::=&\boxml{($\conexp\langle$:$\ty\rangle$)}\cr
       &&\hbox{$\langle\OP\rangle\longvid$\hskip3cm Restriction:}&$\longvid\neq\REF$ and\cr
&&&$\of{\is}{\C(\longvid)}\in\{\isc,\ise\}$\cr}
\medskip

\noindent
All other expressions are said to be {\sl expansive (in $C$)}.
The idea is that the dynamic evaluation of a
non-expansive expression will neither generate an exception nor extend
the domain of the memory, while the evaluation of an expansive
expression might.
}

%
\replacementPage{19}{variable environment}{value 
environment}

%
\replacementPage{19}{\[\cl{A}{\VE}=\{\id\mapsto\cl{A}{(\tau)}\ ;\ \VE(\id)=\tau\}\]}{\[\cl{A}{\VE}=\{\vid\mapsto(\cl{A}{(\tau)},\is)\ ;\ \VE(\vid)=(\tau,\is)\}\]}

%
\deletionPage{20}{with a similar definition for $\cl{A}{\CE}$.}

%
\replacementPage{20}{variable environment}{value environment}

%
\replacementPage{20}{Thus, for each $\var\in\Dom\VE$ 
there is a unique \mbox{\pat\ \ml{=} \exp}
in $\valbind$ which binds $\var$.}{Thus, for each $\vid\in\Dom\VE$ 
there is a unique \mbox{\pat\ \ml{=} \exp}
in $\valbind$ which binds $\vid$.}

%
\replacementPage{20}{$\VE(\var)=\tau$}{$\VE(\vid)=(\tau,\is)$}

%
\replacementPage{20}{$\cl{\C,\valbind}{\VE(\var)}=\longtych$}{$\cl{\C,\valbind}{\VE(\vid)}=(\longtych,\is)$}

%
\replacementPage{20}{\[\alphak=\cases{\TyVarFcn\tau\setminus\TyVarFcn\C,&if $\exp$ 
                                                    is non-expansive;\cr
                 \apptyvars\tau\setminus\TyVarFcn\C,&if $\exp$ is expansive.}
\]
}{
\[\alphak=\cases{\TyVarFcn\tau\setminus\TyVarFcn\C,&if $\exp$ 
                                                    is non-expansive in $\C$;\cr
                 (),&if $\exp$ is expansive in $\C$.}
\]}

%
\deletionPage{20}{Notice that the form of $\valbind$ does not affect the binding of
applicative type variables, only the binding of imperative
type variables.}

%
\replacementPage{20}{$(\theta,\CE)$}{$(\theta,\VE)$}

%
\replacementPage{20}{$(\theta,\CE)$}{$(\theta,\VE)$}

%
\replacementPage{20}{$\CE\neq\emptymap$}{$\VE\neq\emptymap$}

%
\insertionPage{20}{An object or assembly $A$ of semantic objects is {\sl well-formed} if every type structure
occurring in $A$ is well-formed.}

%
\replacementPage{20}{assumed}{required}

%
\replacementPage{20}{$(\t,\CE)$}{$(\t,\VE)$}

%
\replacementPage{20}{$\CE(\con)$}{$\VE(\vid)$}

%
\replacementPage{20}{$\forall\alphak.(\tau\rightarrow\alphak\t)$,}{$(\forall\alphak.(\tau\rightarrow\alphak\t), \is)$,}

%
\replacementPage{20}{$(\con,v)$}{$(\vid,v)$}

%
\replacementPage{20}{$(\t,\CE)$ }{$(\t,\VE)$ }

%
\replacementPage{20}{$\CE\neq\emptymap$}{$\VE\neq\emptymap$}

%
\replacementPage{20}{\[\TE=\{\tycon_i\mapsto(t_i,\CE_i)\ ;\ 1\leq i\leq k\},\]}{\[\TE=\{\tycon_i\mapsto(t_i,\VE_i)\ ;\ 1\leq i\leq k\},\]}

%
\replacementPage{20}{$\CE_i$}{$\VE_i$}

%
\replacementPage{20}{constructor}{value}

%
\replacementPage{20}{$\CE_i$}{$\VE_i$}

%
\replacementPage{21}{$\B=(\M,\T),\F,\G,\E$}{$\B=\T,\F,\G,\E$}

%
\replacementPage{21}{\begin{equation}        % value variable
\label{varexp-rule}
\frac{\C(\longvar)\succ\tau}
     {\C\ts\longvar\ra\tau}
\end{equation}}{\begin{equation}        % value variable
\label{varexp-rule}
\frac{\C(\longvid) = (\sigma,\is)\qquad\sigma\succ\tau}
     {\C\ts\longvid\ra\tau}
\end{equation}}

%
\deletionPage{21}{\begin{equation}        % value constructor
\label{conexp-rule}
\frac{\C(\longcon)\succ\tau}
     {\C\ts\longcon\ra\tau}
\end{equation}

\begin{equation}      % exception constant
%\label{exconexp-rule}
\frac{\C(\longexn)=\tau}
     {\C\ts\longexn\ra\tau}
\end{equation}}

%
\replacementPage{21}{\begin{equation}        % local declaration
\label{let-rule}
\frac{\C\ts\dec\ra\E\qquad\C\oplus\E\ts\exp\ra\tau}
     {\C\ts\letexp\ra\tau}\index{27.1}
\end{equation}}{\begin{equation}        % local declaration
\label{let-rule}
\frac{\C\ts\dec\ra\E\qquad\C\oplus\E\ts\exp\ra\tau\qquad\TyNamesFcn\tau\subseteq\of{\T}{\C}}
     {\C\ts\letexp\ra\tau}\index{27.1}
\end{equation}}

%
\replacementPage{22}{\item{(\ref{varexp-rule}),(\ref{conexp-rule})}
The instantiation of 
type schemes allows different occurrences of a single $\longvar$ 
or $\longcon$ to assume different types.}{\item{(\ref{varexp-rule})}
The instantiation of 
type schemes allows different occurrences of a single $\longvid$ 
to assume different types. Note that the identifier status is not
used in this rule.}

%
\insertionPage{22}{The side condition
prevents type names generated by $\dec$ from escaping outside the local declaration.}

%
\replacementPage{23}{
\begin{equation}        % mrule
%\label{mrule-rule}
\frac{\C\ts\pat\ra(\VE,\tau)\qquad\C+\VE\ts\exp\ra\tau'}
     {\C\ts\longmrule\ \ra\tau\rightarrow\tau'}
\end{equation}}{\begin{equation}        % mrule
\label{mrule-rule}
\frac{\C\ts\pat\ra(\VE,\tau)\qquad\C+\VE\ts\exp\ra\tau'\qquad\TyNamesFcn\VE\subseteq\of{\T}{\C}}
     {\C\ts\longmrule\ \ra\tau\rightarrow\tau'}
\end{equation}}

%
\replacementPage{23}{
\begin{equation}        % value declaration
\label{valdec-rule}
\frac{\plusmap{\C}{\U}\ts\valbind\ra\VE\qquad 
      \VE'=\cl{\C,\valbind}{\VE}\qquad
      \U\cap\TyVarFcn\VE'=\emptyset}
     {\C\ts\valdecS\ra\VE'\ \In\ \Env}\index{28.2}
\end{equation}}{\begin{equation}        % value declaration
\label{valdec-rule}
\frac{\begin{array}{l}
      U = \TyVarsFcn (\tyvarseq)\\
     \plusmap{\C}{\U}\ts\valbind\ra\VE\qquad 
      \VE'=\cl{\C,\valbind}{\VE}\qquad
      \U\cap\TyVarFcn\VE'=\emptyset
      \end{array}}
     {\C\ts\boxml{val $\tyvarseq$ $\valbind$}\ra\VE'\ \In\ \Env}\index{28.2}
\end{equation}}

%
\replacementPage{23}{\CE}{\VE'}

%
\insertionPage{23}{\begin{equation}        % datatype replication
\label{datatyperepldec-rule}
\frac{\C(\longtycon) = (\typefcn,\VE)\quad
      \typefcn = \typefcnk\quad
      \tyvarseq = \alphak\quad
      \TE=\{\tycon\mapsto(\typefcn,\VE)\}
     }
     {\C\ts\datatyperepldec\ra(\VE,\TE)\ \In\ \Env}
\end{equation}}

%
\replacementPage{23}{\CE}{\VE'}

%
\replacementPage{23}{\begin{equation}        % exception declaration
\label{exceptiondec-rule}
\frac{\C\ts\exnbind\ra\EE\quad\VE=\EE }
     {\C\ts\exceptiondec\ra(\VE,\EE)\ \In\ \Env }
\end{equation}}{\begin{equation}        % exception declaration
\label{exceptiondec-rule}
\frac{\C\ts\exnbind\ra\VE}
     {\C\ts\exceptiondec\ra\VE\ \In\ \Env }
\end{equation}}

%
\replacementPage{24}{\begin{equation}                % open declaration
%\label{open-dec-rule}
\frac{ \C(\longstrid_1)=(\m_1,\E_1)
            \quad\cdots\quad
       \C(\longstrid_n)=(\m_n,\E_n) }
     { \C\ts\openstrdec\ra \E_1 + \cdots + \E_n }
\end{equation}}{\begin{equation}                % open declaration
%\label{open-dec-rule}
\frac{ \C(\longstrid_1)= \E_1 
            \quad\cdots\quad
       \C(\longstrid_n)= \E_n  }
     { \C\ts\openstrdec\ra \E_1 + \cdots + \E_n }
\end{equation}}

%
\replacementPage{24}{The closure of $\VE$ is exactly what allows variables to
be used polymorphically, via rule~\ref{varexp-rule}.}{The closure of $\VE$  
allows value identifiers  to
be used polymorphically, via rule~\ref{varexp-rule}.}

%
\replacementPage{24}{Moreover, $\U$ is the set of explicit type variables scoped at this particular
occurrence of $\valdec$, cf. Section~\ref{scope-sec}, 
page~\pageref{scope-def-lab}. The side-condition on $\U$
ensures that these explicit type variables are bound by the closure 
operation.}{The side-condition on $\U$
ensures that the type variables in $\tyvarseq$  are bound 
by the closure operation,
if they occur free in the range of $\VE$.}

%
\replacementPage{24}{On the other hand, no {\sl other} explicit type variable occurring
free in $\VE$ will become bound, since it must be in $\of{\U}{\C}$, and
is therefore excluded from closure by the definition of the closure operation
(Section~\ref{closure-sec}, page~\pageref{clos-def-lab})
since $\of{\U}{\C}\subseteq\TyVarFcn\C$.}{On the other hand,
if the phrase $\boxml{val}\,\tyvarseq\,\valbind$ occurs inside
some larger value binding $\boxml{val}\,\tyvarseq'\,\valbind'$
then no type variable $\alpha$ listed in $\tyvarseq'$ will become
bound by the $\cl{\C,\valbind}{\VE}$ operation; for $\alpha$ must 
be in $\of{\U}{\C}$ and hence excluded from closure by the definition of the closure operation
(Section~\ref{closure-sec}, page~\pageref{clos-def-lab})
since $\of{\U}{\C}\subseteq\TyVarFcn\C$.}

%
\insertionPage{24}{\item{(\ref{datatyperepldec-rule})}
Note that no new type name is generated (i.e., datatype replication is
not generative). 
By the syntactic restriction in Section~\ref{synres.sec} the two type
variable sequences in the conclusion must be equal.}

%
\replacementPage{24}{$\EE$}{$\VE$}

%
\replacementPage{24}{names}{constructors}

%
\deletionPage{24}{
Note that $\EE$ is also recorded in the {\VarEnv} component of
the resulting environment (see Section~\ref{stat-proj}, page~\pageref{stat-proj}).\index{29.0}}

%
\replacementPage{24}{
\begin{equation}        % recursive value binding
\label{recvalbind-rule}
\frac{\C+\VE\ts\valbind\ra\VE}
     {\C\ts\recvalbind\ra\VE}
\end{equation}}{\begin{equation}        % recursive value binding
\label{recvalbind-rule}
\frac{\C+\VE\ts\valbind\ra\VE\qquad\TyNamesFcn\VE\subseteq\of{\T}{\C}}
     {\C\ts\recvalbind\ra\VE}
\end{equation}}

%
\insertionPage{25}{Also note that $\C+\VE$ may overwrite 
identifier status. For example, the program
    {\tt datatype t = f; val rec f = fn x => x;}~~  is legal.}

%
\replacementPage{25}{\begin{equation}        % datatype binding
\frac{\begin{array}{c}
        \tyvarseq=\alphak\qquad\C,\alphakt\ts\constrs\ra\CE\\
        \langle\C\ts\datbind\ra\VE,\TE\qquad
        \forall(\t',\CE)\in\Ran\TE, \t\neq\t'\rangle
      \end{array}
     }
     {\begin{array}{c}
        \C\ts\longdatbind\ra\\
        \qquad\qquad\qquad\cl{}{\CE}\langle +\ \VE\rangle,\
        \{\tycon\mapsto(\t,\cl{}{\CE})\}\ \langle +\ \TE\rangle
      \end{array}
     }\index{30.1}
\end{equation}}{\begin{equation}        % datatype binding
\frac{\begin{array}{c}
        \tyvarseq=\alphak\qquad\C,\alphakt\ts\constrs\ra\VE\\
        \langle\C\ts\datbind'\ra\VE',\TE'\qquad
        \forall(\t',\VE'')\in\Ran\TE', \t\neq\t'\rangle
      \end{array}
     }
     {\begin{array}{l}
        \C\ts\newlongdatbind\ra\\
        \qquad(\cl{}{\VE}\langle +\ \VE'\rangle,\
        \{\tycon\mapsto(\t,\cl{}{\VE})\}\ \langle +\ \TE'\rangle
      \end{array}
     }\index{30.1}
\end{equation}}

%
\replacementPage{25}{The syntactic restrictions ensure $\Dom\VE\cap\Dom\CE = \emptyset$
and $\tycon\notin\Dom\TE$.}{The syntactic restrictions ensure $\Dom\VE\cap\Dom\VE' = \emptyset$
and $\tycon\notin\Dom\TE'$.}

%
\replacementPage{25}{\rulesec{Constructor Bindings}{\C,\tau\ts\constrs\ra\CE}}{\rulesec{Constructor Bindings}{\C,\tau\ts\constrs\ra\VE}}

%
\replacementPage{25}{\begin{equation}        % data constructors
%\label{constrs-rule}
\frac{\langle\C\ts\ty\ra\tau'\rangle\qquad
      \langle\langle\C,\tau\ts\constrs\ra\CE\rangle\rangle }
     {\begin{array}{c}
      \C,\tau\ts\longerconstrs\ra\\
      \qquad\qquad\qquad\{\con\mapsto\tau\}\
     \langle +\ \{\con\mapsto\tau'\to\tau\}\ \rangle\
      \langle\langle +\ \CE\rangle\rangle
      \end{array}
     }\index{30.2}
\end{equation}}{\begin{equation}        % data constructors
%\label{constrs-rule}
\frac{\langle\C\ts\ty\ra\tau'\rangle\qquad
      \langle\langle\C,\tau\ts\constrs\ra\VE\rangle\rangle }
     {\begin{array}{c}
      \C,\tau\ts\longervidconstrs\ra\\
      \qquad\qquad\qquad\{\vid\mapsto(\tau,\isc)\}\
     \langle +\ \{\vid\mapsto(\tau'\to\tau,\isc)\}\ \rangle\
      \langle\langle +\ \VE\rangle\rangle
      \end{array}
     }\index{30.2}
\end{equation}}

%
\replacementPage{25}{$\con\notin\Dom\CE$.}{$\vid\notin\Dom\VE$.}

%
\replacementPage{25}{
\rulesec{Exception Bindings}{\C\ts\exnbind\ra\EE}}{
\rulesec{Exception Bindings}{\C\ts\exnbind\ra\VE}}

%
\replacementPage{26}{\begin{equation}        % exception binding
\label{exnbind1-rule}
\frac{\langle\C\ts\ty\ra\tau\quad\mbox{$\tau$ is imperative}\rangle\qquad
      \langle\langle\C\ts\exnbind\ra\EE\rangle\rangle }
     {\begin{array}{c}
      \C\ts\longexnbinda\ra\\
      \qquad\qquad\qquad\{\exn\mapsto\EXCN\}\
      \langle +\ \{\exn\mapsto\tau\rightarrow\EXCN\}\ \rangle\
      \langle\langle +\ \EE\rangle\rangle
      \end{array}
     }\index{30.3}
\end{equation}}{\begin{equation}        % exception binding
\label{exnbind1-rule}
\frac{\langle\C\ts\ty\ra\tau\rangle\qquad
      \langle\langle\C\ts\exnbind\ra\VE\rangle\rangle }
     {\begin{array}{l}
      \C\ts\longvidexnbinda\ra\\
      \qquad\{\vid\mapsto(\EXCN,\ise)\}\
      \langle +\ \{\vid\mapsto(\tau\rightarrow\EXCN,\ise)\}\ \rangle\
      \langle\langle +\ \VE\rangle\rangle
      \end{array}
     }\index{30.3}
\end{equation}}

%
\replacementPage{26}{\begin{equation}        % exception binding
\label{exnbind2-rule}
\frac{\C(\longexn)=\tau\qquad
      \langle\C\ts\exnbind\ra\EE\rangle }
      {\C\ts\longexnbindb\ra\{\exn\mapsto\tau\}\ \langle +\ \EE\rangle}
\end{equation}}{\begin{equation}        % exception binding
\label{exnbind2-rule}
\frac{\C(\longvid)=(\tau,\ise)\qquad
      \langle\C\ts\exnbind\ra\VE\rangle }
      {\C\ts\longvidexnbindb\ra\{\vid\mapsto(\tau,\ise)\}\ \langle +\ \VE\rangle}
\end{equation}}

%
\replacementPage{26}{\item{(\ref{exnbind1-rule})} Notice that $\tau$ must not contain
any applicative type variables.}{\item{(\ref{exnbind1-rule})} Notice that $\tau$ may contain
type variables.}

%
\replacementPage{26}{
There is a unique $\EE$, for each $\C$ and $\exnbind$,
such that $\C\ts\exnbind\ra\EE$.}{For each $\C$ and $\exnbind$,
there is at most one $\VE$ satisfying $\C\ts\exnbind\ra\VE$.}

%
\replacementPage{26}{
\begin{equation}        % variable pattern
\label{varpat-rule}
\frac{}
     {\C\ts\var\ra (\{\var\mapsto\tau\},\tau) }
\end{equation}}{\begin{equation}        % variable pattern
\label{varpat-rule}
\frac{\hbox{$\vid\notin\Dom(\C)$ or $\of{\is}{C(\vid)} = \isv$}}
     {\C\ts\vid\ra (\{\vid\mapsto(\tau,\isv)\},\tau) }
\end{equation}}

%
\replacementPage{26}{
\begin{equation}        % constant pattern
%\label{constpat-rule}
\frac{\C(\longcon)\succ\tauk\t }
     {\C\ts\longcon\ra (\emptymap,\tauk\t)}
\end{equation}}{\begin{equation}        % constant pattern
\label{constpat-rule}
\frac{\C(\longvid)=(\sigma,\is)\qquad\is\neq\isv\qquad\sigma\succ\tauk\t }
     {\C\ts\longvid\ra (\emptymap,\tauk\t)}
\end{equation}}

%
\deletionPage{26}{
\begin{equation}       % exception constant
%\label{exconapat-rule}
\frac{\C(\longexn)=\EXCN}
     {\C\ts\longexn\ra (\emptymap,\EXCN)}
\end{equation}}

%
\replacementPage{26}{\item{(\ref{varpat-rule})} 
Note that $\var$ can assume a type, not a general type scheme.}{\item{(\ref{varpat-rule}), (\ref{constpat-rule})} The context $\C$ determines which of these
two rules applies. In rule~\ref{varpat-rule}, note that  
$\vid$ can assume a type, not a general type scheme.}

%
\replacementPage{27}{\begin{equation}        % record component
\label{longlab-rule}
\frac{\C\ts\pat\ra(\VE,\tau)\qquad
      \langle\C\ts\labpats\ra(\VE',\varrho)\qquad\lab\notin\Dom\varrho\rangle
}
     {\C\ts\longlabpats\ra
     (\VE\langle +\ \VE'\rangle,\
      \{\lab\mapsto\tau\}\langle +\ \varrho\rangle) }
\end{equation}}{\begin{equation}        % record component
\label{longlab-rule}
\frac{\begin{array}{c}\C\ts\pat\ra(\VE,\tau)\\
      \langle\C\ts\labpats\ra(\VE',\varrho)\qquad\Dom\VE\cap\Dom\VE' = \emptyset\qquad\lab\notin\Dom\varrho\rangle
      \end{array}
}
     {\C\ts\longlabpats\ra
     (\VE\langle +\ \VE'\rangle,\
      \{\lab\mapsto\tau\}\langle +\ \varrho\rangle) }
\end{equation}}

%
\deletionPage{27}{\comment 
\begin{description}
\item{(\ref{longlab-rule})} 
 By the syntactic restrictions, $\Dom\VE\cap\Dom\VE' = \emptyset$.
\end{description}}

%
\replacementPage{27}{
\begin{equation}        % construction pattern
%\label{conpat-rule}
\frac{\C(\longcon)\succ\tau'\to\tau\qquad\C\ts\atpat\ra(\VE,\tau')}
     {\C\ts\conpat\ra (\VE,\tau)}
\end{equation}}{\begin{equation}        % construction pattern
%\label{conpat-rule}
\frac{\C(\longvid) = (\sigma, \is)\qquad\is\neq\isv\qquad \sigma\succ\tau'\to\tau\qquad\C\ts\atpat\ra(\VE,\tau')}
     {\C\ts\vidpat\ra (\VE,\tau)}
\end{equation}}

%
\deletionPage{27}{
\begin{equation}       %  exception construction pattern
%\label{exconpat-rule}
\frac{\C(\longexn)=\tau\rightarrow\EXCN\qquad
      \C\ts\atpat\ra(\VE,\tau)}
     {\C\ts\exconpat\ra(\VE,\EXCN)}
\end{equation}}

%
\replacementPage{27}{
\begin{equation}        % layered pattern
\label{layeredpat-rule}
\frac{\begin{array}{c}
      \C\ts\var\ra(\VE,\tau)\qquad\langle\C\ts\ty\ra\tau\rangle\\
      \C\ts\pat\ra(\VE',\tau)
      \end{array}
     }
     {\C\ts\layeredpat\ra(\plusmap{\VE}{\VE'},\tau)}
\end{equation}}{\begin{equation}        % layered pattern
\label{layeredpat-rule}
\frac{\begin{array}{c}
     \hbox{$\vid\notin\Dom(\C)$ or $\of{\is}{C(\vid)} = \isv$}\\
      \langle\C\ts\ty\ra\tau\rangle\qquad
      \C\ts\pat\ra(\VE,\tau)\qquad \vid\notin\Dom\VE
      \end{array}
     }
     {\C\ts\layeredvidpat\ra(\plusmap{\{\vid\mapsto(\tau,\isv)\}}{\VE},\tau)}
\end{equation}}

%
\deletionPage{27}{\comments
\begin{description}
\item{(\ref{layeredpat-rule})}
By the syntactic restrictions, $\Dom\VE\cap\Dom\VE' = \emptyset$.
\end{description}}

%
\replacementPage{27}{\CE}{\VE}

%
\replacementPage{28}{record}{row}

%
\deletionPage{28}{This restriction is necessary to ensure the
existence of principal type schemes.}

%
\insertionPage{28}{-value}

%
\replacementPage{28}{$\var\ \atpat_1\ \cdots\ \atpat_n\langle : \ty\rangle$\ \ml{=}\ $\exp$}{$\vid\ \atpat_1\ \cdots\ \atpat_n\langle : \ty\rangle$\ \ml{=}\ $\exp$}

%
\deletionPage{28}{
\subsection{Principal Environments}
\label{principal-env-sec}
The\index{33.15} notion of {\sl enrichment}, $\E\succ\E'$, between environments
$\E=(\SE,\TE,\VE,\EE)$ and $\E'=(\SE',\TE',\VE',\EE')$ is defined
in Section~\ref{enrichment-sec}. For the present section,  $\E\succ\E'$
may be taken to mean $\SE=\SE'=\emptymap$, $\TE=\TE'$,
$\EE=\EE'$, $\Dom\VE=\Dom\VE'$ and, for each $\id\in\Dom\VE$,
$\VE(\id)\succ\VE'(\id)$.

Let\index{33.2} $\C$ be a context, and suppose that $\C\ts\dec\ra\E$
according to the preceding Inference Rules. Then $E$ is {\em principal}
(for $\dec$ in the context $\C$) if, for all $\E'$ for
which $\C\ts\dec\ra\E'$, we have $\E\succ\E'$. We claim that if
$\dec$ elaborates to any environment in $\C$ then it elaborates to
a principal environment in $\C$. Strictly, we must allow for the
possibility that type names and imperative
 type variables
which do not occur in $\C$ are chosen
differently for $\E$ and $\E'$. 
The stated claim is therefore made up to such variation.
}

%
\replacementPage{29}{
\begin{displaymath}
\begin{array}{rcl}
\M              & \in   & \StrNameSets = \Fin(\StrNames)\\
\N\ {\rm or}\ (\M,\T)
                & \in   & \NameSets = \StrNameSets\times\TyNameSets\\
\sig\ {\rm or}\ \longsig{}
                & \in   & \Sig =  \NameSets\times\Str \\
\funsig\ {\rm or}\ \longfunsig{}
                & \in   & \FunSig = \NameSets\times
                                         (\Str\times\Sig)\\
\G              & \in   & \SigEnv        =       \finfun{\SigId}{\Sig} \\
\F              & \in   & \FunEnv        =       \finfun{\FunId}{\FunSig} \\
\B\ {\rm or}\ \N,\F,\G,\E
                & \in   & \Basis = \NameSets\times
                                              \FunEnv\times\SigEnv\times\Env\\
\end{array}
\end{displaymath}}{\begin{displaymath}
\begin{array}{rcl}
\sig\ {\rm or}\ \newlongsig{}
                & \in   & \Sig =  \TyNameSets\times\Env \\
\funsig\ {\rm or}\ \newlongfunsig{}
                & \in   & \FunSig = \TyNameSets\times
                                         (\Env\times\Sig)\\
\G              & \in   & \SigEnv        =       \finfun{\SigId}{\Sig} \\
\F              & \in   & \FunEnv        =       \finfun{\FunId}{\FunSig} \\
\B\ {\rm or}\ \T,\F,\G,\E
                & \in   & \Basis = \TyNameSets\times
                                              \FunEnv\times\SigEnv\times\Env\\
\end{array}
\end{displaymath}}

%
\replacementPage{29}{
The prefix $(\N)$, in signatures and functor signatures, binds both type names
and structure names. We shall always consider a set $\N$ of names as
partitioned into a pair $(\M,\T)$ of sets of the two kinds of name.}{The 
prefix $(\T)$, in signatures and functor signatures, binds  type names.}

%
\deletionPage{29}{It is sometimes convenient to work with an arbitrary semantic object $A$, or
assembly $A$ of such objects.
As with the function $\TyNamesFcn$,
$\StrNamesFcn(A)$ and $\NamesFcn(A)$ denote respectively the set of structure names
and the set of names occurring free in $A$.}

%
\replacementPage{29}{Section~\ref{realisation-sec}}{Section~\ref{tyrea.sec}}

%
\deletionPage{29}{imperative, }

%
\deletionPage{29}{For any structure $\S=\longS{}$ we call $m$ the {\sl structure name} or
{\sl name} of $\S$; also, the {\sl proper substructures} of $\S$ are
the members of $\Ran\SE$ and their proper substructures.  The 
{\sl substructures} of
$\S$ are $\S$ itself and its proper substructures.  The structures
{\sl occurring in}
an object or assembly $A$ are the structures and
substructures from which it is built.}

%
\replacementPage{29}{$\of{\N}{\B}$}{$\of{\T}{\B}$}

%
\deletionPage{29}{and
structure names }

%
\replacementPage{29}{$\B+(\NamesFcn\SE,\SE)$}{$\B+(\TyNamesFcn\SE,\SE)$}

%
\insertionPage{29}{
There is no separate kind of semantic object to represent structures: 
structure expressions elaborate to environments, just as structure-level
declarations do. Thus, notions which are commonly associated with structures
(for example the notion of matching a structure against a signature) are defined
in terms of environments.}

%
\deletionPage{29}{
\subsection{Consistency}
\label{consistency-sec}
A\index{35.1} set of type structures is said to be {\sl consistent} if, for all
$(\theta_1,\CE_1)$ and $(\theta_2,\CE_2)$ in the set, if $\theta_1 = \theta_2$
then
\[\CE_1=\emptymap\ {\rm or}\ 
\CE_2=\emptymap\ {\rm or}\ \Dom\CE_1=\Dom\CE_2\]
A semantic object $A$ or assembly $A$ of objects is said to be
{\sl consistent} if (after changing bound names to make all nameset prefixes
in $A$ disjoint) 
for all $\S_1$ and
$\S_2$ occurring in $A$ and for every $\longstrid$ 
and every $\longtycon$
\begin{enumerate}
\item If $\of{\m}{\S_1}=\of{\m}{\S_2}$, and both
      $\S_1(\longstrid)$ and $\S_2(\longstrid)$ exist, then
      \[ \of{\m}{\S_1(\longstrid)}\ =\ \of{\m}{\S_2(\longstrid)}\]

\item If $\of{\m}{\S_1}=\of{\m}{\S_2}$, and both
      $\S_1(\longtycon)$ and $\S_2(\longtycon)$ exist, then
      \[ \of{\theta}{\S_1(\longtycon)}\ =\ \of{\theta}{\S_2(\longtycon)}\]

\item The set of all type structures in $A$ is consistent
\end{enumerate}

As an example, a functor signature 
$\longfunsig{}$ is
consistent if, assuming first that 
$\N\cap\N'=\emptyset$,
the assembly $A=\{\S,\S'\}$ is consistent.

We may loosely say that two 
structures $\S_1$ and $\S_2$
are consistent if
$\{\S_1,\S_2\}$ is consistent, but must remember that this is stronger than
the assertion that $\S_1$ is consistent and $\S_2$ is consistent.

Note that if $A$ is a consistent assembly and $A'\subset A$ then $A'$ is
also a consistent assembly.
}

%
\deletionPage{29}{
\subsection{Well-formedness}
A signature\index{35.2} $\longsig{}$ is {\sl well-formed} 
if $\N\subseteq\NamesFcn\S$,
and also, whenever $(\m,\E)$ is a
substructure of $\S$ and $\m\notin\N$, then $\N\cap(\NamesFcn\E)=\emptyset$.
A functor signature $\longfunsig{}$ is {\sl well-formed} if
$\longsig{}$ and  $(\N')\S'$ are well-formed, and also, whenever
$(\m',\E')$ is a substructure of $\S'$ and $\m'\notin\N\cup\N'$,
then $(\N\cup\N')\cap(\NamesFcn\E')=\emptyset$.

An object or assembly $A$ is {\sl well-formed} if every type environment,
signature and functor signature occurring in $A$ is well-formed.}

%
\deletionPage{29}{\subsection{Cycle-freedom}
An\index{35.3} object or assembly $A$ is {\sl cycle-free} if it contains no
cycle of structure names; that is, there is no sequence
\[\m_0,\cdots,\m_{k-1},\m_k=m_0\ \ (k>0)\]
of structure names such that, for each $i\ (0\leq i<k)$ some structure
with name $m_i$ occurring in $A$ has a proper substructure with name
$m_{i+1}$.
}

%
\deletionPage{29}{
\subsection{Admissibility}
\label{admis-sec}
An\index{36.1} object or assembly $A$ is {\sl admissible} if it is
consistent, well-formed and cycle-free. 
Henceforth it is assumed
that
all objects mentioned are admissible.  
We also require that
\begin{enumerate}
\item In every sentence $A\ts\phrase\ra A'$  inferred by the rules
given in Section~\ref{statmod-rules-sec}, the assembly $\{A,A'\}$ is
admissible.  
\item In the special case of a sentence $\B\ts\sigexp\ra\S$,
we further require that the assembly consisting of all semantic
objects occurring in the entire inference of this sentence be
admissible. This  is important for the definition of principal
signatures in Section~\ref{prinsig-sec}.
\end{enumerate}
In our semantic definition we have not undertaken to
indicate how admissibility should be checked in an implementation.
}

%
\replacementPage{29}{
A {\sl type realisation}\index{36.2} is a map
$\tyrea:\TyNames\to\TypeFcn$
such that
$\t$ and $\tyrea(\t)$ have the same arity, and
if $t$ admits equality then so does $\tyrea(\t)$.

The {\sl support} $\Supp\tyrea$ of a type realisation $\tyrea$ is the set of
type names $\t$ for which $\tyrea(\t)\ne\t$.}{A 
{\sl (type) realisation}\index{36.2} is a map
$\rea:\TyNames\to\TypeFcn$
such that
$\t$ and $\rea(\t)$ have the same arity, and
if $t$ admits equality then so does $\rea(\t)$.

The {\sl support} $\Supp\rea$ of a type realisation $\rea$ is the set of
type names $\t$ for which $\rea(\t)\ne\t$.}

%
\deletionPage{29}{\subsection{Realisation}
\label{realisation-sec}
A {\sl realisation}\index{36.3} is a function $\rea$ of names,
partitioned into a type realisation $\tyrea:\TyNames\to\TypeFcn$
and a function $\strrea : \StrNames\to\StrNames$.
The {\sl support} $\Supp\rea$
of a realisation $\rea$ is the set of
names $\n$ for which $\rea(\n)\ne\n$.}

%
\replacementPage{29}{The {\sl yield}
$\Yield\rea$ of a realisation $\rea$ is the set of
names which occur in some $\rea(\n)$ for which $\n\in\Supp\rea$.}{The
{\sl yield} $\Yield\rea$ of a realisation $\rea$ is the set of
type names which occur in some $\rea(\t)$ for which $\t\in\Supp\rea$.}

%
\replacementPage{29}{$n$ by $\rea(\n)$}{$\t$ by $\rea(\t)$}

%
\replacementPage{30}{$\longsig{}$}{$\newlongsig{}$}

%
\replacementPage{30}{$(\N)$}{$(\T)$}

%
\replacementPage{30}{
\[\N\cap(\Supp\rea\cup\Yield\rea)=\emptyset\ .\]}{\[\T\cap(\Supp\rea\cup\Yield\rea)=\emptyset\ .\]}

%
\deletionPage{30}{
\subsection{Type Explication}
\label{type-explication-sec}
A\index{36.35} signature $(\N)\S$ is {\sl type-explicit\/} if,
whenever $\t\in\N$ and occurs free in $\S$, then some substructure of
$\S$ contains a type environment $\TE$ such that
$\TE(\tycon)=(\t,\CE)$ for some $\tycon$ and some $\CE$.}

%
\replacementPage{30}{
A\index{36.4} structure $\S_2$ {\sl is an instance of} a signature
$\sig_1=\longsig{1}$,
written $\siginst{\sig_1}{}{\S_2}$, if there exists a realisation
$\rea$
such that $\rea(\S_1)=\S_2$ and $\Supp\rea\subseteq\N_1$.}{An\index{36.4} environment $\E_2$ {\sl is an instance of} a signature
$\sig_1=\newlongsig{1}$,
written $\siginst{\sig_1}{}{\E_2}$, if there exists a realisation
$\rea$
such that $\rea(\E_1)=\E_2$ and $\Supp\rea\subseteq\T_1$.}

%
\deletionPage{30}{(Note that if $\sig_1$ is type-explicit then there is at most one
such $\rea$.)}

%
\deletionPage{30}{A signature
$\sig_2=\longsig{2}$ {\sl is an instance of}
$\sig_1 =\longsig{1}$,
written $\siginst{\sig_1}{}{\sig_2}$, if
$\siginst{\sig_1}{}{\S_2}$ and $\N_2\cap(\NamesFcn\sig_1)=\emptyset$.
It can be shown that $\siginst{\sig_1}{}{\sig_2}$ iff, for all $\S$,
whenever $\siginst{\sig_2}{}{\S}$ then $\siginst{\sig_1}{}{\S}$.}

%
\replacementPage{30}{
A\index{36.5} pair $(\S,(\N')\S')$ is called a {\sl functor instance}.
Given $\funsig=\longfunsig{1}$,
a functor instance $(\S_2,(\N_2')\S_2')$ is an {\sl instance} of
$\funsig$,
written $\funsiginst{\funsig}{}{(\S_2,(\N_2')\S_2')}$,
if there exists a realisation $\rea$
such that
$\rea(\S_1,(\N_1')\S_1')=(\S_2,(\N_2')\S_2')$ and
$\Supp\rea\subseteq\N_1$.}{
A\index{36.5} pair $(\E,(\T')\E')$ is called a {\sl functor instance}.
Given $\funsig=\newlongfunsig{1}$,
a functor instance $(\E_2,(\T_2')\E_2')$ is an {\sl instance} of
$\funsig$,
written $\funsiginst{\funsig}{}{(\E_2,(\T_2')\E_2')}$,
if there exists a realisation $\rea$
such that
$\rea(\E_1,(\T_1')\E_1')=(\E_2,(\T_2')\E_2')$ and
$\Supp\rea\subseteq\T_1$.}

%
\replacementPage{30}{a structure}{an environment}

%
\replacementPage{30}{structure}{environment}

%
\deletionPage{30}{structures, }

%
\replacementPage{30}{by mutual recursion}{recursively}

%
\deletionPage{30}{A structure $\S_1=(\m_1,\E_1)$
{\sl enriches} another structure
$\S_2=(\m_2,\E_2)$, written $\S_1\succ\S_2$, if
\begin{enumerate}
\item $\m_1=\m_2$
\item $\E_1\succ\E_2$
\end{enumerate}}

%
\replacementPage{30}{$\E_1=\longE{1}$}{$\E_1=\newlongE{1}$}

%
\replacementPage{30}{$\E_2=$ $\longE{2}$}{$\E_2=
( \SE_2,$\linebreak$\TE_2,\VE_2)$}

%
\replacementPage{30}{$\Dom\VE_1\supseteq\Dom\VE_2$, and $\VE_1(\id)\succ\VE_2(\id)$
                                               for all $\id\in\Dom\VE_2$}{$\Dom\VE_1\supseteq\Dom\VE_2$, and $\VE_1(\vid)\succ\VE_2(\vid)$
                                               for all $\vid\in\Dom\VE_2$,
where $(\sigma_1,\is_1)\succ(\sigma_2,\is_2)$ means $\sigma_1\succ\sigma_2$ and
$$\is_1 = \is_2\quad\hbox{or}\quad \is_2 = \isv$$}

%
\deletionPage{30}{
\item $\Dom\EE_1\supseteq\Dom\EE_2$, and $\EE_1(\exn)=\EE_2(\exn)$
                                               for all $\exn\in\Dom\EE_2$}

%
\replacementPage{30}{\CE}{\VE}

%
\replacementPage{30}{\CE}{\VE}

%
\replacementPage{30}{\CE}{\VE}

%
\replacementPage{30}{\CE}{\VE}

%
\replacementPage{30}{\CE}{\VE}

%
\replacementPage{30}{\CE}{\VE}

%
\replacementPage{30}{\CE}{\VE}

%
\replacementPage{30}{
A\index{37.2} structure $\S$ {\sl matches} a signature $\sig_1$ if there exists
a structure $\S^-$ such that $\sig_1\geq\S^-\prec\S$. Thus matching
is a combination of instantiation and enrichment. There is at most
one such $\S^-$, given $\sig_1$ and $\S$.}{An\index{37.2} environment $\E$ {\sl matches} a signature $\sig_1$ if there exists
an environment $\E^-$ such that $\sig_1\geq\E^-\prec\E$. Thus matching
is a combination of instantiation and enrichment. There is at most
one such $\E^-$, given $\sig_1$ and $\E$.}

%
\deletionPage{30}{Moreover, writing $\sig_1=
\longsig{1}$, if $\sig_1\geq\S^-$ then there exists a realisation $\rea$
with $\Supp\rea\subseteq\N_1$ and $\rea(\S_1)=\S^-$.
We shall then say that $\S$ matches $\sig_1$ {\em via} $\rea$.
(Note that if $\sig_1$ is type-explicit 
then $\rea$ is uniquely determined by $\sig_1$ and $\S$.)}

%
\deletionPage{30}{A\index{37.2.5} signature $\sig_2$ {\em matches} a signature $\sig_1$
if for all structures $\S$, if $\S$ matches $\sig_2$ then $\S$
matches $\sig_1$. It can be shown that $\sig_2=\longsig{2}$ matches
$\sig_1=\longsig{1}$ if and only if there exists a realisation
$\rea$ with $\Supp\rea\subseteq\N_1$ and $\rea(\S_1)\prec\S_2$
and $\N_2\cap\NamesFcn\sig_1=\emptyset$.}

%
\deletionPage{30}{\subsection{Principal Signatures}
\label{prinsig-sec}
The definitions in this section concern the elaboration of signature
expressions; more precisely they concern inferences of sentences of the
form $\B\ts\sigexp\ra\S$, where $\S$ is a structure and $\B$ is a basis.
Recall, from Section~\ref{admis-sec}, that the assembly of all semantic
objects in such an inference must be admissible.

For any basis $\B$ and any structure $\S$, 
we say that $\B$ {\sl covers} $\S$
if for every substructure $(m,E)$ of $\S$ such that
$m\in\of{\N}{\B}$:
\begin{enumerate}
\item
For every structure identifier $\strid\in\Dom\E$,
$\B$ contains a substructure $(m,\E')$ with $m$
free and $\strid\in\Dom\E'$
\item
For every type constructor $\tycon\in\Dom\E$,
$\B$ contains a substructure $(m,\E')$ with $m$ free
and $\tycon\in\Dom\E'$
\end{enumerate}
(This condition is not a consequence of consistency of $\{\B,\S\}$; 
informally, it states that if $\S$ shares a substructure with $\B$,
then $\S$ mentions no more components of the substructure than
$\B$ does.)



We\index{38.1} say that a signature
$\longsig{}$ is {\sl principal for $\sigexp$ in $\B$} if, choosing $\N$
so that $(\of{\N}{\B})\cap\N=\emptyset$,
\begin{enumerate}
\item $\B$ covers $\S$ 
\item $\B\vdash\sigexp\ra\S$
\item Whenever $\B\vdash\sigexp\ra\S'$, then $\sigord{\longsig{}}{}{\S'}$
\end{enumerate}
We claim that if $\sigexp$ elaborates in $\B$ to some structure covered
by $\B$, then it possesses a principal signature in $\B$.

Analogous to the definition given for type environments in
Section~\ref{typeenv-wf-sec}, we say that a semantic object $A$
{\sl respects equality} if every type environment occurring in 
$A$ respects equality. 
%
%
%Further, let $T$ be the set of type names
%$\t$ such that $(\t,\CE)$ occurs in $A$ for some
%$\CE\neq\emptymap$.  Then $A$ is said to {\sl maximise equality}
%if (a) $A$ respects equality, and also (b) if any larger subset of
%$T$ were to admit equality (without any change in the equality
%attribute of any type names not in $T$) then $A$ would cease to
%respect equality.
%
\oldpagebreak
Now\index{38.5} let us assume that $\sigexp$ possesses a principal signature
$\sig_0=\longsig{0}$ in $B$. We wish to
define, in terms of $\sig_0$, another signature $\sig$ which provides more
information about the equality attributes of structures which will
match $\sig_0$. To this end, let $\T_0$ be the set of type names $\t\in\N_0$
which do not admit equality, and such that $(\t,\CE)$ occurs in $\S_0$
for some $\CE\neq\emptymap$.  Then we say $\sig$ is 
{\sl equality-principal for $\sigexp$ in $\B$} if
\begin{enumerate}
\item
$\sig$ respects equality
\item
$\sig$ is obtained from $\sig_0$ just by making as many
members of $\T_0$ admit equality as possible, subject to 1.~above
\end{enumerate}
It is easy to show that, if any such $\sig$ exists, it is determined
uniquely by $\sig_0$; moreover, $\sig$ exists if $\sig_0$ itself
respects equality.
\bigskip}

%
\replacementPage{31}{$\B=\N,\F,\G,\E$}{$\B=\T,\F,\G,\E$}

%
\replacementPage{31}{$\NamesFcn\F\ \cup\NamesFcn\G\cup\NamesFcn\E\subseteq\N$}{$\TyNamesFcn\F
\ \cup\TyNamesFcn\G\cup\TyNamesFcn\E\subseteq\T$}

%
\replacementPage{31}{
This is not
the case for bases in which signature expressions and specifications are
elaborated, but the following Theorem can be proved:}{
The following Theorem can be proved:}

%
\replacementPage{31}{
Moreover, if S$'$ is a sentence of the form
$\B''\ts\phrase\ra A$ occurring in a proof of S, where $\phrase$ is
either a structure expression or a structure-level declaration, then $\B''$
also satisfies the condition.}{
Moreover, if S$'$ is a sentence of the form
$\B''\ts\phrase\ra A$ occurring in a proof of S, where $\phrase$ is
any Modules phrase, then $\B''$ also satisfies the condition.}

%
\replacementPage{31}{Finally, if $\T,\U,\E\ts\phrase\ra A$ occurs
in a proof of S, where $\phrase$ is a phrase of the Core, then
$\TyNamesFcn\E\subseteq\T$.}{Finally, if $\T,\U,\E\ts\phrase\ra A$ occurs
in a proof of S, where $\phrase$ is a phrase of Modules or of the Core, then
$\TyNamesFcn\E\subseteq\T$.}

%
\replacementPage{31}{
\rulesec{Structure Expressions}{\B\ts\strexp\ra \S}}
{\rulesec{Structure Expressions}{\B\ts\strexp\ra \E}}

%
\replacementPage{31}{\begin{equation}        % generative strexp
\label{generative-strexp-rule}
\frac{\B\ts\strdec\ra\E\qquad\m\notin(\of{\N}{\B})\cup\NamesFcn\E}
     {\B\ts\encstrexp\ra(\m,\E)}\index{39.2}
\end{equation}}{\begin{equation}        % generative strexp
\label{generative-strexp-rule}
\frac{\B\ts\strdec\ra\E}
     {\B\ts\encstrexp\ra \E }\index{39.2}
\end{equation}}

%
\replacementPage{31}{
\begin{equation}        % longstrid
%\label{longstrid-strexp-rule}
\frac{\B(\longstrid)=\S}
     {\B\ts\longstrid\ra\S}
\end{equation}}{\begin{equation}        % longstrid
%\label{longstrid-strexp-rule}
\frac{\B(\longstrid)=\E}
     {\B\ts\longstrid\ra\E}
\end{equation}}

%
\insertionPage{31}{
\begin{equation}
\label{transparent-constraint-rule}
\frac{B\ts\strexp\ra\E\quad\B\ts\sigexp\ra\Sigma\quad\Sigma\geq\E'\prec\E}
     {\B\ts\transpconstraint\ra\E'}
\end{equation}
}

%
\insertionPage{31}{
\begin{equation}
\label{opaque-constraint-rule}
\frac{\begin{array}{c}
   B\ts\strexp\ra\E\quad\B\ts\sigexp\ra(\T')\E'\\
   (\T')\E'\geq\E''\prec\E\quad \T' \cap(\of{\T}{\B}) = \emptyset
      \end{array}}
     {\B\ts\opaqueconstraint\ra\E'}
\end{equation}
}

%
\replacementPage{31}{
\begin{equation}                % functor application
\label{functor-application-rule}
\frac{ \begin{array}{c}
        \B\ts\strexp\ra\S\\
        \funsiginst{\B(\funid)}{}{(\S'',(\N')\S')}\ ,
                                                    \ \S\succ\S''\\
        (\of{\N}{\B})\cap\N'=\emptyset
       \end{array}
     }
     {\B\ts\funappstr\ra\S'}
\end{equation}}{\begin{equation}                % functor application
\label{functor-application-rule}
\frac{ \begin{array}{c}
        \B\ts\strexp\ra\E\\
        \funsiginst{\B(\funid)}{}{(\E'',(\T')\E')}\ ,
                                                    \ \E\succ\E''\\
        (\TyNamesFcn \E\; \cup\; \of{\T}{\B})\cap\T'=\emptyset
       \end{array}
     }
     {\B\ts\funappstr\ra\E'}
\end{equation}}

%
\replacementPage{31}{
\begin{equation}        % let strexp
\label{letstrexp-rule}
\frac{\B\ts\strdec\ra\E\qquad\B\oplus\E\ts\strexp\ra\S}
     {\B\ts\letstrexp\ra\S}
\end{equation}}{\begin{equation}        % let strexp
\label{letstrexp-rule}
\frac{\B\ts\strdec\ra\E_1\qquad\B\oplus\E_1\ts\strexp\ra\E_2}
     {\B\ts\letstrexp\ra\E_2}
\end{equation}}

%
\deletionPage{32}{
\item{(\ref{generative-strexp-rule})}
   The side condition ensures that each generative structure
expression receives a new name. If the expression occurs in
a functor body the structure name will be bound by $(\N')$ in
rule~\ref{funbind-rule}; this will ensure that for each application of the 
functor, by rule~\ref{functor-application-rule}, a new distinct name
will be chosen for the structure generated.}

%
\replacementPage{32}{$ (\of{\N}{\B})\cap\N'=\emptyset$}{$(\TyNamesFcn\E \cup \of{\T}{\B})\cap\T'=\emptyset$}

%
\replacementPage{32}{$(\N')S'$}{$(\T')E'$}

%
\replacementPage{32}{structures}{datatypes}

%
\replacementPage{32}{Let $\B(\funid)=(N)(\S_f,(N')\S_f')$.}{Let $\B(\funid)=(\T)(\E_f,(T')\E_f')$.}

%
\replacementPage{32}{Assuming that $(\N)\S_f$ is
type-explicit, the realisation $\rea$ for which
$\rea(\S_f,(N')\S_f')=(\S'',(\N')\S')$ is uniquely determined by $\S$,
since $\S\succ\S''$ can only hold if the type names and structure
names in $\S$ and $\S''$ agree.  Recall that enrichment $\succ$ allows
more components and more polymorphism, while instantiation $\geq$ does
not.\par}{Let $\rea$ be a realisation such that\linebreak
$\rea(\E_f,(T')\E_f')=(\E'',(\T')\E')$.}

%
\replacementPage{32}{
Sharing between argument and result specified in the declaration of
the functor $\funid$ is represented by the occurrence of the same name
in both $\S_f$ and $\S_f'$, and this repeated occurrence is preserved
by $\rea$, yielding sharing between the argument structure $\S$ and
the result structure $\S'$ of this functor application.}{
Sharing between argument and result specified in the declaration of
the functor $\funid$ is represented by the occurrence of the same name
in both $\E_f$ and $\E_f'$, and this repeated occurrence is preserved
by $\rea$, yielding sharing between the argument structure $\E$ and
the result structure $\E'$ of this functor application.}

%
\deletionPage{32}{structure
and }

%
\replacementPage{32}{
\begin{equation}                % core declaration
\label{dec-rule}
\frac{ \of{\C}{\B}\ts\dec\ra\E
       \quad\E\ {\rm principal\ for\ \dec\ in\ } (\of{\C}{\B})
}
     { \B\ts\dec\ra\E }\index{40.2}
\end{equation}}{\begin{equation}                % core declaration
\label{dec-rule}
\frac{ \of{\C}{\B}\ts\dec\ra\E
}
     { \B\ts\dec\ra\E }\index{40.2}
\end{equation}}

%
\deletionPage{32}{
\comments
\begin{description}
\item{(\ref{dec-rule})}
The side condition ensures that all type schemes in $\E$ are as
general as possible.
% and that no imperative type variables occur
%free in $\E$.
%from version 1:
%   The side condition ensures that all type schemes in $\E$ are as
%general as possible and that all new type names in $\E$ admit
%equality, if possible.
\end{description}}

%
\replacementPage{32}{
\begin{equation}                % structure binding
\label{structure-binding-rule}
\frac{ \begin{array}{cl}
       \B\ts\strexp\ra\S\qquad\langle\B\ts\sigexp\ra\sig\ ,
                                      \ \sig\geq\S'\prec\S\rangle\\
       \langle\langle\plusmap{\B}{\TyNamesFcn\S}\ts
                                      \strbind\ra\SE\rangle\rangle
       \end{array}
     }
     { \B\ts\strbinder\ra\{\strid\mapsto\S\langle'\rangle\}
       \ \langle\langle +\ \SE\rangle\rangle }\index{41.1}
\end{equation}}{\begin{equation}                % structure binding
\label{structure-binding-rule}
\frac{ 
       \B\ts\strexp\ra\E\quad
       \langle\plusmap{\B}{\TyNamesFcn\E}\ts
                                      \strbind\ra\SE\rangle
     }
     { \B\ts\barestrbindera\ra\{\strid\mapsto\E\}
       \ \langle +\ \SE\rangle}\index{41.1}
\end{equation}}

%
\deletionPage{32}{ 
\comment If present, $\sigexp$ has the effect of restricting the
view which $\strid$ provides of $\S$ while retaining sharing of names.
The notation $\S\langle'\rangle$ means $\S'$, if the first option is present,
and $\S$ if not.}

%
\replacementPage{32}{
\rulesec{Signature Expressions}{\B\ts\sigexp\ra\S}
\begin{equation}                % encapsulation sigexp
\label{encapsulating-sigexp-rule}
\frac{\B\ts\spec\ra\E }
     {\B\ts\encsigexp\ra  (\m,\E) }\index{41.2}
\end{equation}}{\rulesec{Signature Expressions}{\B\ts\sigexp\ra\E}
\begin{equation}                % encapsulation sigexp
\label{encapsulating-sigexp-rule}
\frac{\B\ts\spec\ra\E }
     {\B\ts\encsigexp\ra  \E }\index{41.2}
\end{equation}}

%
\replacementPage{33}{\begin{equation}                % signature identifier
\label{signature-identifier-rule}
\frac{ \sigord{\B(\sigid)}{}{\S} }
     { \B\ts\sigid\ra\S }
\end{equation}}{\begin{equation}                % signature identifier
\label{signature-identifier-rule}
\frac{  \B(\sigid) = (\T)\E \quad \T\cap (\of{\T}{\B}) = \emptyset}
     { \B\ts\sigid\ra\E }
\end{equation}}

%
\insertionPage{33}{
\begin{equation}
\label{wheretype-rule}
\frac{
  \begin{array}{c}
     \B\ts\sigexp\ra \E\quad \tyvarseq = \alphak\quad \of{\C}{\B}\ts\ty\ra \tau\\
     \E(\longtycon) = (\t, \VE)\quad t\notin(\of{\T}{\B})\cup\TyNamesFcn\tau\\
     \rea = \{\t\mapsto \Lambda\alphak.\tau\}\quad
     \hbox{$\Lambda\alphak.\tau$ admits equality, if $\t$ does\quad $\rea(\E)$ well-formed}
  \end{array}
 }
 {\B\ts\wheretypesigexp\ra\rea(\E)}
\end{equation}}

%
\deletionPage{33}{
\item{(\ref{encapsulating-sigexp-rule})}
   In contrast to rule~\ref{generative-strexp-rule}, $m$ is not here 
required to be new. 
The name $m$ may be chosen to achieve the sharing required
in rule~\ref{strshareq-rule}, or to achieve the enrichment side conditions
of rule~\ref{structure-binding-rule} or \ref{funbind-rule}. 
The choice of $m$ must result in an admissible object.}

%
\replacementPage{33}{The instance $\S$ of $\B(\sigid)$ is not determined by this rule,
but -- as in rule~\ref{encapsulating-sigexp-rule} -- the instance
may  be chosen to achieve sharing properties or enrichment
conditions.}{The bound names of $\B(\sigid)$ can always be renamed to satisfy $\T\cap(\of{\T}{\B}) = \emptyset$,
if necessary.}

%
\replacementPage{33}{\begin{equation}                % any sigexp
\label{topmost-sigexp-rule}
\frac{\begin{array}{c}
\B\ts\sigexp\ra\S\quad\mbox{$(\N)\S$ equality-principal for $\sigexp$ in $\B$}\\
\mbox{$(\N)\S$ type-explicit}
      \end{array}}
     {\B\ts\sigexp\ra (\N)\S}\index{41.25}
\end{equation}}{\begin{equation}                % any sigexp
\label{topmost-sigexp-rule}
\frac{
\B\ts\sigexp\ra\E\quad\T= \TyNamesFcn\E\setminus(\of{\T}{\B})
}
     {\B\ts\sigexp\ra (\T)\E}\index{41.25}
\end{equation}}

%
\deletionPage{33}{a structure binding, }

%
\replacementPage{33}{, a 
functor binding or a
functor signature}{, a signature constraint, or a
functor binding }

%
\replacementPage{33}{an equality-principal and type-explicit }{a }

%
\replacementPage{33}{\ref{structure-binding-rule}, }{
\ref{transparent-constraint-rule}, \ref{opaque-constraint-rule}, }

%
\deletionPage{33}{, 
\ref{funsigexp-rule}}

%
\deletionPage{33}{By contrast, signature 
expressions occurring in structure descriptions are elaborated to
structures using the liberal rules
\ref{encapsulating-sigexp-rule} and \ref{signature-identifier-rule}, 
see rule~\ref{strdesc-rule}, so that names can be chosen to achieve
sharing, when necessary.}

%
\deletionPage{33}{
\begin{equation}        % empty signature declaration
%\label{empty-sigdec-rule}
\frac{}
     { \B\ts\emptysigdec\ra\emptymap }
\end{equation}

\begin{equation}        % sequential signature declaration
\label{sequence-sigdec-rule}
\frac{ \B\ts\sigdec_1\ra\G_1 \qquad \plusmap{\B}{\G_1}\ts\sigdec_2\ra\G_2 }
     { \B\ts\seqsigdec\ra\plusmap{\G_1}{\G_2} }
\end{equation}}

%
\deletionPage{33}{\comments
\begin{description}
%
\item{(\ref{single-sigdec-rule})}
The first closure restriction of Section~\ref{closure-restr-sec}
can be  enforced by replacing the $\B$ in the premise by $\B_0+\of{\G}{\B}$.
\item{(\ref{sequence-sigdec-rule})}
   A signature declaration does not create any new structures
or types; hence the use of $+$ instead of $\oplus$.
\end{description}
}

%
\deletionPage{33}{\comment The  condition that $\sig$ be equality-principal,
implicit in the first premise, ensures that the
signature found is as general as possible given the sharing
constraints present in $\sigexp$.}

%
\replacementPage{33}{
\begin{equation}        % type specification
\label{typespec-rule}
\frac{ \of{\C}{\B}\ts\typdesc\ra\TE }
     { \B\ts\typespec\ra\TE\ \In\ \Env }
\end{equation}}{\begin{equation}        % type specification
\label{typespec-rule}
\frac{ 
         \of{\C}{\B}\ts\typdesc\ra\TE \quad
         \forall(\t,\VE)\in\Ran\TE,\hbox{\ $t$ does not admit equality}
     }
     { \B\ts\typespec\ra\TE\ \In\ \Env }
\end{equation}}

%
\replacementPage{34}{
\begin{equation}        % eqtype specification
\label{eqtypspec-rule}
\frac{ \of{\C}{\B}\ts\typdesc\ra\TE \qquad
       \forall(\theta,\CE)\in \Ran\TE,\ \theta {\rm\ admits\ equality} }
     { \B\ts\eqtypespec\ra\TE\ \In\ \Env }
\end{equation}}{\begin{equation}        % eqtype specification
\label{eqtypspec-rule}
\frac{ \of{\C}{\B}\ts\typdesc\ra\TE \qquad
       \forall(\t,\VE)\in \Ran\TE,\ \t {\rm\ admits\ equality} }
     { \B\ts\eqtypespec\ra\TE\ \In\ \Env }
\end{equation}}

%
\replacementPage{34}{
\begin{equation}        % data specification
\label{datatypespec-rule}
\frac{ \plusmap{\of{\C}{\B}}{\TE}\ts\datdesc\ra\VE,\TE }
     { \B\ts\datatypespec\ra(\VE,\TE)\ \In\ \Env }
\end{equation}}{
\begin{equation}        % data specification
\label{datatypespec-rule}
\frac{ \begin{array}{c}
       \of{\C}{\B}\oplus\TE\ts\datdesc\ra\VE,\TE \quad
       \forall(\t,\VE')\in\Ran\TE, \t\notin\of{\T}{\B}\\
       \hbox{$\TE$ maximises equality}
       \end{array}}
     { \B\ts\datatypespec\ra(\VE,\TE)\ \In\ \Env }
\end{equation}}

%
\insertionPage{34}{\begin{equation}
\label{datatypereplspec-rule}
\frac{ \B(\longtycon) = (\typefcn,\VE)\quad
       \typefcn = \typefcnk\quad
       \tyvarseq = \alphak\quad
       \TE = \{\tycon\mapsto(\typefcn,\VE)\}
     }
     {\B\ts\datatypereplspec\ra (\VE,\TE)\ \In\ \Env}
\end{equation}}

%
\replacementPage{34}{
\begin{equation}        % exception specification
\label{exceptionspec-rule}
\frac{ \of{\C}{\B}\ts\exndesc\ra\EE\quad\VE=\EE }
     { \B\ts\exceptionspec\ra(\VE,\EE)\ \In\ \Env }
\end{equation}}{\begin{equation}        % exception specification
\label{exceptionspec-rule}
\frac{ \of{\C}{\B}\ts\exndesc\ra\VE }
     { \B\ts\exceptionspec\ra\VE\ \In\ \Env }
\end{equation}}

%
\deletionPage{34}{
\begin{equation}        % sharing specification
%\label{sharingspec-rule}
\frac{ \B\ts\shareq\ra\emptymap }
     { \B\ts\sharingspec\ra\emptymap\ \In\ \Env }\index{42.3}
\end{equation}}

%
\deletionPage{34}{\begin{equation}        % local specification
%\label{localspec-rule}
\frac{ \B\ts\spec_1\ra\E_1 \qquad \plusmap{\B}{\E_1}\ts\spec_2\ra\E_2 }
     { \B\ts\localspec\ra\E_2 }
\end{equation}}

%
\deletionPage{34}{\begin{equation}        % open specification
%\label{openspec-rule}
\frac{ \B(\longstrid_1)=(\m_1,\E_1)\quad\cdots\quad
       \B(\longstrid_n)=(\m_n,\E_n) }
     { \B\ts\openspec\ra\E_1 + \cdots +\E_n }
\end{equation}}

%
\replacementPage{34}{\begin{equation}        % include signature specification
\label{inclspec-rule}
\frac{ \sigord{\B(\sigid_1)}{}{(\m_1,\E_1)} \quad\cdots\quad
       \sigord{\B(\sigid_n)}{}{(\m_n,\E_n)} }
     { \B\ts\inclspec\ra\E_1 + \cdots +\E_n }
\end{equation}}{\begin{equation}        % include signature specification
\label{inclspec-rule}
\frac{  \B\ts\sigexp\ra\E}
     { \B\ts\singleinclspec\ra\E }
\end{equation}}

%
\replacementPage{34}{
\begin{equation}        % sequential specification
%\label{seqspec-rule}
\frac{ \B\ts\spec_1\ra\E_1 \qquad \plusmap{\B}{\E_1}\ts\spec_2\ra\E_2 }
     { \B\ts\seqspec\ra\plusmap{\E_1}{\E_2} }
\end{equation}}{\begin{equation}        % sequential specification
\label{seqspec-rule}
\frac{ \B\ts\spec_1\ra\E_1 \qquad \B\oplus\E_1\ts\spec_2\ra\E_2\qquad\Dom(\E_1)\cap\Dom(\E_2) = \emptyset }
     { \B\ts\seqspec\ra\plusmap{\E_1}{\E_2} }
\end{equation}}

%
\insertionPage{34}{
\begin{equation}
\label{sharspec-rule}
\frac{\begin{array}{c}
        \B\ts \spec \ra \E\quad \E(\longtycon_i) = (\t_i,\VE_i), \;i = 1..n\\
        t\in\{\t_1,\ldots,\t_n\}\quad\hbox{$\t$ admits equality, if some $\t_i$ does}\\
        \{\t_1,\ldots,\t_n\}\cap\of{\T}{\B} = \emptyset\quad
           \rea = \{\t_1\mapsto\t,\ldots,\t_n\mapsto \t\}
      \end{array}
     }
     {\B\ts \newsharingspec\ra \rea(\E)}
\end{equation}}

%
\replacementPage{35}{The type functions in $\TE$ may be chosen to achieve the sharing hypothesis
of rule~\ref{typshareq-rule} or the enrichment conditions of 
rules~\ref{structure-binding-rule} and~\ref{funbind-rule}. In particular, the type
names in $\TE$ in rule~\ref{datatypespec-rule} need not be new.
Also, in rule~\ref{typespec-rule} the type functions in $\TE$ may admit
equality.}{The type
names in $\TE$ are new.}

%
\replacementPage{35}{$\EE$}{$\VE$}

%
\deletionPage{35}{\item{(\ref{inclspec-rule})}
   The names $\m_i$ in the instances may be chosen to achieve sharing or
enrichment conditions.\index{43.0}}

%
\insertionPage{35}{\item{(\ref{seqspec-rule})}
   Note that no sequential specification is allowed to specify the
   same identifier twice.}

%
\replacementPage{35}{
\begin{equation}         % value description
%\label{valdesc-rule}
\frac{ \C\ts\ty\ra\tau\qquad
       \langle\C\ts\valdesc\ra\VE\rangle }
     { \C\ts\valdescription\ra\{\var\mapsto\tau\}
       \ \langle +\ \VE\rangle }\index{43.1}
\end{equation}}{\begin{equation}         % value description
%\label{valdesc-rule}
\frac{ \C\ts\ty\ra\tau\qquad
       \langle\C\ts\valdesc\ra\VE\rangle }
     { \C\ts\valviddescription\ra\{\vid\mapsto(\tau,\isv)\}
       \ \langle +\ \VE\rangle }\index{43.1}
\end{equation}}

%
\replacementPage{35}{
\begin{equation}         % type description
\label{typdesc-rule}
\frac{ \tyvarseq = \alphak
       \qquad\langle \C\ts\typdesc\ra\TE\rangle\qquad\arity\theta=k }
     { \C\ts\typdescription\ra\{\tycon\mapsto(\theta,\emptymap)\}
       \ \langle +\ \TE\rangle }\index{43.2}
\end{equation}}{\begin{equation}         % type description
\label{typdesc-rule}
\frac{ \begin{array}{c}
         \tyvarseq = \alphak\quad\t\notin\of{\T}{\C}\quad\arity\t=k \\
       \langle \C\ts\typdesc\ra\TE\qquad\t\notin\TyNamesFcn\TE\rangle
      \end{array}}
     { \C\ts\typdescription\ra\{\tycon\mapsto(\t,\emptymap)\}
       \ \langle +\ \TE\rangle }\index{43.2}
\end{equation}}

%
\deletionPage{35}{any $\theta$ of arity $k$ may be chosen but that}

%
\replacementPage{35}{constructor}{value}

%
\replacementPage{35}{For example, \mbox{\ml{datatype s=c type t sharing s=t}}\  }{For example, \mbox{\ml{datatype s=C type t}} \mbox{\ml{sharing type t=s}}\  }

%
\replacementPage{35}{\begin{equation}         % datatype description
\label{datdesc-rule}
\frac{ \tyvarseq = \alphak\qquad\C,\alphakt\ts\condesc\ra\CE
       \qquad\langle\C\ts\datdesc\ra\VE,\TE\rangle }
     { \begin{array}{cl}
       \C\ts\datdescription\ra\\
       \qquad\qquad\cl{}{\CE}\langle +\ \VE\rangle,\
       \{\tycon\mapsto(t,\cl{}{\CE})\}\ \langle +\ \TE\rangle
       \end{array}
     }\index{43.3}
\end{equation}}{\begin{equation}         % datatype description
\label{datdesc-rule}
\frac{ \begin{array}{c}
        \tyvarseq = \alphak\qquad\C,\alphakt\ts\condesc\ra\VE
           \quad\arity \t  =  k\\
        \langle\C\ts\datdesc'\ra\VE',\TE'\qquad \forall(\t',\VE'')\in\Ran\TE',\,\t\neq\t'\rangle 
       \end{array}}
     { \begin{array}{l}
       \C\ts\newdatdescription\ra\\
       \qquad\cl{}{\VE} \langle +\ \VE'\rangle,\
       \{\tycon\mapsto(t,\cl{}{\VE})\}\ \langle +\ \TE'\rangle
       \end{array}
     }\index{43.3}
\end{equation}}

%
\replacementPage{35}{\rulesec{Constructor Descriptions}{\C,\tau\ts\condesc\ra\CE}}{\rulesec{Constructor Descriptions}{\C,\tau\ts\condesc\ra\VE}}

%
\replacementPage{35}{\begin{equation}         % constructor description
%\label{condesc-rule}
\frac{\langle\C\ts\ty\ra\tau'\rangle\qquad
      \langle\langle\C,\tau\ts\condesc\ra\CE\rangle\rangle }
     {\begin{array}{l}
      \C,\tau\ts\longcondescription\ra\\
      \qquad\{\con\mapsto\tau\}\
     \langle +\ \{\con\mapsto\tau'\to\tau\}\ \rangle\
      \langle\langle +\ \CE\rangle\rangle
      \end{array}
     }\index{43.35}
\end{equation}}{\begin{equation}         % constructor description
%\label{condesc-rule}
\frac{\langle\C\ts\ty\ra\tau'\rangle\qquad
      \langle\langle\C,\tau\ts\condesc\ra\VE\rangle\rangle }
     {\begin{array}{l}
      \C,\tau\ts\longconviddescription\ra\\
      \qquad\{\vid\mapsto(\tau,\isc)\}\
     \langle +\ \{\vid\mapsto(\tau'\to\tau,\isc)\}\ \rangle\
      \langle\langle +\ \VE\rangle\rangle
      \end{array}
     }\index{43.35}
\end{equation}}

%
\replacementPage{35}{
\rulesec{Exception Descriptions}{\C\ts\exndesc\ra\EE}}{\rulesec{Exception 
Descriptions}{\C\ts\exndesc\ra\VE}}

%
\replacementPage{35}{
\begin{equation}         % exception description
\label{exndesc-rule}
\frac{ \langle\C\ts\ty\ra\tau\qquad\TyVarsFcn(\tau)=\emptyset\rangle\qquad
       \langle\langle\C\ts\exndesc\ra\EE\rangle\rangle }
     { \begin{array}{l}
        \C\ts\exndescriptiona\ra\\
        \quad\quad\{\exn\mapsto\EXCN\}\ \langle +\ \{\exn\mapsto\tau\rightarrow\EXCN\}\rangle\ \langle\langle +\ \EE\rangle\rangle 
       \end{array}
     }\index{43.4}
\end{equation}}{\begin{equation}         % exception description
\label{exndesc-rule}
\frac{ \langle\C\ts\ty\ra\tau\qquad\TyVarsFcn(\tau)=\emptyset\rangle\qquad
       \langle\langle\C\ts\exndesc\ra\VE\rangle\rangle }
     { \begin{array}{l}
        \C\ts\exnviddescriptiona\ra\\
        \quad\quad\{\vid\mapsto(\EXCN,\ise)\}\ \langle +\ \{\vid\mapsto(\tau\rightarrow\EXCN,\ise)\}\rangle\ \langle\langle +\ \VE\rangle\rangle 
       \end{array}
     }\index{43.4}
\end{equation}}

%
\replacementPage{36}{
\begin{equation}
\label{strdesc-rule}
\frac{ \B\ts\sigexp\ra\S\qquad\langle\B\ts\strdesc\ra\SE\rangle }
     { \B\ts\strdescription\ra\{\strid\mapsto\S\}\ \langle +\ \SE\rangle }\index{43.5}
\end{equation}}{\begin{equation}
\label{strdesc-rule}
\frac{ \B\ts\sigexp\ra\E\qquad\langle\B + \TyNamesFcn\E\ts\strdesc\ra\SE\rangle }
     { \B\ts\strdescription\ra\{\strid\mapsto\E\}\ \langle +\ \SE\rangle }\index{43.5}
\end{equation}}

%
\deletionPage{36}{
\rulesec{Sharing Equations}{\B\ts\shareq\ra\emptymap}
\begin{equation}          % structure sharing equation
\label{strshareq-rule}
\frac{ \of{\m}{\B(\longstrid_1)}=\cdots =\of{\m}{\B(\longstrid_n)} }
     { \B\ts\strshareq\ra\E,\emptymap }\index{44.1}
\end{equation}
\vspace{6pt}
\begin{equation}          % type sharing equation
\label{typshareq-rule}
\frac{ \of{\typefcn}{\B(\longtycon_1)}=\cdots=\of{\typefcn}{\B(\longtycon_n)} }
     { \B\ts\typshareq\ra\emptymap }
\end{equation}

\vspace{6pt}
\begin{equation}          % multiple sharing equation
%\label{multshareq-rule}
\frac{ \B\ts\shareq_1\ra\emptymap\qquad\B\ts\shareq_2\ra\emptymap }
     { \B\ts\multshareq\ra\emptymap }
\end{equation}

%
\comments
\begin{description}
\item{(\ref{strshareq-rule})}
   By the definition of consistency the premise is weaker than\linebreak
$\B(\longstrid_1) = \cdots = \B(\longstrid_n)$.
Two different structures with the same name may be thought of
as representing different views. The requirement that $\B$ is 
consistent forces different views to be consistent.
\end{description}
%
\oldpagebreak
\begin{description}
\item{(\ref{typshareq-rule})}
   By\index{44.1.5} 
the definition of consistency the premise is weaker than\linebreak
$\B(\longtycon_1) = \cdots = \B(\longtycon_n)$.
A type structure with empty constructor environment may have the
same type name as one with a non-empty constructor environment;
the former could arise from a type description, and the latter
from a datatype description. 
However, the requirement that $\B$ is
consistent will prevent two type structures with constructor
environments which have different 
non-empty domains from sharing the same type name.
\end{description}
}

%
\deletionPage{36}{
%
%                       Functor Specification rules
%
\rulesec{Functor Specifications}{\B\ts\funspec\ra\F}
\begin{equation}        % single functor specification
\label{singfunspec-rule}
\frac{ \B\ts\fundesc\ra\F }
     { \B\ts\singfunspec\ra\F }\index{44.2}
\end{equation}

\vspace{6pt}
\begin{equation}        % empty functor specification
%\label{emptyfunspec-rule}
\frac{}
     { \B\ts\emptyfunspec\ra\emptymap }
\end{equation}

\vspace{6pt}
\begin{equation}        % sequential functor specification
%\label{seqfunspec-rule}
\frac{ \B\ts\funspec_1\ra\F_1\qquad
       \B+\F_1\ts\funspec_2\ra\F_2 }
     { \B\ts\seqfunspec\ra\plusmap{\F_1}{\F_2} }
\end{equation}
\comments
\begin{description}
\item{(\ref{singfunspec-rule})}
The second closure restriction of Section~\ref{closure-restr-sec}
can be enforced by replacing the $\B$ in the premise by $\B_0+\of{\G}{\B}$.
\end{description}
\rulesec{Functor Descriptions}{\B\ts\fundesc\ra\F}
\begin{equation}        % functor description
%\label{fundesc-rule}
\frac{ \B\ts\funsigexp\ra\funsig\qquad
       \langle\B\ts\fundesc\ra\F\rangle}
     { \B\ts\longfundesc\ra\{\funid\mapsto\funsig\}
       \langle +\ \F\rangle}\index{44.3}
\end{equation}

\rulesec{Functor Signature Expressions}{\B\ts\funsigexp\ra\funsig}
\begin{equation}        % functor signature
\label{funsigexp-rule}
%version 1:
%\frac{
%      \begin{array}{c}
%      \B\ts\sigexp\ra\S\qquad\longsig{}{\rm\ principal\ in\ }\B\\
%      \B\oplus\{\strid\mapsto\S\} \ts\sigexp'\ra\S'\\
%      \N' = \NamesFcn\S'\setminus((\of{\N}{\B})\cup\N) 
%      \end{array}
%     }
%     {\B\ts\longfunsigexpa\ra(\N)(\S,(\N')\S')}\index{44.4}
%version2: \frac{\begin{array}{rl}
%      \B\ts\sigexp\ra\S&\mbox{$(N)S$ principal in $\B$}\\
%      \B\oplus\{\strid\mapsto\S\}\ts\sigexp'\ra\S'&
%      \mbox{$(N')S'$ principal in $\B\oplus\{\strid\mapsto\S\}$}
%      \end{array}}
%     {\B\ts\longfunsigexpa\ra(N)(S,(N')S')}\index{44.4}
%\end{equation}
\frac{\B\ts\sigexp\ra(\N)\S\qquad
      \B\oplus\{\strid\mapsto\S\}\ts\sigexp'\ra(\N')\S'}
     {\B\ts\longfunsigexpa\ra(N)(S,(N')S')}\index{44.4}
\end{equation}
\comment
The signatures $(\N)\S$ and $(\N')\S'$ are equality-principal 
and type-explicit, see rule~\ref{topmost-sigexp-rule}.
}

%
\deletionPage{36}{
\vspace{6pt}
\begin{equation}        % empty functor declaration
%\label{emptyfundec-rule}
\frac{}
     { \B\ts\emptyfundec\ra\emptymap }
\end{equation}

\vspace{6pt}
\oldpagebreak

\begin{equation}        % sequential functor declaration
%\label{seqfundec-rule}
\frac{ \B\ts\fundec_1\ra\F_1\qquad
       \B+\F_1\ts\fundec_2\ra\F_2 }
     { \B\ts\seqfundec\ra\plusmap{\F_1}{\F_2} }\index{45.1.5}
\end{equation}}

%
\deletionPage{36}{\comments
\begin{description}
\item{(\ref{singfundec-rule})}
The third closure restriction of Section~\ref{closure-restr-sec}
can be enforced by replacing the $\B$ in the premise 
by $\B_0+(\of{\G}{\B})+(\of{\F}{\B})$.
\end{description}}

%
\replacementPage{36}{\begin{equation}        % functor binding
\label{funbind-rule}
\frac{
      \begin{array}{c}
      \B\ts\sigexp\ra(\N)\S\qquad
      \B\oplus\{\strid\mapsto\S\} \ts\strexp\ra\S' \\
       \langle
      \B\oplus\{\strid\mapsto\S\} \ts\sigexp'\ra\sig',\ \sig'\geq\S''\prec\S'
       \rangle\\
      \N' = \NamesFcn\S'\setminus((\of{\N}{\B})\cup\N) \\
       \langle\langle\B\ts\funbind\ra\F\rangle\rangle
      \end{array}
     }
     {
      \begin{array}{c}
       \B\ts\funstrbinder\ \optfunbind\ra\\
       \qquad\qquad \qquad
              \{\funid\mapsto(\N)(\S,(\N')\S'\langle'\rangle)\}
              \ \langle\langle +\ \F\rangle\rangle
      \end{array}
     }\index{45.2}
\end{equation}}{\begin{equation}        % functor binding
\label{funbind-rule}
\frac{
      \begin{array}{c}
      \B\ts\sigexp\ra(\T)\E\qquad
      \B\oplus\{\strid\mapsto\E\} \ts\strexp\ra\E' 
      \\
      \T\cap(\of{\T}{\B}) = \emptyset\quad \T' = \TyNamesFcn\E'\setminus((\of{\T}{\B})\cup\T) \\
       \langle\B\ts\funbind\ra\F\rangle
      \end{array}
     }
     {
      \begin{array}{c}
       \B\ts\barefunstrbinder\ \langle\boxml{and \funbind}\rangle\ra\\
       \qquad\qquad \qquad
              \{\funid\mapsto(\T)(\E,(\T')\E')\}
              \ \langle +\ \F\rangle
      \end{array}
     }\index{45.2}
\end{equation}
}

%
\deletionPage{36}{The  requirement that $(\N)\S$ be equality-principal,
implicit in the first premise, forces $(\N)\S$ to be
as general as possible given the sharing constraints in $\sigexp$.
The requirement that $(\N)\S$ be type-explicit ensures that there is
at most one realisation via which an actual argument can match
$(\N)\S$.}

%
\deletionPage{36}{structure name $\m$ and}

%
\replacementPage{36}{$\S$ }{$\E$ }

%
\deletionPage{36}{$\m$ and }

%
\replacementPage{36}{The set $\N'$ is
chosen such that every  name free
in $(\N)\S$ or $(\N)(\S,(\N')\S')$ is free in $\B$.}{The set $\T'$ is
chosen such that every  name free
in $(\T)\E$ or $(\T)(\E,(\T')\E')$ is free in $\B$.}

%
\replacementPage{36}{\begin{equation}	% structure-level declaration
\label{strdectopdec-rule}
\frac{\B\ts\strdec\ra\E \quad\imptyvars\E=\emptyset}
     {\B\ts\strdec\ra
      (\NamesFcn\E,\E)\ \In\ \Basis
     }\index{45.3}
\end{equation}}{\begin{equation}        % structure-level declaration
\label{strdectopdec-rule}
\frac{\begin{array}{c}
        \B\ts\strdec\ra\E \quad \langle \B\oplus \E\ts \topdec\ra \B'\rangle\\
        B'' = (\TyNamesFcn\E,\E)\In\ \Basis\; \langle + \B'\rangle\quad\TyVarFcn\B''=\emptyset
      \end{array}}
     {\B\ts\strdecintopdec\ra\B''}
\end{equation}}

%
\replacementPage{36}{\begin{equation}	% signature declaration
%\label{sigdectopdec-rule}
\frac{\B\ts\sigdec\ra\G \quad\imptyvars\G=\emptyset}
     {\B\ts\sigdec\ra
      (\NamesFcn\G,\G)\ \In\ \Basis
     }\index{46.0}
\end{equation}}{\begin{equation}        % signature declaration
\frac{\begin{array}{c}
        \B\ts\sigdec\ra \G\quad \langle\B\oplus\G\ts\topdec\ra\B'\rangle\\
        \B'' = (\TyNamesFcn\G,G)\ \In\ \Basis\;\langle + \B'\rangle
      \end{array}}
     {\B\ts\sigdecintopdec\ra \B''
     }\index{46.0}
\end{equation}}

%
\replacementPage{36}{\begin{equation}	% functor declaration
\label{fundectopdec-rule}
\frac{\B\ts\fundec\ra\F \quad\imptyvars\F=\emptyset}
     {\B\ts\fundec\ra
      (\NamesFcn\F,\F)\ \In\ \Basis
     }
\end{equation}}{\begin{equation}        % functor declaration
\label{fundectopdec-rule}
\frac{\begin{array}{c}
          \B\ts\fundec\ra\F\quad\langle\B\oplus\F\ts\topdec\ra\B'\rangle\\
          B'' = (\TyNamesFcn\F,\F)\ \In\ \Basis\; \langle+\B'\rangle\quad \TyVarsFcn\B''=\emptyset
      \end{array}}
     {\B\ts\fundecintopdec\ra\B''}
\end{equation}}

%
\replacementPage{36}{
\begin{description}
\item{(\ref{strdectopdec-rule})--(\ref{fundectopdec-rule})} The side
conditions ensure that no free imperative 
type variables enter the 
basis.\index{46.01}
\end{description}}{
\begin{description}
\item{(\ref{strdectopdec-rule})--(\ref{fundectopdec-rule})} 
No free type variables enter the  basis: if $\B\ts\topdec\ra\B'$
then $\TyVarsFcn(\B') = \emptyset$.\index{46.01}
\end{description}}

%
\deletionPage{36}{\subsection{Functor Signature Matching}
\label{fun-sig-match-sec}
As\index{46} pointed out in Section~\ref{mod-gram-sec} on the 
grammar for Modules, there is no phrase class whose elaboration 
requires matching one functor signature to another functor signature.
But a precise definition of this matching is needed, since a 
functor $g$ may only be separately compiled in the presence of 
specification of any functor $f$ to which $g$ refers, and then a 
real functor $f$ must match this specification.
In the case, then, that $f$ has been specified by a functor signature
\[\funsig_1\ =\ \longfunsig{1}\]
and that later $f$ is declared with functor signature
\[\funsig_2\ =\ \longfunsig{2}\]
the following matching rule will be employed:

A functor signature
$\funsig_2\ =\ \longfunsig{2}$ {\sl matches} another functor signature,
$\funsig_1\ =\ \longfunsig{1}$, if there exists a realisation $\rea$ 
such that
\begin{enumerate}
\item $\longsig{1}$ matches $\longsig{2}$ via $\rea$, and
\item $\rea((\N_2')\S_2')$ matches $(\N_1')\S_1'$.
\end{enumerate}
The first condition ensures that the real functor signature $\funsig_2$
for $f$ requires the argument $\strexp$ of any application $\f(\strexp)$
to have no more sharing, and no more richness, than was predicted by
the specified signature $\funsig_1$.
The second condition ensures that the real functor signature $\funsig_2$,
instantiated to $(\rea\S_2,\rea((\N_2')\S_2'))$, provides in the result of
the application $\f(\strexp)$
no less sharing, and no less richness, than was predicted by
the specified signature $\funsig_1$.

%We claim that any phrase -- e.g. the declaration of the functor $g$ above --
%which elaborates successfully in a basis $\B$ with $\B(f)=\funsig_1$ will
%also elaborate successfully in the basis $\B+\{f\mapsto\funsig_2\}$.  This
%claim justifies our definition of functor matching.
% -- this claim is false because of open.

}

%
\replacementPage{37}{
Since\index{47.1} types are fully dealt with in the static semantics,
the dynamic semantics ignores them.  The Core syntax is therefore
reduced by the following transformations, for the purpose of the dynamic
semantics:}{Since\index{47.1} types are mostly dealt with in the static semantics,
the Core syntax is 
reduced by the following transformations, 
for the purpose of the dynamic
semantics:}

%
\insertionPage{37}{constructor and}

%
\deletionPage{37}{
\item Any declaration of the form ``$\,\typedec\,$'' 
      or ``$\datatypedec$'' is replaced by the empty declaration.}

%
\deletionPage{37}{\item A declaration of the form ``$\,\abstypedec\,$'' is replaced by ``$\dec\,$''.}

%
\deletionPage{37}{TypBind, DatBind, $\ConBind$, }

%
\insertionPage{37}{, word}

%
\insertionPage{37}{or character}

%
\deletionPage{37}{ASCII }

%
\replacementPage{37}{Exception constructors evaluate to exception names, unlike value constructors
which simply evaluate to themselves.}{Exception constructors evaluate to exception names.}

%
\replacementPage{37}{We generally omit the injection functions taking $\Con$,
$\Con\times\Val$ etc into $\Val$.}{We generally omit the injection functions taking $\VId$,
$\VId\times\Val$ etc into $\Val$.}

%
\replacementPage{38}{$\Val =\{\mbox{\tt :=}\}\cup\SVal\cup\BasVal\cup\Con$}{$\Val =\{\mbox{\tt :=}\}\cup\SVal\cup\BasVal\cup\VId$}

%
\replacementPage{38}{$\qquad\cup(\Con\times\Val)\cup\ExVal$}{$\qquad\cup(\VId\times\Val)\cup\ExVal$}

%
\replacementPage{38}{\Closure}{\FcnClosure}

%
\replacementPage{38}{$\Closure = \Match\times\Env\times\VarEnv$}{$\FcnClosure = \Match\times\Env\times\ValEnv$}

%
\insertionPage{38}{\TE,}

%
\deletionPage{38}{,\EE}

%
\replacementPage{38}{$\Env = \StrEnv\times\VarEnv\times\ExnEnv$}{$\Env = \StrEnv\times\TyEnv\times\ValEnv$}

%
\insertionPage{38}{$        \TE      \in    \TyEnv = \finfun{\TyCon}{\ValEnv}$}

%
\replacementPage{38}{$\VarEnv = \finfun{\Var}{\Val}$}{$\ValEnv = \finfun{\VId}{\Val\times\IdStatus}$}

%
\deletionPage{38}{$        \EE	 \in	 \ExnEnv = \finfun{\Exn}{\Exc}$}

%
\deletionPage{38}{
An important point is that structure names $\m$ have
no significance at all in the dynamic semantics; this explains why the
object class $\Str = \StrNames\times\Env$ is absent here -- for the dynamic
semantics the concepts {\sl structure} and {\sl environment} coincide.}

%
\deletionPage{38}{the }

%
\replacementPage{38}{identifiers}{value variables}

%
\replacementPage{38}{
These values are denoted by the identifiers to which they are bound in the
initial dynamic basis (see Appendix~\ref{init-dyn-bas-app}), 
and are as follows:
{\tt 
           abs  floor  real  sqrt  sin  cos  arctan  exp  ln
              size  chr  ord  explode  implode  div  mod
                  ~  /  *  +  -  =  <>  <  >  <=  >=
        std\_in  std\_out  open\_in  open\_out  close\_in  close\_out
                input  output  lookahead  end\_of\_stream
}
The meaning of basic values (almost all of which are functions) is
represented by the function
$\APPLY\ :\ \BasVal\times\Val\to\Val\cup\Pack $
 which is detailed in Appendix~\ref{init-dyn-bas-app}.
}{In this document, we take $\BasVal$ to be the singleton set $\{\boxml{=}\}$;
however, libraries may define a larger set of basic values. 
The meaning of basic values is
represented by a function
$ \APPLY\ :\ \BasVal\times\Val\to\Val\cup\Pack $
which satisfies that $\APPLY(\boxml{=}, \{1\mapsto v_1, 2\mapsto v_2\})$ 
is {\tt true} or {\tt false} according as the values $\V_1$ and $\V_2$ are, or are
not, identical values. 
}

%
\replacementPage{39}{
A\index{49.2} subset $\BasExc\subset\Exc$ of the exception names are bound to predefined
exception constructors.
These names are denoted by the identifiers to which they are bound in the
initial dynamic basis (see Appendix~\ref{init-dyn-bas-app}), 
and are as follows:}{A\index{49.2} 
subset $\BasExc\subset\Exc$ of the exception names are bound to predefined
exception constructors in the initial dynamic basis 
(see Appendix~\ref{init-dyn-bas-app}).
These names are denoted by the identifiers 
to which they are bound in the
initial basis, 
and are as follows:}

%
\replacementPage{39}{{\tt 
        Abs  Ord  Chr   Div  Mod  Quot  Prod  
        Neg  Sum  Diff  Floor  Sqrt  Exp  Ln
        Io   Match  Bind  Interrupt
}}{$$\boxml{ Match\ \  Bind}$$}

%
\replacementPage{39}{
The exceptions on the first two  lines are raised by 
corresponding basic functions, where \verb+~+ {\tt /} {\tt *}
{\tt +} {\tt -} correspond respectively to {\tt Neg} {\tt Quot}
{\tt Prod} {\tt Sum} {\tt Diff}. The details are given
in Appendix~\ref{init-dyn-bas-app}. The exception $(\mbox{{\tt Io}},s)$,
where $s$ is a string, is raised
by certain of the basic input/output functions,
as detailed in Appendix~\ref{init-dyn-bas-app}.  
}{The}

%
\deletionPage{39}{
Finally, ~\ml{Interrupt}~
is raised by external intervention.}

%
\replacementPage{39}{\subsection{Closures}}{\subsection{Function Closures}}

%
\insertionPage{39}{function}

%
\insertionPage{39}{function}

%
\deletionPage{39}{function }

%
\replacementPage{39}{where\[ \Rec\ :\ \VarEnv\to\VarEnv \]is defined as follows:}{where\[ \Rec\ :\ \ValEnv\to\ValEnv \]is defined as follows:}

%
\replacementPage{39}{$\VE(\var)\notin\Closure$}{$\VE(\vid)\notin\FcnClosure\times\{\isv\}$}

%
\replacementPage{39}{If $\VE(\var)=(\match',\E',\VE')$
      then $(\Rec\VE)(\var)=(\match',\E',\VE)$}{If $\VE(\vid)=((\match',\E',\VE'), \isv)$
      then $(\Rec\VE)(\vid)=((\match',\E',\VE), \isv)$}

%
\replacementPage{39}{closure values}{function closures}

%
\insertionPage{40}{function}

%
\replacementPage{41}{\begin{equation}	% value variable
\label{varexp-dyn-rule}
\frac{\E(\longvar)=\V}
     {\E\ts\longvar\ra\V}\index{51.2}
\end{equation}
\oldpagebreak
\begin{equation}	% value constructor
\label{conexp-dyn-rule}
\frac{\longcon=\strid_1.\cdots.\strid_k.\con}
     {\E\ts\longcon\ra\con}\index{52.1}
\end{equation}

\begin{equation}       %  exception constant
\label{exconexp-dyn-rule}
\frac{\E(\longexn)=\e}
     {\E\ts\longexn\ra\e}
\end{equation}}{\begin{equation}	% value variable
\label{varexp-dyn-rule}
\frac{\E(\longvid)=(\V,\is)}
     {\E\ts\longvid\ra\V}\index{51.2}
\end{equation}}

%
\deletionPage{41}{\item{(\ref{conexp-dyn-rule})}
   Value constructors denote themselves.}

%
\insertionPage{41}{\item{(\ref{varexp-dyn-rule})}
As in the static semantics,
value identifiers are looked up in the environment and the
identifier status is not used.}

%
\deletionPage{41}{
\item{(\ref{exconexp-dyn-rule})}
   Exception constructors are looked up in the exception environment
   component of $\E$.}

%
\replacementPage{41}{\begin{equation}	% constructor application
\label{conapp-dyn-rule}
\frac{\E\ts\exp\ra\con\qquad\con\neq\REF\qquad\E\ts\atexp\ra\V}
     {\E\ts\appexp\ra(\con,\V)}
\end{equation}}{\begin{equation}	% constructor application
\label{conapp-dyn-rule}
\frac{\E\ts\exp\ra\vid\qquad\vid\neq\REF\qquad\E\ts\atexp\ra\V}
     {\E\ts\appexp\ra(\vid,\V)}
\end{equation}}

%
\replacementPage{42}{
\begin{equation}	% basic function application
%\label{basapp-dyn-rule}
\frac{\E\ts\exp\ra b
      \qquad\E\ts\atexp\ra\V\qquad\APPLY(b,\V)=\V'}
     {\E\ts\appexp\ra\V'}\index{53.1}
\end{equation}}{
\begin{equation}	% basic function application
%\label{basapp-dyn-rule}
\frac{\E\ts\exp\ra b
      \qquad\E\ts\atexp\ra\V\qquad\APPLY(b,\V)=\V'/\p}
     {\E\ts\appexp\ra\V'/\p}\index{53.1}
\end{equation}}

%
\deletionPage{42}{ (``garbage
collection'')}

%
\insertionPage{43}{function}

%
\replacementPage{43}{
\begin{equation}	% match 3
%\label{match-dyn-rule}
\frac{\E,\V\ts\mrule\ra\FAIL\qquad\E,\V\ts\match\ra\V'/\FAIL}
     {\E\ts\longmatcha\ra\V'/\FAIL}
\end{equation}}{\begin{equation}	% match 3
%\label{match-dyn-rule}
\frac{\E,\V\ts\mrule\ra\FAIL\qquad\E,\V\ts\match\ra\V'/\FAIL}
     {\E,\V\ts\longmatcha\ra\V'/\FAIL}
\end{equation}}

%
\replacementPage{43}{
\begin{equation}	% value declaration
%\label{valdec-dyn-rule}
\frac{\E\ts\valbind\ra\VE}
     {\E\ts\valdec\ra\VE\ \In\ \Env}\index{54.3}
\end{equation}}{\begin{equation}	% value declaration
%\label{valdec-dyn-rule}
\frac{\E\ts\valbind\ra\VE}
     {\E\ts\explicitvaldec\ra\VE\ \In\ \Env}\index{54.3}
\end{equation}}

%
\insertionPage{43}{\begin{equation}
\frac{\ts\typbind\ra\TE}
     {\E\ts\typedec\ra\TE\ \In\ \Env}
\end{equation}}

%
\insertionPage{43}{
\begin{equation}
\frac{\ts\datbind\ra\VE,\TE}
     {\E\ts\datatypedec\ra(\VE,\TE)\ \In\ \Env}
\end{equation}}

%
\insertionPage{43}{\begin{equation}
\frac{\E(\longtycon) = \VE}
     {\begin{array}{r}
          \E\ts\datatyperepldec\ra\qquad\qquad\\
          (\VE,\{\tycon\mapsto\VE\})\ \In\ \Env
      \end{array}}
\end{equation}}

%
\insertionPage{44}{
\begin{equation}
\frac{\ts\datbind\ra\VE\qquad \E+\VE\ts\dec\ra \E'}
     {\E\ts\abstypedec \ra \E'}
\end{equation}}

%
\replacementPage{44}{
\begin{equation}	% exception declaration
%\label{exceptiondec-dyn-rule}
\frac{\E\ts\exnbind\ra\EE }
     {\E\ts\exceptiondec\ra\EE\ \In\ \Env }
\end{equation}}{\begin{equation}	% exception declaration
%\label{exceptiondec-dyn-rule}
\frac{\E\ts\exnbind\ra\VE }
     {\E\ts\exceptiondec\ra\VE\ \In\ \Env }
\end{equation}}

%
\replacementPage{44}{
\begin{equation}                % open declaration
%\label{open-strdec-dyn-rule}
\frac{ \E(\longstrid_1)=\E_1
            \quad\cdots\quad
       \E(\longstrid_k)=\E_k }
     { \E\ts\openstrdec\ra \E_1 + \cdots + \E_k }
\end{equation}}{
\begin{equation}                % open declaration
%\label{open-strdec-dyn-rule}
\frac{ \E(\longstrid_1)=\E_1
            \quad\cdots\quad
       \E(\longstrid_n)=\E_n }
     { \E\ts\openstrdec\ra \E_1 + \cdots + \E_n }
\end{equation}}

%
\insertionPage{44}{
\rulesec{Type Bindings}{\ts\typbind\ra\TE}
\begin{equation}
\frac{\langle\ts\typbind\ra\TE\rangle}
     {\ts\longtypbind\ra\{\tycon\mapsto\emptymap\}\langle+\TE\rangle}
\end{equation}}

%
\insertionPage{44}{
\rulesec{Datatype Bindings}{\ts\datbind\ra\VE,\TE}
\begin{equation}
\frac{\ts\constrs\ra\VE\qquad\langle\ts\datbind'\ra\VE',\TE'\rangle}
     {\ts\tyvarseq\;\tycon\boxml{=}\constrs\;\langle\boxml{and}\,\datbind'\rangle\ra\VE\langle+\VE'\rangle,\{\tycon\mapsto\VE\}\langle+\TE'\rangle}
\end{equation}

\rulesec{Constructor Bindings}{\ts\constrs\ra\VE}
\begin{equation}
\frac{\langle\ts\constrs\ra\VE\rangle}
     {\ts\vid\langle\boxml{|}\,\constrs\rangle\ra
               \{\vid\mapsto(\vid,\isc)\}\,\langle+\VE\rangle}
\end{equation}
}

%
\replacementPage{45}{
\rulesec{Exception Bindings}{\E\ts\exnbind\ra\EE/\p}}{\rulesec{Exception 
Bindings}{\E\ts\exnbind\ra\VE}}

%
\replacementPage{45}{
\begin{equation}	% exception binding 1
\label{exnbind-dyn-rule1}
\frac{\e\notin\of{\excs}{\s}\qquad\s'=\s+\{\e\}\qquad
      \langle\s',\E\ts\exnbind\ra\EE,\s''\rangle }
     {\s,\E\ts\longexnbindaa\ra\{\exn\mapsto\e\}\langle +\ \EE\rangle,\
                               \s'\langle'\rangle}\index{55.2}
\end{equation}}{\begin{equation}	% exception binding 1
\label{exnbind-dyn-rule1}
\frac{\e\notin\of{\excs}{\s}\qquad\s'=\s+\{\e\}\qquad
      \langle\s',\E\ts\exnbind\ra\VE,\s''\rangle }
     {\s,\E\ts\longvidexnbindaa\ra\{\vid\mapsto(\e,\ise)\}\langle +\ \VE\rangle,\
                               \s'\langle'\rangle}\index{55.2}
\end{equation}}

%
\replacementPage{45}{
\begin{equation}	% exception binding 2
%\label{exnbind-dyn-rule2}
\frac{\E(\longexn)=\e\qquad
      \langle\E\ts\exnbind\ra\EE\rangle }
     {\E\ts\longexnbindb\ra\{\exn\mapsto\e\}\langle +\ \EE\rangle}
\end{equation}}{\begin{equation}	% exception binding 2
%\label{exnbind-dyn-rule2}
\frac{\E(\longvid)=(\e,\ise)\qquad
      \langle\E\ts\exnbind\ra\VE\rangle }
     {\E\ts\longvidexnbindb\ra\{\vid\mapsto(\e,\ise)\}\langle +\ \VE\rangle}
\end{equation}}

%
\replacementPage{45}{\begin{equation}	% variable pattern
%\label{varpat-dyn-rule}
\frac{}
     {\E,\V\ts\var\ra \{\var\mapsto\V\} }
\end{equation}}{\begin{equation}	% variable pattern
%\label{varpat-dyn-rule}
\frac{\hbox{$\vid\notin\Dom(\E)$ or $\of{\is}{\E(\vid)} = \isv$}}
     {\E,\V\ts\vid\ra \{\var\mapsto(\V,\isv)\} }
\end{equation}}

%
\deletionPage{45}{\begin{equation}	% constant pattern
%\label{conapat-dyn-rule1}
\frac{\longcon=\strid_1.\cdots.\strid_k.\con\qquad\V=\con }
     {\E,\V\ts\longcon\ra \emptymap}
\end{equation}

\begin{equation}
\label{conapat-dyn-rule2}
\frac{\longcon=\strid_1.\cdots.\strid_k.\con\qquad\V\neq\con}
     {\E,\V\ts\longcon\ra\FAIL}
\end{equation}}

%
\replacementPage{45}{\begin{equation}        % exception constant
%\label{exconapat-dyn-rule1}
\frac{\E(\longexn)=\V}
     {\E,\V\ts\longexn\ra\emptymap}
\end{equation}}{\begin{equation}        % exception constant
%\label{exconapat-dyn-rule1}
\frac{\E(\longvid)=(\V,\is)\qquad\is\neq\isv}
     {\E,\V\ts\longvid\ra\emptymap}
\end{equation}}

%
\replacementPage{45}{\begin{equation}	
\label{exconapat-dyn-rule2}
\frac{\E(\longexn)\neq\V}
     {\E,\V\ts\longexn\ra\FAIL}\index{56.0}
\end{equation}}{\begin{equation}	
\label{exconapat-dyn-rule2}
\frac{\E(\longvid)=(\V',\is)\qquad\is\neq\isv\qquad\V\neq\V'}
     {\E,\V\ts\longvid\ra\FAIL}\index{56.0}
\end{equation}}

%
\deletionPage{46}{(\ref{conapat-dyn-rule2}),}

%
\deletionPage{46}{rule~\ref{conapat-dyn-rule2},}

%
\replacementPage{46}{\begin{equation}	% construction pattern
%\label{conpat-dyn-rule1}
\frac{\begin{array}{c}
       \longcon=\strid_1.\cdots.\strid_k.\con\neq\REF\qquad
      \V=(\con,\V')\\
      \E,\V'\ts\atpat\ra\VE/\fail
      \end{array}}
     {\E,\V\ts\conpat\ra \VE/\fail}
\end{equation}}{\begin{equation}	% construction pattern
%\label{conpat-dyn-rule1}
\frac{\begin{array}{c}
       \E(\longvid) = (\vid,\isc)\qquad\vid\neq\REF\qquad
      \V=(\vid,\V')\\
      \E,\V'\ts\atpat\ra\VE/\fail
      \end{array}}
     {\E,\V\ts\vidpat\ra \VE/\fail}
\end{equation}}

%
\replacementPage{46}{\begin{equation}	% construction pattern
\label{conpat-dyn-rule2}
\frac{\longcon=\strid_1.\cdots.\strid_k.\con\neq\REF\qquad
      \V\notin\{\con\}\times\Val}
     {\E,\V\ts\conpat\ra \FAIL}
\end{equation}}{\begin{equation}	% construction pattern
\label{conpat-dyn-rule2}
\frac{\E(\longvid) = (\vid,\isc)\qquad\vid\neq\REF\qquad
      \V\notin\{\vid\}\times\Val}
     {\E,\V\ts\vidpat\ra \FAIL}
\end{equation}}

%
\replacementPage{46}{\begin{equation}        % exception construction
%\label{exconpat-dyn-rule1}
\frac{\begin{array}{c}
      \E(\longexn)=\e\qquad\V=(\e,\V')\\
      \E,\V'\ts\atpat\ra\VE/\FAIL
      \end{array}
     }
     {\E,\V\ts\exconpat\ra\VE/\FAIL}
\end{equation}}{\begin{equation}        % exception construction
%\label{exconpat-dyn-rule1}
\frac{\begin{array}{c}
      \E(\longvid)=(\e,\ise)\qquad\V=(\e,\V')\\
      \E,\V'\ts\atpat\ra\VE/\FAIL
      \end{array}
     }
     {\E,\V\ts\vidpat\ra\VE/\FAIL}
\end{equation}}

%
\replacementPage{46}{\begin{equation} 
\label{exconpat-dyn-rule2}
\frac{\E(\longexn)=\e\qquad\V\notin\{\e\}\times\Val}
     {\E,\V\ts\exconpat\ra\FAIL}
\end{equation}}{\begin{equation} 
\label{exconpat-dyn-rule2}
\frac{\E(\longvid)=(\e,\ise)\qquad\V\notin\{\e\}\times\Val}
     {\E,\V\ts\vidpat\ra\FAIL}
\end{equation}}

%
\replacementPage{47}{\begin{equation}	% layered pattern
%\label{layeredpat-dyn-rule}
\frac{\E,\V\ts\pat\ra\VE/\fail}
     {\E,\V\ts\layeredpata\ra\{\var\mapsto\V\}+\VE/\fail}
\end{equation}}{\begin{equation}	% layered pattern
%\label{layeredpat-dyn-rule}
\frac{\begin{array}{c}
      \hbox{ $\vid\notin\Dom(\E)$ or $\of{\is}{\E(\vid)}=\isv$}\\
      \E,\V\ts\pat\ra\VE/\fail
      \end{array}}
     {\E,\V\ts\layeredvidpata\ra\{\vid\mapsto(\V,\isv)\}+\VE/\fail}
\end{equation}}

%
\deletionPage{47}{rule~\ref{conapat-dyn-rule2},}

%
\replacementPage{48}{Unlike types,
it cannot ignore them completely; }{However,
they cannot be ignored completely; }

%
\insertionPage{48}{ and 
imposing identifier status on value identifiers}

%
\replacementPage{48}{However, the types
and the sharing properties of structures and signatures are irrelevant to
dynamic evaluation; the syntax is therefore
reduced by the following transformations (in addition to those for the Core),
for the purpose of the dynamic semantics of Modules:}{The syntax is therefore
reduced by the following transformations (in addition to those for the Core),
for the purpose of the dynamic semantics of Modules:}

%
\insertionPage{48}{constructor and}

%
\deletionPage{48}{\item Any specification of the form ``$\typespec$'', ``$\eqtypespec$'',
``$\DATATYPE$\ $\datdesc$\,'' or
``$\sharingspec$'' is replaced by the empty specification.}

%
\deletionPage{48}{The Modules phrase classes TypDesc, DatDesc, ConDesc and SharEq
      are omitted.}

%
\insertionPage{48}{Any qualification \boxml{sharing type $\cdots$} on
a specification or \boxml{where type $\cdots$} on a signature expression is omitted.}

%
\replacementPage{48}{$(\strid:\I,\strexp\langle:\I'\rangle,\B)$}{$(\strid:\I,\strexp,\B)$}

%
\replacementPage{48}{$(\StrExp\langle\times\Int\rangle)\times\Basis$}{$\StrExp\times\Basis$}

%
\replacementPage{48}{$(\IE,\vars,\exns)\ {\rm or}\ \I$}{$\I\ {\rm or}\ (\SI,\TI,\VI)$}

%
\replacementPage{48}{$\Int = \IntEnv\times\Fin(\Var)\times\Fin(\Exn)$}{$\Int = \StrInt\times\TyInt\times\ValInt$}

%
\replacementPage{48}{\IE}{\SI}

%
\replacementPage{48}{\IntEnv}{\StrInt}

%
\insertionPage{48}{\begin{minipage}{\textwidth}\halign{\indent$#$\hfil&$#$\hfil&$#$\hfil&$#$\hfil\cr
        \TI     & \in   & \TyInt =  \finfun{\TyCon}{\ValInt}\cr
        \VI     & \in   & \ValInt = \finfun{\VId}{\IdStatus}\cr}\end{minipage}}

%
\replacementPage{48}{\IE}{\I}

%
\replacementPage{48}{\IntEnv}{\Int}

%
\insertionPage{48}{
An {\sl interface} $\I\in\Int$ represents a ``view'' of a structure.
Specifications and signature expressions will evaluate to interfaces;
moreover, during the evaluation of a specification or signature expression, 
structures (to which a specification or signature expression may
refer via datatype replicating specifications) are represented 
only by their interfaces.  To extract a value interface from
a dynamic value environment we define the operation $\Inter: \ValEnv \to\ValInt$
as follows:
\[\Inter(\VE) = \{\vid\mapsto\is\;;\;\VE(\vid) = (\V,\is)\}\]
In other words, $\Inter(\VE)$ is the value interface obtained from $\VE$ by
removing all values from $\VE$. We then extend $\Inter$ to a function
$\Inter:\Env\to\Int$ as follows:
\[ \Inter(\SE,\TE,\VE)\ =\ (\SI,\TI,\VI)\]
where $\VI$ = $\Inter(\VE)$ and 
\begin{eqnarray*}
\SI & = & \{\strid\mapsto\Inter\E\;;\;\SE(\strid) = \E\}\\
\TI & = & \{\tycon\mapsto\Inter\VE'\;;\;\TE(\tycon) = \VE'\}
\end{eqnarray*}
An {\sl interface basis} $\IB=(\G,\I)$ is a value-free part of a basis, sufficient to
evaluate signature expressions and specifications.
The function $\Inter$ is extended to create an interface basis
from a basis $\B$ as follows:
\[ \Inter(\F,\G,\E)\ =\ (\G, \Inter\E) \]
}

%
\deletionPage{49}{
An {\sl interface} $\I\in\Int$ represents a ``view'' of a structure.
Specifications and signature expressions will evaluate to interfaces; 
moreover, during the evaluation of a specification or signature expression, 
structures (to which a specification or signature expression may
refer via ``$\OPEN$'') are represented only by their interfaces.  To extract an
interface from a dynamic environment we define the operation
\[ \Inter\ :\ \Env\to\Int \]
as follows:
\[ \Inter(\SE,\VE,\EE)\ =\ (\IE,\Dom\VE,\Dom\EE)\]
where
\[ \IE\ =\ \{\strid\mapsto\Inter\E\ ;\ \SE(\strid)=\E\}\ .\]
An {\sl interface basis}\index{59.1} $\IB=(\G,\IE)$ is that part of a basis needed to
evaluate signature expressions and specifications.
The function $\Inter$ is extended to create an interface basis
from a basis $\B$ as follows:
\[ \Inter(\F,\G,\E)\ =\ (\G, \of{\IE}{(\Inter\E)}) \]}

%
\insertionPage{49}{
A further operation
\[ \downarrow\ :\ \Env\times\Int\to\Env\]
is required, to cut down an environment $\E$ to a given interface $\I$,
representing the effect of an explicit signature ascription. We first
define $\downarrow: \ValEnv\times\ValInt\to\ValEnv$ by
\[\VE\downarrow\VI = \{\vid\mapsto(\V,\is)\;;\;\VE(\vid) = (\V,\is')\ {\rm and}\ \VI(\vid) = \is\}\]
(Note that the identifier status is taken from $\VI$.) 
We then define $\downarrow: \StrEnv \times \StrInt \to \StrEnv$,
$\downarrow: \TyEnv \times\TyInt\to\TyEnv$ and
$\downarrow: \Env\times\Int\to\Env$ simultaneously as follows:
\label{downarrowdef}
\begin{center}
 $\SE\downarrow\SI  =  \{\strid\mapsto\E\downarrow\I\ ;\
          	\SE(\strid)=\E\ {\rm and}\ \SI(\strid)=\I\}$\\[6pt]
 $ \TE\downarrow\TI =  \{\tycon\mapsto \VE'\downarrow\VI'\ ;\ 
               \TE(\tycon) = \VE'\ {\rm and}\ \TI(\tycon) = \VI'\}$ \\[6pt]
 $
 (\SE,\TE,\VE)\downarrow(\SI,\TE,\VI)  = 
               (\SE\downarrow\SI, \TE\downarrow\TI, \VE\downarrow\VI) $
\end{center}}

%
\deletionPage{49}{
A further operation
\[ \downarrow\ :\ \Env\times\Int\to\Env\]
is required, to cut down an environment $\E$ to a given interface $\I$,
representing the effect of an explicit signature ascription.  It is defined
as follows:
\[ (\SE,\VE,\EE)\downarrow(\IE,\vars,\exns)\ =\ (\SE',\VE',\EE') \]
where
\[ \SE'\ =\ \{\strid\mapsto\E\downarrow\I\ ;\
          \SE(\strid)=\E\ {\rm and}\ \IE(\strid)=\I\} \]
and (taking $\downarrow$ now to mean restriction of a function domain)
\[\VE'=\VE\downarrow\vars,\ \EE'=\EE\downarrow\exns.\]
}

%
\replacementPage{49}{is also a projection of}{can
also be obtained from}

%
\replacementPage{49}{omitting structure names $\m$ and type environments
$\TE$,}{first replacing every type
structure $(\typefcn, \VE)$
in the range of every type environment $\TE$ by $\VE$}

%
\insertionPage{49}{then}

%
\replacementPage{49}{variable environment $\VE$ and each 
exception environment $\EE$ by its domain.}{pair $(\sigma,\is)$ in the range
of every value environment $\VE$ by $\is$.}

%
\replacementPage{49}{ or an interface basis}{, a signature environment}

%
\insertionPage{50}{\begin{equation}        % transparent signature constraint
\frac{\B\ts\strexp\ra\E\qquad\Inter\B\ts\sigexp\ra\I}
     {\B\ts\transpconstraint\ra\E\downarrow\I}
\end{equation}

\begin{equation}        % opaque signature constraint
\frac{\B\ts\strexp\ra\E\qquad\Inter\B\ts\sigexp\ra\I}
     {\B\ts\opaqueconstraint\ra\E\downarrow\I}
\end{equation}}

%
\replacementPage{50}{\begin{equation}                % functor application
\label{functor-application-dyn-rule}
\frac{ \begin{array}{c}
        \B(\funid)=(\strid:\I,\strexp'\langle:\I'\rangle,\B')\\
        \B\ts\strexp\ra\E\qquad
       \B'+\{\strid\mapsto\E\downarrow\I\}\ts\strexp'\ra\E'\\
       \end{array}
     }
     {\B\ts\funappstr\ra\E'\langle\downarrow\I'\rangle}
\end{equation}}{\begin{equation}                % functor application
\label{functor-application-dyn-rule}
\frac{ \begin{array}{c}
        \B(\funid)=(\strid:\I,\strexp',\B')\\
        \B\ts\strexp\ra\E\qquad
       \B'+\{\strid\mapsto\E\downarrow\I\}\ts\strexp'\ra\E'\\
       \end{array}
     }
     {\B\ts\funappstr\ra\E'}
\end{equation}}

%
\replacementPage{51}{
\begin{equation}                % structure binding
\frac{ \begin{array}{cl}
       \B\ts\strexp\ra\E\qquad\langle\Inter\B\ts\sigexp\ra\I\rangle\\
       \langle\langle\B\ts\strbind\ra\SE\rangle\rangle
       \end{array}
     }
     {\begin{array}{c}
      \B\ts\strbinder\ra\\
      \qquad\qquad\qquad\{\strid\mapsto\E\langle\downarrow\I\rangle\}
      \ \langle\langle +\ \SE\rangle\rangle
      \end{array}
     }\index{61.1}
\end{equation}
\comment As in the static semantics, when present, $\sigexp$ constrains the
``view'' of the structure. The restriction must be done in the
dynamic semantics to ensure that any dynamic opening of the structure
produces no more components than anticipated during the static
elaboration.}{\begin{equation}                % structure binding
\frac{ 
       \B\ts\strexp\ra\E\qquad
       \langle\B\ts\strbind\ra\SE\rangle
     }
     {
      \B\ts\barestrbindera\ra\{\strid\mapsto\E\}
      \ \langle +\ \SE\rangle
     }\index{61.1}
\end{equation}
}

%
\deletionPage{51}{
\begin{equation}        % empty signature declaration
%\label{empty-sigdec-dyn-rule}
\frac{}
     { \IB\ts\emptysigdec\ra\emptymap }
\end{equation}

\begin{equation}        % sequential signature declaration
%\label{sequence-sigdec-dyn-rule}
\frac{ \IB\ts\sigdec_1\ra\G_1 \qquad \plusmap{\IB}{\G_1}\ts\sigdec_2\ra\G_2 }
     { \IB\ts\seqsigdec\ra\plusmap{\G_1}{\G_2} }
\end{equation}
}

%
\replacementPage{51}{\begin{equation}        % value specification
%\label{valspec-dyn-rule}
\frac{ \ts\valdesc\ra\vars }
     { \IB\ts\valspec\ra\vars\ \In\ \Int }\index{61.5}
\end{equation}}{\begin{equation}        % value specification
%\label{valspec-dyn-rule}
\frac{ \ts\valdesc\ra\VI }
     { \IB\ts\valspec\ra\VI\ \In\ \Int }\index{61.5}
\end{equation}}

%
\insertionPage{51}{
\begin{equation}
\frac{\ts\typdesc\ra\TI}
     {\IB\ts\typespec\ra\TI\ \In\ \Int}
\end{equation}

\begin{equation}
\frac{\ts\typdesc\ra\TI}
     {\IB\ts\eqtypespec\ra\TI\ \In\ \Int}
\end{equation}}

%
\insertionPage{52}{
\begin{equation}
\frac{\ts\datdesc\ra\VI,\TI}
     {\IB\ts\datatypespec\ra(\VI,\TI)\ \In\ \Int}
\end{equation}
}

%
\insertionPage{52}{
\begin{equation}
\frac{\IB(\longtycon) = \VI\qquad \TI = \{\tycon\mapsto\VI\}}
     {\IB\ts\datatypereplspec\ra(\VI,\TI)\ \In\ \Int}
\end{equation}
}

%
\replacementPage{52}{
\begin{equation}        % exception specification
%\label{exceptionspec-dyn-rule}
\frac{ \ts\exndesc\ra\exns}
     { \IB\ts\exceptionspec\ra\exns\ \In\ \Int }
\end{equation}}{\begin{equation}        % exception specification
%\label{exceptionspec-dyn-rule}
\frac{ \ts\exndesc\ra\VI}
     { \IB\ts\exceptionspec\ra \VI\ \In\ \Int }
\end{equation}}

%
\replacementPage{52}{\begin{equation}        % structure specification
\label{structurespec-dyn-rule}
\frac{ \IB\ts\strdesc\ra\IE }
     { \IB\ts\structurespec\ra\IE\ \In\ \Int }\index{62.1}
\end{equation}}{\begin{equation}        % structure specification
\label{structurespec-dyn-rule}
\frac{ \IB\ts\strdesc\ra\SI }
     { \IB\ts\structurespec\ra\SI\ \In\ \Int }\index{62.1}
\end{equation}}

%
\deletionPage{52}{
\begin{equation}        % local specification
\label{localspec-dyn-rule}
\frac{ \IB\ts\spec_1\ra\I_1 \qquad
       \plusmap{\IB}{\of{\IE}{\I_1}}\ts\spec_2\ra\I_2 }
     { \IB\ts\localspec\ra\I_2 }
\end{equation}}

%
\deletionPage{52}{
\begin{equation}        % open specification
%\label{openspec-dyn-rule}
\frac{ \IB(\longstrid_1)=\I_1\quad\cdots\quad
       \IB(\longstrid_n)=\I_n }
     { \IB\ts\openspec\ra\I_1 + \cdots +\I_n }
\end{equation}}

%
\replacementPage{52}{
\begin{equation}        % include signature specification
%\label{inclspec-dyn-rule}
\frac{ \IB(\sigid_1)=\I_1 \quad\cdots\quad
       \IB(\sigid_n)=\I_n }
     { \IB\ts\inclspec\ra\I_1 + \cdots +\I_n }
\end{equation}}{\begin{equation}        % include signature specification
%\label{inclspec-dyn-rule}
\frac{ \IB\ts\sigexp\ra\I}
     { \IB\ts\singleinclspec\ra\I}
\end{equation}}

%
\replacementPage{52}{\begin{equation}        % sequential specification
\label{seqspec-dyn-rule}
\frac{ \IB\ts\spec_1\ra\I_1
       \qquad \plusmap{\IB}{\of{\IE}{\I_1}}\ts\spec_2\ra\I_2 }
     { \IB\ts\seqspec\ra\plusmap{\I_1}{\I_2} }
\end{equation}}{\begin{equation}        % sequential specification
\label{seqspec-dyn-rule}
\frac{ \IB\ts\spec_1\ra\I_1
       \qquad \IB+\I_1\ts\spec_2\ra\I_2 }
     { \IB\ts\seqspec\ra\plusmap{\I_1}{\I_2} }
\end{equation}}

%
\deletionPage{52}{
\noindent\comments
\begin{description}
\item{(\ref{localspec-dyn-rule}),(\ref{seqspec-dyn-rule})}
Note that $\of{\vars}{\I_1}$ and $\of{\exns}{\I_1}$ are
not needed for the evaluation of $\spec_2$.
\end{description}}

%
\replacementPage{52}{\rulesec{Value Descriptions}{\ts\valdesc\ra\vars}}{\rulesec{Value Descriptions}{\ts\valdesc\ra\VI}}

%
\replacementPage{52}{\begin{equation}         % value description
%\label{valdesc-dyn-rule}
\frac{ \langle\ts\valdesc\ra\vars\rangle }
     { \ts\var\ \langle\AND\ \valdesc\rangle\ra
       \{\var\}\ \langle\cup\ \vars\rangle }\index{62.2}
\end{equation}}{\begin{equation}         % value description
%\label{valdesc-dyn-rule}
\frac{ \langle\ts\valdesc\ra\VI\rangle }
     { \ts\vid\ \langle\AND\ \valdesc\rangle\ra
       \{\vid\mapsto\isv\}\ \langle+\,\VI\rangle }\index{62.2}
\end{equation}}

%
\insertionPage{52}{
\rulesec{Type Descriptions}{\ts\typdesc\ra\TI}
\begin{equation}
\frac{\langle\ts\typdesc\ra\TI\rangle}
     {\ts\typdescription\ra\{\tycon\mapsto\emptymap\}\langle+\TI\rangle}
\end{equation}
}

%
\insertionPage{52}{
\rulesec{Datatype Descriptions}{\ts\datdesc\ra\VI, \TI}
\begin{equation}
\frac{\ts\condesc\ra\VI\qquad\langle\ts\datdesc'\ra\VI',\TI'\rangle}
     {\ts\datdescriptiona\ra\VI\,\langle+\,\VI'\rangle, \{\tycon\mapsto\VI\}\langle+\TI'\rangle}
\end{equation}

\rulesec{Constructor Descriptions}{\ts\condesc\ra\VI}
\begin{equation}
\frac{\langle\ts\condesc\ra\VI\rangle}
     {\ts\shortconviddesc\ra\{\vid\mapsto\isc\}\,\langle+\VI\rangle}
\end{equation}
}

%
\replacementPage{53}{\rulesec{Exception Descriptions}{\ts\exndesc\ra\exns}}{\rulesec{Exception Descriptions}{\ts\exndesc\ra\VI}}

%
\replacementPage{53}{
\begin{equation}         % exception description
%\label{exndesc-dyn-rule}
\frac{ \langle\ts\exndesc\ra\exns\rangle }
     { \ts\exn\ \langle\exndesc\rangle\ra\{\exn\}\ \langle\cup\ \exns\rangle }\index{62.3}
\end{equation}}{\begin{equation}         % exception description
%\label{exndesc-dyn-rule}
\frac{ \langle\ts\exndesc\ra\VI\rangle }
     { \ts\vid\ \langle\boxml{and\ }\exndesc\rangle\ra\{\vid\mapsto\ise\}\ \langle+ \VI\rangle }\index{62.3}
\end{equation}}

%
\replacementPage{53}{\rulesec{Structure Descriptions}{\IB\ts\strdesc\ra\IE}}{\rulesec{Structure Descriptions}{\IB\ts\strdesc\ra\SI}}

%
\replacementPage{53}{\begin{equation}
%\label{strdesc-dyn-rule}
\frac{ \IB\ts\sigexp\ra\I\qquad\langle\IB\ts\strdesc\ra\IE\rangle }
     { \IB\ts\strdescription\ra\{\strid\mapsto\I\}\ \langle +\ \IE\rangle }\index{62.4}
\end{equation}}{\begin{equation}
%\label{strdesc-dyn-rule}
\frac{ \IB\ts\sigexp\ra\I\qquad\langle\IB\ts\strdesc\ra\SI\rangle }
     { \IB\ts\strdescription\ra\{\strid\mapsto\I\}\ \langle +\ \SI\rangle }\index{62.4}
\end{equation}}

%
\replacementPage{53}{
\begin{equation}        % functor binding
%\label{funbind-dyn-rule}
\frac{
      \begin{array}{c}
      \Inter\B\ts\sigexp\ra\I\qquad
      \langle\Inter\B+\{\strid\mapsto\I\} \ts\sigexp'\ra\I'\rangle \\
       \langle\langle\B\ts\funbind\ra\F\rangle\rangle
      \end{array}
     }
     {
      \begin{array}{c}
       \B\ts\funstrbinder\ \optfunbind\ra\\
       \qquad\qquad \qquad
              \{\funid\mapsto(\strid:\I,\strexp\langle:\I'\rangle,\B)\}
              \ \langle\langle +\ \F\rangle\rangle
      \end{array}
     }\index{62.5}
\end{equation}}{\begin{equation}        % functor binding
%\label{funbind-dyn-rule}
\frac{
      \Inter\B\ts\sigexp\ra\I\qquad
      \langle\IB\ts\funbind\ra\F\rangle
     }
     {
      \begin{array}{c}
       \IB\ts\barefunstrbinder\ \optfunbinda\ra\\
       \qquad\qquad \qquad
              \{\funid\mapsto(\strid:\I,\strexp,\B)\}
              \ \langle +\ \F\rangle
      \end{array}
     }\index{62.5}
\end{equation}}

%
\deletionPage{53}{
\vspace{6pt}
\begin{equation}        % empty functor declaration
%\label{emptyfundec-dyn-rule}
\frac{}
     { \B\ts\emptyfundec\ra\emptymap }
\end{equation}

\vspace{6pt}
\begin{equation}        % sequential functor declaration
%\label{seqfundec-dyn-rule}
\frac{ \B\ts\fundec_1\ra\F_1\qquad
       \B+\F_1\ts\fundec_2\ra\F_2 }
     { \B\ts\seqfundec\ra\plusmap{\F_1}{\F_2} }
\end{equation}}

%
\replacementPage{53}{
\begin{equation}        % structure-level declaration
%\label{strdectopdec-dyn-rule}
\frac{\B\ts\strdec\ra\E}
     {\B\ts\strdec\ra\E\ \In\ \Basis
     }\index{63.2}
\end{equation}

\vspace{6pt}
\begin{equation}        % signature declaration
%\label{sigdectopdec-dyn-rule}
\frac{\Inter\B\ts\sigdec\ra\G}
     {\B\ts\sigdec\ra\G\ \In\ \Basis
     }
\end{equation}

\vspace{6pt}
\begin{equation}        % functor declaration
%\label{fundectopdec-dyn-rule}
\frac{\B\ts\fundec\ra\F}
     {\B\ts\fundec\ra\F\ \In\ \Basis
     }
\end{equation}}{\begin{equation}        % structure-level declaration
%\label{strdectopdec-dyn-rule}
\frac{\B\ts\strdec\ra\E\quad\B' =\E\ \In\ \Basis\quad\langle \B+\B'\ts\topdec\ra\B''\rangle }
     {\B\ts\strdecintopdec\ra\B'\langle'\rangle
     }\index{63.2}
\end{equation}

\vspace{6pt}
\begin{equation}        % signature declaration
%\label{sigdectopdec-dyn-rule}
\frac{\Inter\B\ts\sigdec\ra\G\quad B' = \G\ \In\ \Basis\quad
       \langle \B + \B'\ts\topdec\ra\B''\rangle}
     {\B\ts\sigdecintopdec\ra \B'\langle'\rangle 
     }
\end{equation}

\vspace{6pt}
\begin{equation}        % functor declaration
%\label{fundectopdec-dyn-rule}
\frac{\B\ts\fundec\ra\F\quad \B' = \F\ \In\ \Basis\quad
       \langle \B + \B'\ts\topdec\ra\B''\rangle}
     {\B\ts\fundecintopdec\ra\B'\langle'\rangle
     }
\end{equation}}

%
\deletionPage{54}{ or external intervention}

%
\deletionPage{55}{
\item Omit\index{65.5} the {\OPEN} declaration from the syntax class of
declarations $\dec$
\item Restrict the long identifier classes to identifiers, i.e.
omit qualified identifiers.}

%
\deletionPage{55}{
This means that  several components of a basis, for example the
signature and functor environments, are irrelevant to the execution
of a core language program.}

%
\insertionPage{56}{\tyvarseq\ }

%
\insertionPage{56}{\tyvarseq\ }

%
\insertionPage{56}{\tyvarseq\ }

%
\insertionPage{56}{\tyvarseq\ }

%
\deletionPage{56}{
These forms are currently more experimental than the bare syntax of modules, 
but we recommend implementers to include them so that they can be
tested in practice.
In the derived forms for functor bindings and functor signature expressions,
$\strid$ is a new structure identifier and
the form of $\sigexp'$ depends
on the form of $\sigexp$ as follows. 
If $\sigexp$ is simply a signature identifier
$\sigid$, then $\sigexp'$ is also $\sigid$; otherwise $\sigexp$ must take
the form  ~$\SIG\ \spec_1\ \END$~,
and then $\sigexp'$ is
$\mbox{\SIG\ \LOCAL\ \OPEN\ \strid\ \IN\ $\spec_1$\ \END\ \END}$.
%(where $\strid$ is new).
}

%
\insertionPage{56}{
Finally, Figure~\ref{spec-der-forms-fig} shows the derived forms for specifications.
The last derived form for specifications allows sharing between structure
identifiers as a shorthand for type sharing specifications. 
%Standard ML no 
%longer has a semantic notion of structure sharing; 
%however, for compatability with Standard ML '90, a weaker form
%of structure sharing specification is provided. 
The phrase
\[
\boxml{$\spec$ sharing $\longstrid_1$ = $\cdots$ = $\longstrid_k$}
\]
is a derived form whose equivalent form is
\pagebreak
%\medskip

\halign{\qquad\ignore{#}&#\hfil\cr
\boxml{$\spec$ sharing $\longstrid_1$ = $\cdots$} & \boxml{$\spec$}\cr
\boxml{\ \ \qquad\qquad\qquad\qquad\qquad = $\longstrid_k$} & \boxml{\ sharing type $\longtycon_{1}$ =  $\longtycon_{1}'$}\cr
  & \boxml{\ $\cdots$}\cr
  & \boxml{\ sharing type $\longtycon_{m}$ =  $\longtycon_{m}'$}\cr}
\medskip

\noindent
determined as follows. 
First, note that  $\spec$  specifies a set of 
     (possibly long) type constructors and structure identifiers, either 
     directly or via signature identifiers and $\INCLUDE$ specifications.  
     Then the equivalent form contains all type-sharing constraints 
     of the form 
\[\boxml{sharing type $\longstrid_i.\longtycon$ = $\longstrid_j.longtycon$}\]
     $(1\leq i<j\leq k)$,  such that both sides of the equation are long type 
     constructors specified by  $\spec$. 

     The meaning of the derived form does not depend on the order of the 
     type-sharing constraints in the equivalent form.

}

%
\replacementPage{57}{\begin{tabular}{|l|l|l}
\multicolumn{1}{c}{Derived Form} & \multicolumn{1}{c}{Equivalent Form} &
\multicolumn{1}{c}{}\\
\multicolumn{3}{c}{}\\
\multicolumn{2}{l}{{\bf Expressions} \exp}\\
%\multicolumn{2}{l}{EXPRESSIONS \exp}\\
\cline{1-2}
\ml{()}         & \ml{\lttbrace\ \rttbrace} \\
\cline{1-2}
\ml{(}$\exp_1$ \ml{,} $\cdots$ \ml{,} $\exp_\n$\ml{)}
            & \ml{\lttbrace 1=}$\exp_1$\ml{,}\ $\cdots$\ml{,}\
                             $\overline{n}$\ml{=}$\exp_\n$\ml{\rttbrace}
                                                           & $(\n\geq 2)$\\
\cline{1-2}
\ml{\#}\ \lab      & \FN\ \ml{\lttbrace}\lab\ml{=}\var\ml{,...\rttbrace\  => }\var
                                                           & (\var\ new)\\
%\cline{1-2}
%\RAISE\ \longexn    & \RAISE\ \longexn\ \WITH\ \ml{()} \\
\cline{1-2}
\CASE\ \exp\ \OF\ \match
                & \ml{(}\FN\ \match\ml{)(}\exp\ml{)} \\
\cline{1-2}
\IF\ $\exp_1$\ \THEN\ $\exp_2$\ \ELSE\ $\exp_3$
                & \CASE\ $\exp_1$\ \OF\ \TRUE\ \ml{=>}\ \exp$_2$\\
                & \ \ \qquad\qquad\ml{|}\ \FALSE\ \ml{=>}\ \exp$_3$ \\
\cline{1-2}
\exp$_1$\ \ORELSE\ \exp$_2$
                & \IF\ \exp$_1$\ \THEN\ \TRUE\ \ELSE\ \exp$_2$ \\
\cline{1-2}
\exp$_1$\ \ANDALSO\ \exp$_2$
                & \IF\ \exp$_1$\ \THEN\ \exp$_2$\ \ELSE\ \FALSE \\
\cline{1-2}
\ml{(}$\exp_1$ \ml{;} $\cdots$ \ml{;} $\exp_\n$ \ml{;} \exp\ml{)}\
                & \CASE\ \exp$_1$\ \OF\ \ml{(\_) =>}
                                                           & $(\n\geq 1)$ \\
                & \qquad$\cdots$ \\
                & \CASE\ \exp$_n$\ \OF\ \ml{(\_) =>}\ \exp \\
\cline{1-2}
\LET\ \dec\ \IN
                & \LET\ \dec\ \IN                          & $(\n\geq 2)$ \\
\qquad$\exp_1$ \ml{;} $\cdots$ \ml{;} $\exp_\n$ \END
                & \ \ \ml{(}$\exp_1$ \ml{;} $\cdots$ \ml{;} $\exp_\n$\ml{)}\
                                                                         \END\\
\cline{1-2}
\WHILE\ \exp$_1$\ \DO\ \exp$_2$
                & \LET\ \VAL\ \REC\ \var\ \ml{=}\ \FN\ \ml{() =>}
                                                           & (\var\ new)\\
                & \ \ \IF\ \exp$_1$\ \THEN\
                    \ml{(}\exp$_2$\ml{;}\var\ml{())}\ \ELSE\ \ml{()} \\
                & \ \ \IN\ \var\ml{()}\ \END\\
\cline{1-2}
\ml{[}$\exp_1$ \ml{,} $\cdots$ \ml{,} $\exp_\n$\ml{]}
                & \exp$_1$\ \ml{::}\ $\cdots$\ \ml{::}\ \exp$_n$\
                            \ml{::}\ \NIL                 & $(n\geq 0)$ \\
\cline{1-2}
\multicolumn{3}{c}{}\\
\end{tabular}}{\begin{tabular}{|l|l|l}
\multicolumn{1}{c}{Derived Form} & \multicolumn{1}{c}{Equivalent Form} &
\multicolumn{1}{c}{}\\
\multicolumn{3}{c}{}\\
\multicolumn{2}{l}{{\bf Expressions} \exp}\\
%\multicolumn{2}{l}{EXPRESSIONS \exp}\\
\cline{1-2}
\ml{()}         & \ml{\lttbrace\ \rttbrace} \\
\cline{1-2}
\ml{(}$\exp_1$ \ml{,} $\cdots$ \ml{,} $\exp_\n$\ml{)}
            & \ml{\lttbrace 1=}$\exp_1$\ml{,}\ $\cdots$\ml{,}\
                             $\overline{n}$\ml{=}$\exp_\n$\ml{\rttbrace}
                                                           & $(\n\geq 2)$\\
\cline{1-2}
\ml{\#}\ \lab      & \FN\ \ml{\lttbrace}\lab\ml{=}\vid\ml{,...\rttbrace\  => }\vid
                                                           & (\vid\ new)\\
%\cline{1-2}
%\RAISE\ \longexn    & \RAISE\ \longexn\ \WITH\ \ml{()} \\
\cline{1-2}
\CASE\ \exp\ \OF\ \match
                & \ml{(}\FN\ \match\ml{)(}\exp\ml{)} \\
\cline{1-2}
\IF\ $\exp_1$\ \THEN\ $\exp_2$\ \ELSE\ $\exp_3$
                & \CASE\ $\exp_1$\ \OF\ \TRUE\ \ml{=>}\ \exp$_2$\\
                & \ \ \qquad\qquad\ml{|}\ \FALSE\ \ml{=>}\ \exp$_3$ \\
\cline{1-2}
\exp$_1$\ \ORELSE\ \exp$_2$
                & \IF\ \exp$_1$\ \THEN\ \TRUE\ \ELSE\ \exp$_2$ \\
\cline{1-2}
\exp$_1$\ \ANDALSO\ \exp$_2$
                & \IF\ \exp$_1$\ \THEN\ \exp$_2$\ \ELSE\ \FALSE \\
\cline{1-2}
\ml{(}$\exp_1$ \ml{;} $\cdots$ \ml{;} $\exp_\n$ \ml{;} \exp\ml{)}\
                & \CASE\ \exp$_1$\ \OF\ \ml{(\_) =>}
                                                           & $(\n\geq 1)$ \\
                & \qquad$\cdots$ \\
                & \CASE\ \exp$_n$\ \OF\ \ml{(\_) =>}\ \exp \\
\cline{1-2}
\LET\ \dec\ \IN
                & \LET\ \dec\ \IN                          & $(\n\geq 2)$ \\
\qquad$\exp_1$ \ml{;} $\cdots$ \ml{;} $\exp_\n$ \END
                & \ \ \ml{(}$\exp_1$ \ml{;} $\cdots$ \ml{;} $\exp_\n$\ml{)}\
                                                                         \END\\
\cline{1-2}
\WHILE\ \exp$_1$\ \DO\ \exp$_2$
                & \LET\ \VAL\ \REC\ \vid\ \ml{=}\ \FN\ \ml{() =>}
                                                           & (\vid\ new)\\
                & \ \ \IF\ \exp$_1$\ \THEN\
                    \ml{(}\exp$_2$\ml{;}\vid\ml{())}\ \ELSE\ \ml{()} \\
                & \ \ \IN\ \vid\ml{()}\ \END\\
\cline{1-2}
\ml{[}$\exp_1$ \ml{,} $\cdots$ \ml{,} $\exp_\n$\ml{]}
                & \exp$_1$\ \ml{::}\ $\cdots$\ \ml{::}\ \exp$_n$\
                            \ml{::}\ \NIL                 & $(n\geq 0)$ \\
\cline{1-2}
\multicolumn{3}{c}{}\\
\end{tabular}}

%
\replacementPage{58}{
\id$\langle$\ml{:}\ty$\rangle
    \ \langle\AS\ \pat\rangle
    \ \langle$\ml{,} \labpats$\rangle$
                }{\vid$\langle$\ml{:}\ty$\rangle
    \ \langle\AS\ \pat\rangle
    \ \langle$\ml{,} \labpats$\rangle$}

%
\replacementPage{58}{\id\ml{ = }\id$\langle$\ml{:}\ty$\rangle
                                 \ \langle\AS\ \pat\rangle
                                 \ \langle$\ml{,} \labpats$\rangle$}{\vid\ml{ = }\vid$\langle$\ml{:}\ty$\rangle
                                 \ \langle\AS\ \pat\rangle
                                 \ \langle$\ml{,} \labpats$\rangle$}

%
\replacementPage{58}{
\begin{tabular}{|l|l|}
\multicolumn{1}{c}{Derived Form} & \multicolumn{1}{c}{Equivalent Form}\\
\multicolumn{2}{c}{}\\
\multicolumn{2}{l}{{\bf Function-value Bindings} \fvalbind}\\
%\multicolumn{2}{l}{FUNCTION-VALUE BINDINGS \fvalbind}\\
\hline
               & $\langle\OP\rangle$\var\ \ml{=} \FN\ \var$_1$\ml{=>} $\cdots$
                              \FN\ \var$_n$\ml{=>} \\
               & \CASE\
                 \ml{(}\var$_1$\ml{,} $\cdots$ \ml{,} \var$_n$\ml{)} \OF \\
\ \ $\langle\OP\rangle\var\ \atpat_{11}\cdots\atpat_{1n}
                                              \langle$\ml{:}\ty$\rangle$
                                              \ml{=} \exp$_1$
               & \ \ \ml{(}\atpat$_{11}$\ml{,}$\cdots$\ml{,}\atpat$_{1n}$
                             \ml{)=>}\exp$_1\langle$\ml{:}\ty$\rangle$\\
\ml{|}$\langle\OP\rangle\var\ \atpat_{21}\cdots\atpat_{2n}
                                              \langle$\ml{:}\ty$\rangle$
                                              \ml{=} \exp$_2$
               & \ml{|(}\atpat$_{21}$\ml{,}$\cdots$\ml{,}\atpat$_{2n}$
                             \ml{)=>}\exp$_2\langle$\ml{:}\ty$\rangle$\\
\ml{|}\qquad$\cdots\qquad\cdots$
               & \ml{|}\qquad$\cdots\qquad\cdots$\\
\ml{|}$\langle\OP\rangle\var\ \atpat_{m1}\cdots\atpat_{mn}
                                              \langle$\ml{:}\ty$\rangle$
                                              \ml{=} \exp$_m$
               & \ml{|(}\atpat$_{m1}$\ml{,}$\cdots$\ml{,}\atpat$_{mn}$
                             \ml{)=>}\exp$_m\langle$\ml{:}\ty$\rangle$\\
\qquad\qquad\qquad$\langle\AND\ \fvalbind\rangle$
               & \qquad\qquad\qquad$\langle\AND\ \fvalbind\rangle$\\
\hline
\multicolumn{2}{r}{($m,n\geq1$; $\var_1,\cdots,\var_n$ distinct and new)}\\
\multicolumn{2}{c}{}\\
\multicolumn{2}{l}{{\bf Declarations} \dec}\\
%\multicolumn{2}{l}{DECLARATIONS \dec}\\
\hline
\FUN\  \fvalbind
               & \VAL\ \REC\ \fvalbind  \\
\hline
\DATATYPE\ \datbind\ \WITHTYPE\ \typbind
               & \DATATYPE\ \datbind$\/'$\ \ml{;}\ \TYPE\ \typbind \\
\hline
\ABSTYPE\ \datbind\ \WITHTYPE\ \typbind
               & \ABSTYPE\ \datbind$\/'$ \\
\qquad\qquad\WITH\ \dec\ \END
               & \qquad\WITH\ \TYPE\ \typbind\ \ml{;}\ \dec\ \END\\
\hline
\multicolumn{2}{r}{(see note in text concerning \datbind$\/'$)}\\
\multicolumn{2}{c}{}\\
\end{tabular}}{\begin{tabular}{|l|l|}
\multicolumn{1}{c}{Derived Form} & \multicolumn{1}{c}{Equivalent Form}\\
\multicolumn{2}{c}{}\\
\multicolumn{2}{l}{{\bf Function-value Bindings} \fvalbind}\\
%\multicolumn{2}{l}{FUNCTION-VALUE BINDINGS \fvalbind}\\
\hline
               & $\langle\OP\rangle$\vid\ \ml{=} \FN\ \vid$_1$\ml{=>} $\cdots$
                              \FN\ \vid$_n$\ml{=>} \\
               & \CASE\
                 \ml{(}\vid$_1$\ml{,} $\cdots$ \ml{,} \vid$_n$\ml{)} \OF \\
\ \ $\langle\OP\rangle\vid\ \atpat_{11}\cdots\atpat_{1n}
                                              \langle$\ml{:}\ty$\rangle$
                                              \ml{=} \exp$_1$
               & \ \ \ml{(}\atpat$_{11}$\ml{,}$\cdots$\ml{,}\atpat$_{1n}$
                             \ml{)=>}\exp$_1\langle$\ml{:}\ty$\rangle$\\
\ml{|}$\langle\OP\rangle\vid\ \atpat_{21}\cdots\atpat_{2n}
                                              \langle$\ml{:}\ty$\rangle$
                                              \ml{=} \exp$_2$
               & \ml{|(}\atpat$_{21}$\ml{,}$\cdots$\ml{,}\atpat$_{2n}$
                             \ml{)=>}\exp$_2\langle$\ml{:}\ty$\rangle$\\
\ml{|}\qquad$\cdots\qquad\cdots$
               & \ml{|}\qquad$\cdots\qquad\cdots$\\
\ml{|}$\langle\OP\rangle\vid\ \atpat_{m1}\cdots\atpat_{mn}
                                              \langle$\ml{:}\ty$\rangle$
                                              \ml{=} \exp$_m$
               & \ml{|(}\atpat$_{m1}$\ml{,}$\cdots$\ml{,}\atpat$_{mn}$
                             \ml{)=>}\exp$_m\langle$\ml{:}\ty$\rangle$\\
\qquad\qquad\qquad$\langle\AND\ \fvalbind\rangle$
               & \qquad\qquad\qquad$\langle\AND\ \fvalbind\rangle$\\
\hline
\multicolumn{2}{r}{($m,n\geq1$; $\vid_1,\cdots,\vid_n$ distinct and new)}\\
\multicolumn{2}{c}{}\\
\multicolumn{2}{l}{{\bf Declarations} \dec}\\
%\multicolumn{2}{l}{DECLARATIONS \dec}\\
\hline
\FUN\ \tyvarseq\ \fvalbind
               & \VAL\ \tyvarseq\ \REC\ \fvalbind  \\
\hline
\DATATYPE\ \datbind\ \WITHTYPE\ \typbind
               & \DATATYPE\ \datbind$\/'$\ \ml{;}\ \TYPE\ \typbind \\
\hline
\ABSTYPE\ \datbind\ \WITHTYPE\ \typbind
               & \ABSTYPE\ \datbind$\/'$ \\
\qquad\qquad\WITH\ \dec\ \END
               & \qquad\WITH\ \TYPE\ \typbind\ \ml{;}\ \dec\ \END\\
\hline
\multicolumn{2}{r}{(see note in text concerning \datbind$\/'$)}\\
\multicolumn{2}{c}{}\\
\end{tabular}}

%
\replacementPage{59}{
\begin{tabular}{|l|l|}
\multicolumn{1}{c}{Derived Form} & \multicolumn{1}{c}{Equivalent Form} \\
\multicolumn{2}{c}{}\\
\multicolumn{2}{l}{{\bf Structure  Expressions} \strexp}\\
%\multicolumn{2}{l}{STRUCTURE EXPRESSIONS \strexp}\\
\cline{1-2}
\funappdec & \mbox{\funid\ \ml{(} \STRUCT\ \strdec\ \END\ \ml{)}}\\
\cline{1-2}
\multicolumn{2}{c}{}\\
%\multicolumn{2}{l}{FUNCTOR BINDINGS \funbind}\\
\multicolumn{2}{l}{{\bf Functor Bindings} \funbind}\\
\cline{1-2}        
\mbox{\funid\ \ml{(}\ \spec\ \ml{)}\ $\langle$\ml{:}\ \sigexp$\rangle$\ \ml{=}}&
\mbox{\funid\ \ml{(}\ \strid\ \ml{:} \SIG\ \spec\ \END\ \ml{)} 
              $\langle$\ml{:}\ $\sigexp'\rangle$\ \ml{=}}\\
\mbox{\ \ \strexp\ $\langle$\AND\ \funbind$\rangle$} &
  \mbox{\ \ \LET\ \OPEN\ \strid\ \IN\ \strexp\ \END\ $\langle$\AND\ \funbind$\rangle$} \\
\cline{1-2}
\multicolumn{2}{r}{($\strid$ new; see note in text concerning $\sigexp'$)}\\
\multicolumn{2}{c}{}\\
\multicolumn{2}{l}{{\bf Functor Signature Expressions} \funsigexp}\\
%\multicolumn{2}{l}{FUNCTOR SIGNATURES \funsigexp}\\
\cline{1-2}
\longfunsigexp & \mbox{\ml{(} \strid\ \ml{:}\ \SIG\ \spec\ \END\ \ml{)}
                \ml{:}\ \sigexp$'$} \\
\cline{1-2}
\multicolumn{2}{r}{($\strid$ new; see note in text concerning $\sigexp'$)}\\
\multicolumn{2}{c}{}\\
\multicolumn{2}{l}{{\bf Top-level Declarations} \topdec}\\
\cline{1-2}
\exp           & \VAL\ \ml{it =} \exp  \\
\cline{1-2}
\multicolumn{2}{c}{}\\
\end{tabular}}{
\begin{tabular}{|l|l|}
\multicolumn{1}{c}{Derived Form} & \multicolumn{1}{c}{Equivalent Form} \\
\multicolumn{2}{c}{}\\
\multicolumn{2}{l}{{\bf Structure  Bindings} \strbind}\\
\cline{1-2}
\derivedstrbinder & \equivalentstrbinder\\
\cline{1-2}
\derivedabststrbinder & \equivalentabststrbinder\\
\cline{1-2}
\multicolumn{2}{c}{}\\
\multicolumn{2}{l}{{\bf Structure  Expressions} \strexp}\\
\cline{1-2}
\funappdec & \mbox{\funid\ \ml{(} \STRUCT\ \strdec\ \END\ \ml{)}}\\
\cline{1-2}
\multicolumn{2}{c}{}\\
%\multicolumn{2}{l}{FUNCTOR BINDINGS \funbind}\\
\multicolumn{2}{l}{{\bf Functor Bindings} \funbind}\\
\cline{1-2}        
\mbox{\funid\ \ml{(}\strid\ml{:}\sigexp\ml{)}\ml{:} $\sigexp'$ \ml{=}}&
\mbox{\funid\ \ml{(}\strid\ \ml{:} \sigexp\ml{)} \ \ml{=}}\\
\mbox{\ \ \strexp\ $\langle$\AND\ \funbind$\rangle$} &
  \mbox{\ \ \strexp\ml{:}$\sigexp'$\  $\langle$\AND\ \funbind$\rangle$} \\
\cline{1-2}        
\mbox{\funid\ \ml{(}\strid\ml{:}\sigexp\ml{)}\ABSTRACT $\sigexp'$ \ml{=}}&
\mbox{\funid\ \ml{(}\strid\ \ml{:} \sigexp\ml{)} \ \ml{=}}\\
\mbox{\ \ \strexp\ $\langle$\AND\ \funbind$\rangle$} &
  \mbox{\ \ \strexp\ABSTRACT$\sigexp'$\  $\langle$\AND\ \funbind$\rangle$} \\
\cline{1-2}        
\mbox{\funid\ \ml{(}\ \spec\ \ml{)}\ $\langle$\ml{:}\ \sigexp$\rangle$\ \ml{=}}&
\mbox{\funid\ \ml{(}\ $\strid_\nu$\ \ml{:} \SIG\ \spec\ \END\ \ml{)} 
              \ \ml{=}}\\
\mbox{\ \ \strexp\ $\langle$\AND\ \funbind$\rangle$} &
  \mbox{\ \ \LET\ \OPEN\ $\strid_\nu$ \IN\ \strexp$\langle$\ml{:}\ $\sigexp\rangle$}\\
& \mbox{\ \ \END\ $\langle$\AND\ \funbind$\rangle$} \\
\cline{1-2}        
\mbox{\funid\ \ml{(}\ \spec\ \ml{)}\ $\langle$\ABSTRACT\ \sigexp$\rangle$\ \ml{=}}&
\mbox{\funid\ \ml{(}\ $\strid_\nu$\ \ml{:} \SIG\ \spec\ \END\ \ml{)} 
              \ \ml{=}}\\
\mbox{\ \ \strexp\ $\langle$\AND\ \funbind$\rangle$} &
  \mbox{\ \ \LET\ \OPEN\ $\strid_\nu$ \IN\ \strexp$\langle$\ABSTRACT $\sigexp\rangle$}\\
& \mbox{\ \ \END\ $\langle$\AND\ \funbind$\rangle$} \\
\cline{1-2}
\multicolumn{2}{r}{($\strid_\nu$ new)}\\
\multicolumn{2}{c}{}\\
\multicolumn{2}{l}{{\bf Programs} \program}\\
\cline{1-2}
$\exp\boxml{;}\langle\program\rangle$           & $\VAL\ \boxml{it =}\; \exp\boxml{;}\langle\program\rangle$\\
\cline{1-2}
\multicolumn{2}{c}{}\\
\end{tabular}}

%
\insertionPage{60}{
\begin{tabular}{|l|l|}
\multicolumn{1}{c}{Derived Form} & \multicolumn{1}{c}{Equivalent Form} \\
\multicolumn{2}{c}{}\\
\multicolumn{2}{l}{{\bf Specifications} \spec}\\ 
\cline{1-2}
$\TYPE\;\tyvarseq\;\tycon\,\boxml{=}\,\ty$ & \boxml{include}\\
 &\boxml{\ sig $\TYPE\;\tyvarseq\;\tycon$}\\
 &\boxml{\ end where type $\tyvarseq\;\tycon\,\boxml{=}\,\ty$} \\
\cline{1-2}
\boxml{type $\tyvarseq_1\;\tycon_1$ = $\ty_1$} & \boxml{type $\tyvarseq_1\;\tycon_1$ = $\ty_1$}\\
\boxml{ and $\cdots$} & \boxml{type $\cdots$}\\
\boxml{ $\cdots$} & \boxml{ $\cdots$}\\
\boxml{ and $\tyvarseq_n\;\tycon_n$ = $\ty_n$} & \boxml{type $\tyvarseq_n\;\tycon_n$ = $\ty_n$}\\
\cline{1-2}
$\inclspec$ ${}_{(n\geq2)}$ & $\INCLUDE\,\sigid_1; \cdots\, ; \INCLUDE \,sigid_n$\\
\cline{1-2}
\boxml{$\spec$ sharing $\longstrid_1$ = $\cdots$} & \boxml{$\spec$}\\
\boxml{\ \ \qquad\qquad\qquad\qquad\qquad = $\longstrid_k$} & \boxml{\ sharing type $\longtycon_{1}$ = }\\
  & \boxml{\ \ \ \ \ \qquad\qquad\qquad\qquad $\longtycon_{1}'$}\\
  & \boxml{\ $\cdots$}\\
  & \boxml{\ sharing type $\longtycon_{m}$ = }\\
  & \boxml{\ \ \ \ \ \qquad\qquad\qquad\qquad  $\longtycon_{m}'$}\\
\cline{1-2}
\multicolumn{2}{r}{\vrule height14pt depth0pt width0pt(see note in text concerning $\longtycon_{1},\ldots,\longtycon_{m}'$)}\\
%\multicolumn{2}{r}{($n\geq 1$)}\\
\multicolumn{2}{c}{}\\
\end{tabular}}

%
\replacementPage{61}{Figure~\ref{functor-der-forms-fig}}{Figures~\ref{functor-der-forms-fig} and \ref{spec-der-forms-fig}}

%
\deletionPage{63}{\begin{minipage}{\textwidth}\halign{\indent$#$\hfil&$#$\hfil&$#$\hfil&#\hfil\cr
        & 	& \opp\longvar	& value variable\cr
	&	& \opp\longcon	& value constructor\cr
        &       & \opp\longexn  & exception constructor\cr}\end{minipage}}

%
\insertionPage{63}{$\opp\longvid$\quad value identifier}

%
\replacementPage{63}{$\inexp_1$\ \id\ $\inexp_2$}{$\inexp_1$\ \vid\ $\inexp_2$}

%
\replacementPage{64}{\valdec}{\explicitvaldec}

%
\replacementPage{64}{\FUN\ \fvalbind}{\FUN\ \tyvarseq\  \fvalbind}

%
\insertionPage{64}{$\datatyperepldecb$\quad datatype replication}

%
\replacementPage{64}{\longinfix}{\newlonginfix}

%
\replacementPage{64}{\longinfixr}{\newlonginfixr}

%
\replacementPage{64}{\longnonfix}{\newlongnonfix}

%
\deletionPage{64}{\begin{minipage}{\textwidth}\halign{\indent$#$\hfil&$#$\hfil&#\hfil&#\hfil\cr
\fvalbind& ::=  & \ \ $\langle\OP\rangle\var\ \atpat_{11}\cdots\atpat_{1n}
                  \langle$\ml{:}\ty$\rangle$\ml{=}\exp$_1$ & $m,n\geq 1$\cr
        &       & \ml{|}$\langle\OP\rangle\var\ \atpat_{21}\cdots\atpat_{2n}
                  \langle$\ml{:}\ty$\rangle$\ml{=}\exp$_2$ & See also note
                                                                     below\cr
        &       & \ml{|}\qquad$\cdots\qquad\cdots$ &\cr
        &       & \ml{|}$\langle\OP\rangle\var\ \atpat_{m1}\cdots\atpat_{mn}
                  \langle$\ml{:}\ty$\rangle$\ml{=}\exp$_m$ &\cr
        &       & \qquad\qquad\qquad$\langle\AND\ \fvalbind\rangle$ &\cr}\end{minipage}}

%
\insertionPage{64}{\begin{minipage}{\textwidth}\halign{\indent$#$\hfil&$#$\hfil&#\hfil&#\hfil\cr
\fvalbind& ::=  & \ \ $\langle\OP\rangle\vid\ \atpat_{11}\cdots\atpat_{1n}
                  \langle$\ml{:}\ty$\rangle$\ml{=}\exp$_1$ & $m,n\geq 1$\cr
        &       & \ml{|}$\langle\OP\rangle\vid\ \atpat_{21}\cdots\atpat_{2n}
                  \langle$\ml{:}\ty$\rangle$\ml{=}\exp$_2$ & See also note
                                                                     below\cr
        &       & \ml{|}\qquad$\cdots\qquad\cdots$ &\cr
        &       & \ml{|}$\langle\OP\rangle\vid\ \atpat_{m1}\cdots\atpat_{mn}
                  \langle$\ml{:}\ty$\rangle$\ml{=}\exp$_m$ &\cr
        &       & \qquad\qquad\qquad$\langle\AND\ \fvalbind\rangle$ &\cr}\end{minipage}}

%
\replacementPage{64}{\opp\longconstrs}{\opp\longvidconstrs}

%
\replacementPage{64}{\generativeexnbind}{\generativeexnvidbind}

%
\replacementPage{64}{\eqexnbind}{\eqexnvidbind}

%
\replacementPage{64}{In the $\fvalbind$ form, if $\var$ has infix status then either
~\OP~ must be present, or $\var$ must be infixed.  Thus, at the start of
any clause, ``~\OP\ \var\ \ml{(}\atpat\ml{,}\atpat$'$\ml{)} $\cdots$'' may be
written
``\ml{(}\atpat\ \var\ \atpat$'$\ml{)} $\cdots$''; the parentheses may also be
dropped if ``\ml{:}\ty'' or ``\ml{=}'' follows immediately.}{In the $\fvalbind$ form, if $\vid$ has infix status then either
~\OP~ must be present, or $\vid$ must be infixed.  Thus, at the start of
any clause, ``~\OP\ \vid\ \ml{(}\atpat\ml{,}\atpat$'$\ml{)} $\cdots$'' may be
written
``\ml{(}\atpat\ \vid\ \atpat$'$\ml{)} $\cdots$''; the parentheses may also be
dropped if ``\ml{:}\ty'' or ``\ml{=}'' follows immediately.}

%
\deletionPage{65}{\begin{minipage}{\textwidth}\halign{\indent$#$\hfil&$#$\hfil&$#$\hfil&$#$\hfil\cr
  	&	& \opp\var  	& variable\cr
	&	& \opp\longcon  & constant\cr
        &       & \opp\longexn  & exception constant\cr}\end{minipage}}

%
\insertionPage{65}{$\opp\longvid$\quad value identifier}

%
\replacementPage{65}{\id}{\vid}

%
\deletionPage{65}{\begin{minipage}{\textwidth}\halign{\indent$#$\hfil&$#$\hfil&$#$\hfil&#\hfil\cr
	&	& \opp\conpat	& value construction\cr
        &       & \opp\exconpat  & exception construction\cr
	&	& \infpat       & infixed value construction\cr
        &       & \infexpat     & infixed exception construction\cr
	&	& \typedpat	& typed\cr
	&	& \opp\layeredpat	& layered\cr}\end{minipage}}

%
\insertionPage{65}{\begin{minipage}{\textwidth}\halign{\indent$#$\hfil&$#$\hfil&$#$\hfil&#\hfil\cr
	&	& \opp\vidpat	& constructed value\cr
	&	& \vidinfpat       & constructed value (infix)\cr
	&	& \typedpat	& typed\cr
	&	& \opp\layeredvidpat	& layered\cr}\end{minipage}}

%
\insertionPage{65}{(R)}

%
\insertionPage{66}{
In this appendix (and the next) we define a minimal initial basis for
execution. Richer bases may be provided by libraries.}

%
\insertionPage{66}{
We\index{73.1} shall indicate components of the initial basis by the subscript 0.
The initial static basis is
$\B_0 = \T_0,\F_0,\G_0,\E_0$,
where $F_0 = \emptymap$, $\G_0 = \emptymap$ and 
$$\T_0\ =\ \{\BOOL,\INT,\REAL,\STRING,\CHAR,\WORD,\LIST,\REF,\EXCN\}$$
The members of $\T_0$ are type names, not type constructors; for convenience
we have used type-constructor identifiers
to stand also for the type names which are bound to them in the initial
static type environment $\TE_0$.  Of these type names,
~\LIST~ and ~\REF~
have arity 1, the rest have arity 0;  
all except $\EXCN$ admit equality.
Finally, $\E_0 = (\SE_0,\TE_0,\VE_0)$, where $\SE_0 = \emptymap$, 
while $\TE_0$ and $\VE_0$ are shown in Figures~\ref{stat-te} and \ref{stat-ve},
respectively.
}

%
\deletionPage{66}{
We\index{73.1} shall indicate components of the initial basis by the subscript 0.
The initial static basis is
\[ \B_0\ =\ (\M_0,\T_0),\F_0,\G_0,\E_0\]
where
\begin{itemize}
\item $\M_0\ =\ \emptyset$
\item $\T_0\ =\ \{\BOOL,\INT,\REAL,\STRING,\LIST,\REF,\EXCN,\INSTREAM,\OUTSTREAM\}$
\item $\F_0\ =\ \emptymap$
\item $\G_0\ =\ \emptymap$
\item $\E_0\ =\ \longE{0}$
\end{itemize}
The members of $\T_0$ are type names, not type constructors; for convenience
we have used type-constructor identifiers
to stand also for the type names which are bound to them in the initial
static type environment $\TE_0$.  Of these type names,
~\LIST~ and ~\REF~
have arity 1, the rest have arity 0;  
all except \EXCN, \INSTREAM~ 
and ~\OUTSTREAM~ admit equality.

The components of $\E_0$ are as follows:
\begin{itemize}
\item $\SE_0\ =\ \emptymap$
\item $\VE_0$ is shown in Figures~\ref{stat-ve} and \ref{stat-veio}.
Note that
      $\Dom\VE_0$  contains those identifiers 
      ({\tt true}, {\tt false}, {\tt nil},
      \verb+::+, {\tt ref}) which are basic value constructors,
      for reasons discussed in Section~\ref{stat-proj}. 
      $\VE_0$ also includes $\EE_0$, for the same reasons.
\item $\TE_0$ is shown in Figure~\ref{stat-te}. Note that the type
      structures in $\TE_0$ contain the type schemes of all basic value
      constructors.
\item $\Dom\EE_0\ =\ \BasExc$~, the set of basic exception names listed in
Section~\ref{bas-exc}.
In each case the associated type is ~\EXCN~, except that
~$\EE_0({\tt Io})=\STRING\rightarrow\EXCN$.
\end{itemize}
}

%
\deletionPage{66}{
\begin{figure}
\begin{center}
\begin{tabular}{|rl|}
\hline
$\var$            & $\mapsto\ \tych$\\
\hline
{\tt std\_in}     & $\mapsto\ \INSTREAM$\\
{\tt open\_in}    & $\mapsto\ \STRING\to\INSTREAM$\\
{\tt input}       & $\mapsto\ \INSTREAM\ \ast\ \INT\to\STRING$\\
{\tt lookahead}   & $\mapsto\ \INSTREAM\to\STRING$\\
{\tt close\_in}   & $\mapsto\ \INSTREAM\to\UNIT$\\
{\tt end\_of\_stream}
                  & $\mapsto\ \INSTREAM\to\BOOL$\\
\multicolumn{2}{|c|}{}\\
{\tt std\_out}    & $\mapsto\ \OUTSTREAM$\\
{\tt open\_out}   & $\mapsto\ \STRING\to\OUTSTREAM$\\
{\tt output}      & $\mapsto\ \OUTSTREAM\ \ast\ \STRING\to\UNIT$\\
{\tt close\_out}  & $\mapsto\ \OUTSTREAM\to\UNIT$\\
\hline
\end{tabular}
\end{center}
\vspace{3pt}
\caption{Static $\VE_0$ (Input/Output)\index{75.1}}
\label{stat-veio}
\end{figure}}

%
\insertionPage{66}{
\begin{figure}[h]
\begin{center}
\begin{tabular}{|rll|}
\hline
$\tycon$   & $\mapsto\ (\ \typefcn$, & $\{\vid_1\mapsto(\tych_1,\is_1),\ldots,\vid_n\mapsto(\tych_n,\is_n)\}\ )\quad (n\geq0)$\\
\hline
\UNIT      & $\mapsto\ (\ \Lambda().\{ \}$,
                                      & $\emptymap\ )$ \\
\BOOL      & $\mapsto\ (\ \BOOL$,    & $\{\TRUE\mapsto(\BOOL,\isc),
                                         \ \FALSE\mapsto(\BOOL,\isc)\}\ )$\\
\INT       & $\mapsto\ (\ \INT$,     & $\{\}\ )$\\
\WORD       & $\mapsto\ (\ \WORD$,     & $\{\}\ )$\\
\REAL      & $\mapsto\ (\ \REAL$,    & $\{\}\ )$\\
\STRING    & $\mapsto\ (\ \STRING$,  & $\{\}\ )$\\
%\UNISTRING    & $\mapsto\ (\ \UNISTRING$,  & $\{\}\ )$\\
\CHAR    & $\mapsto\ (\ \CHAR$,  & $\{\}\ )$\\
%\UNICHAR    & $\mapsto\ (\ \UNICHAR$,  & $\{\}\ )$\\
\LIST      & $\mapsto\ (\ \LIST$,    & $\{\NIL\mapsto(\forall\atyvar\ .\ \atyvar\ \LIST, \isc)$,\\
           &                          & \ml{::}$\mapsto(\forall\atyvar\ .
                                           \ \atyvar\ast\atyvar\ \LIST
                                           \to\atyvar\ \LIST, \isc)\}\ )$\\
\REF       & $\mapsto\ (\ \REF$,     & $\{\REF\mapsto(\forall\ \atyvar\ .\ 
                                           \atyvar\to\atyvar\ \REF,\isc)\}\ )$\\
\EXCN      & $\mapsto\ (\ \EXCN$,     & $\emptymap\ )$\\
\hline
\end{tabular}
\end{center}
\vskip-3mm
\caption{Static $\TE_0$\index{75.2}}
\vskip-3mm
\label{stat-te}
\end{figure}}

%
\notePage{66}{Figure (Static $VE_0$): many entries removed (now in library/basis)}

%
\insertionPage{66}{Inserted figure with caption: Static $\VE_0$\index{74}}
%
\insertionPage{67}
The initial dynamic basis is $\B_0 = \F_0,\G_0,\E_0$, 
where $\F_0 = \emptymap$, $\G_0 = \emptymap$ and $\E_0 = (\SE_0, \TE_0,\VE_0)$,
where $\SE_0 = \emptymap$, $\TE_0$ is shown in Figure~\ref{dynTE0.fig}

%
\deletionPage{67}{
The initial dynamic basis is
\[ \B_0\ =\ \F_0,\G_0,\E_0\]
where
\begin{itemize}
\item $\F_0\ =\ \emptymap$
\item $\G_0\ =\ \emptymap$
\item $\E_0\ =\ \E_0'+\E_0''$
\end{itemize}
$\E_0'$ contains bindings of identifiers to the basic values BasVal and
basic exception names \BasExc; in fact
~$\E_0'\ =\ \SE_0',\VE_0',\EE_0'$~, where:
\begin{itemize}
\item $\SE_0'\ =\ \emptymap$
\item $\VE_0' = \{{\it id}\mapsto{\it id}\ ;\ {\it id}\in\hbox{BasVal}\}
       +\{\hbox{\boxml{:=}}\mapsto\hbox{\boxml{:=}}\} +\EE_0'$\\
       \vrule height0pt width3mm depth 0pt$+\;\{\boxml{true}\mapsto\boxml{true},\,\boxml{false}\mapsto\boxml{false},\,\boxml{nil}\mapsto\boxml{nil},\,
\hbox{\boxml{::}}\mapsto\hbox{\boxml{::}},\,
\hbox{\boxml{ref}}\mapsto\hbox{\boxml{ref}}\}$
\item $\EE_0'\ =\ \{\id\mapsto\id\ ;\ \id\in$ \BasExc$\}$
\end{itemize}
}

%
\deletionPage{67}{Note that $\VE_0'$ is the identity function on BasVal; this is because
we have chosen to denote these values by the names of variables 
to which they are initially bound.
The semantics of these basic values (most of which are functions)
lies principally in their behaviour under APPLY, which we describe below.
On the other hand the semantics of \boxml{:=} is provided by a special
semantic rule, rule~\ref{assapp-dyn-rule}.
Similarly, $\EE_0'$ is the identity function on \BasExc, the set of
basic exception names, because we have also chosen
these names to be just those exception constructors to which they
are initially bound.
These exceptions are raised by APPLY as described below.

 $\E_0''$ contains initial variable bindings which, unlike BasVal, are
 definable in ML; it is the result of evaluating
 the following declaration in the basis $\F_0,\G_0,\E_0'$.  For convenience,
 we have also included all basic infix directives in this declaration.

 [omitted]

}

%
\insertionPage{67}{
\begin{figure}[h]
\begin{center}
\begin{tabular}{|rll|}
\hline
$\tycon$   & $\mapsto$  & $\{\vid_1\mapsto\V_1,\ldots,\vid_n\mapsto\V_n\}\quad (n\geq0)$\\
\hline
\UNIT      & $\mapsto $ &  $\emptymap$ \\
\BOOL      & $\mapsto $ & $\{\TRUE\mapsto\TRUE
                                         \ \FALSE\mapsto\FALSE\}$\\
\INT       & $\mapsto $ & $\{\}$\\
\WORD      & $\mapsto $ & $\{\}$\\
\REAL      & $\mapsto $ & $\{\}$\\
\STRING    & $\mapsto $ & $\{\}$\\
%\UNISTRING & $\mapsto $ & $\{\}$\\
\CHAR      & $\mapsto $ & $\{\}$\\
%\UNICHAR   & $\mapsto $ & $\{\}$\\
\LIST      & $\mapsto $ & $\{\NIL\mapsto\NIL,\ml{::}\mapsto\ml{::}\}$\\
\REF       & $\mapsto $ & $\{\REF\mapsto\REF\}$\\
\EXCN      & $\mapsto $ & $\emptymap$\\
\hline
\end{tabular}
\end{center}
\caption{Dynamic $\TE_0$}
\label{dynTE0.fig}
\end{figure}}

%
\insertionPage{68}{
\section{Overloading}
\label{overload.sec}
Two forms of overloading are available:
\begin{itemize}
\item Certain special constants are overloaded.  For example,
{\tt 0w5} may have type $\WORD$ or some other type, depending on
the surrounding program text;
\item Certain operators are overloaded. For example,
{\tt +} may have type $\INT\ast\INT\to\INT$ or
$\REAL\ast\REAL\to\REAL$, depending on
the surrounding program text;
\end{itemize}
Programmers cannot define their own overloaded constants or operators.

Although a formal treatment of overloading is outside the scope
of this document, we do give a complete list of the overloaded operators
and of types with overloaded special constants.
This list is consistent with the Basis Library\cite{mllib96}.

Every overloaded constant and value identifier has among its types a 
{\em default type},
which is ascribed to it, when the surrounding text does not resolve the overloading.
For this purpose, the surrounding text is no larger than the smallest
enclosing structure-level declaration; an implementation may require
that a smaller context determines the type.

\subsection{Overloaded special constants}
Libraries may extend the set $\T_0$ of
Appendix~\ref{init-stat-bas-app} with additional type names. Thereafter, certain
subsets of $T_0$ have a special significance;
they are called {\sl overloading classes}
and they are:\medskip

\halign{\indent#\ \hfil&\ $#$\ &\ $#$\hfil\cr
\Int &\supseteq&\{\INT\}\cr
\Real &\supseteq&\{\REAL\}\cr
\Word &\supseteq&\{\WORD\}\cr
\String&\supseteq&\{\STRING\}\cr
\Char&\supseteq&\{\CHAR\}\cr
\WordInt&=&\Word\cup\INt\cr
\RealInt&=&\Real\cup\INt\cr
\Num&=&\Word\cup\Real\cup\INt\cr
\NumTxt&=&\Word\cup\Real\cup\INt\cup\String\cup\Char\cr}
\medskip

\noindent 
Among these, the five first ($\Int$, $\Real$, $\Word$, $\String$ and $\Char$) are said to be
{\sl basic}; the remaining are said to be {\sl composite}.
The reason that the basic classes are specified using
$\supseteq$ rather than $=$ is that libraries may extend 
each of the  basic overloading
classes with further type names.  Special constants are overloaded
within each of the basic overloading classes.  However, the basic
overloading classes must be arranged so that every special constant can be
ascribed types from at most one of the basic overloading classes.  For
example, to \boxml{0w5} may be ascribed type $\WORD$, or
some other member of $\Word$, depending on the surrounding text.  If
the surrounding text does not determine the type of the constant, a
default type is used. The default types for the five sets are $\INT$,
$\REAL$, $\WORD$, $\STRING$ and $\CHAR$ respectively.

       Once overloading resolution has determined the type of a special constant,
       it is a compile-time error if the constant does not make sense or does not 
       denote a value within the machine representation chosen for the type.
       For example, an escape sequence of the form $\uconst$ in a string constant
       of 8-bit characters only makes sense if $xxxx$  denotes
       a number in the range $[0, 255]$. 


\subsection{Overloaded value identifiers}
Overloaded identifiers all have identifier status $\isv$. An
overloaded identifier may be re-bound with any status ($\isv$, $\isc$
and $\ise$) but then it is not overloaded within the scope of
the binding.

\begin{figure}
\begin{center}
\vskip-12pt
\begin{tabular}{|rl|rl|}
\multicolumn{2}{c}{NONFIX}&     \multicolumn{2}{c}{INFIX}\\
\hline
$\var$     & $\mapsto\ \hbox{set of monotypes}$    
                          & $\var$ & $\mapsto\ \hbox{set of monotypes}$\\
\hline
\boxml{abs} & $\mapsto \REALINT\to\REALINT$ 
                       & \multicolumn{2}{l|}{Precedence 7, left associative :} \\
\NEG    & $\mapsto \REALINT\to\REALINT $                      &
                            \boxml{div} & $\mapsto \WORDINT\ \ast\ \WORDINT
                                                                 \to\WORDINT$\\
 &  
                                             &
                            \boxml{mod} & $\mapsto \WORDINT\ \ast\ \WORDINT
                                                                 \to\WORDINT$\\
  &                       &
                            \boxml{*} &$\mapsto \NUM\ \ast\ \NUM
                                                                 \to\NUM$\\
  &                       &
                            \boxml{/} &$\mapsto \RREAL\ \ast\ \RREAL
                                                                 \to\RREAL$\\
  & &
                            \multicolumn{2}{l|}{Precedence 6, left associative :} \\
  &                       &
                            \boxml{+} &$\mapsto \NUM\ \ast\ \NUM
                                                                 \to\NUM$\\
  &                       &
                            \boxml{-} &$\mapsto \NUM\ \ast\ \NUM\to\NUM$\\
  & 
                          & \multicolumn{2}{l|}{Precedence 4, left associative :}\\
              &           &
                            \boxml{<} & $\mapsto\NUMTEXT *\NUMTEXT \to \NUMTEXT$\\
              &           &
                            \boxml{>} & $\mapsto\NUMTEXT *\NUMTEXT \to \NUMTEXT$\\
              &           &
                            \boxml{<=} & $\mapsto\NUMTEXT *\NUMTEXT \to \NUMTEXT$\\
              &           &
                            \boxml{>=} & $\mapsto\NUMTEXT *\NUMTEXT \to \NUMTEXT$\\
\hline
\end{tabular}
\end{center}
\vskip-15pt
\caption{Overloaded identifiers}
\label{overload.fig}
\end{figure}
}

%
\notePage{70}{This appendix has been revised and extended in 
several ways. A detailed list of changes is not available.}

%
\notePage{77}{This appendix is new!}

%
\notePage{94}{The Index has been revised; it is not accurate for
the version that has marginal notes.}
