\section{Dynamic Semantics for the Core}
\subsection{Reduced Syntax}
Since\index{47.1} types are fully dealt with in the static semantics,
the dynamic semantics ignores them.  The Core syntax is therefore
reduced by the following transformations, for the purpose of the dynamic
semantics:
\begin{itemize}
\item All explicit type ascriptions ``\ml{:} $\ty$'' are omitted, and
      qualifications ``$\OF\  \ty$'' are omitted from exception
      bindings.
\item Any declaration of the form ``$\typedec$'' or ``$\datatypedec$''
      is replaced by the empty declaration.
\item A declaration of the form ``$\abstypedec$'' is replaced by ``$\dec$''.
\item The Core phrase classes TypBind, DatBind, $\ConBind$, Ty and
      TyRow are omitted.
%version 2:\item The Core phrase classes $\typbind$, $\datbind$, $\constrs$, $\ty$ and
%      $\labtys$ are omitted.
\end{itemize}

\subsection{Simple Objects}
All\index{47.2} objects in the dynamic semantics are built from
identifier classes together with the simple object classes shown (with the
variables which range over them) in Figure~\ref{simp-dyn-obj}.

\begin{figure}[h]
\vspace{2pt}
\begin{displaymath}
\begin{array}{rclr}
\A               & \in   & \Addr	& \mbox{addresses}\\
\e               & \in   & \Exc 	& \mbox{exception names}\\
b      		& \in	& \BasVal	& \mbox{basic values}\\
\sv             & \in   & \SVal         & \mbox{special values}\\
                &       & \{\FAIL\}     & \mbox{failure}\\   
\end{array}
\end{displaymath}
\caption{Simple Semantic Objects}
\label{simp-dyn-obj}
\vspace{3pt}
\end{figure}

$\Addr$ and $\Exc$ are infinite sets. BasVal is described below.
{\SVal} is the class of values denoted by the special constants
\SCon. Each integer or real constant denotes a value according to normal 
mathematical conventions; each string constant denotes a sequence of ASCII
characters as explained in Section~\ref{cr:speccon}. The value denoted
by {\scon} is written $\sconval(\scon)$.
$\FAIL$ is the result of a failing attempt to match a value and a
pattern. Thus $\FAIL$ is neither a value nor an exception, but simply
a semantic object used in the rules to express operationally
how matching proceeds.

Exception constructors evaluate to exception names, unlike value constructors
which simply evaluate to themselves. This is to accommodate the generative
nature of exception bindings;\index{48.0} each evaluation of a declaration of a
exception constructor binds it to a new unique name.

\subsection{Compound Objects}

The\index{48.2} compound objects for the dynamic semantics are
shown in Figure~\ref{comp-dyn-obj}.
Many conventions and notations are adopted as in the static semantics; in
particular projection, injection and modification all retain
their meaning.  We generally omit the injection functions taking $\Con$,
$\Con\times\Val$ etc into $\Val$. For records $\r\in\Record$ however,
we write this injection explicitly as ``$\In\ \Val$''; this accords with
the fact that there is a separate phrase class ExpRow, whose members
evaluate to records. 

We take $\cup$ to mean disjoint union over
semantic object classes. We also understand all the defined object
classes to be disjoint. A particular case deserves mention; $\ExVal$
and $\Pack$ (exception values and packets) are isomorphic classes,
but the latter class corresponds to exceptions which have been
raised, and therefore has different semantic significance from the
former, which is just a subclass of values.


\begin{figure}[t]
\vspace{2pt}
\begin{displaymath}
\begin{array}{rcl}
        \V	&\in	&\Val =\{\mbox{\tt :=}\}\cup\SVal\cup\BasVal\cup\Con\\
                &       &\qquad\cup(\Con\times\Val)\cup\ExVal\\
                &       &\qquad\cup\Record\cup\Addr\cup\Closure\\
        \r      & \in   & \Record =  \finfun{\Lab}{\Val}\\
{\exval}      & \in   & \ExVal = \Exc \cup (\Exc\times\Val)\\
{[\exval]}\ {\rm or}\ \p
                & \in   & \Pack = \ExVal\\
(\match,\E,\VE) & \in   & \Closure = \Match\times\Env\times\VarEnv\\
        \mem    & \in   & \Mem = \finfun{\Addr}{\Val}\\
        \excs   & \in   & \ExcSet = \Fin(\Exc)\\
(\mem,\excs)\ {\rm or}\ \s
                & \in   & \State = \Mem\times\ExcSet\\
(\SE,\VE,\EE)\ {\rm or}\ \E
                & \in   & \Env = \StrEnv\times\VarEnv\times\ExnEnv\\
        \SE     & \in   & \StrEnv = \finfun{\StrId}{\Env}\\
        \VE	& \in	& \VarEnv = \finfun{\Var}{\Val}\\
        \EE	& \in	& \ExnEnv = \finfun{\Exn}{\Exc}\\
\end{array}
\end{displaymath}
\caption{Compound Semantic Objects\index{48.1}}
\label{comp-dyn-obj}
\vspace{3pt}
\end{figure}
%
%
Although the same names, e.g. $\E$ for an environment, are used
as in the static semantics, the objects denoted are different.  This need cause
no confusion since the static and dynamic semantics are presented %completely
separately. An important point is that structure names $\m$ have
no significance at all in the dynamic semantics; this explains why the
object class $\Str = \StrNames\times\Env$ is absent here -- for the dynamic
semantics the concepts {\sl structure} and {\sl environment} coincide.

\subsection{Basic Values}
The\index{49.1} basic values in $\BasVal$ are the values bound to predefined variables.
These values are denoted by the identifiers to which they are bound in the
initial dynamic basis (see Appendix~\ref{init-dyn-bas-app}), 
and are as follows:
\begin{verbatim}
           abs  floor  real  sqrt  sin  cos  arctan  exp  ln
              size  chr  ord  explode  implode  div  mod
                  ~  /  *  +  -  =  <>  <  >  <=  >=
        std_in  std_out  open_in  open_out  close_in  close_out
                input  output  lookahead  end_of_stream
\end{verbatim}
The meaning of basic values (almost all of which are functions) is
represented by the function
\[ \APPLY\ :\ \BasVal\times\Val\to\Val\cup\Pack \]
 which is detailed in Appendix~\ref{init-dyn-bas-app}.

\subsection{Basic Exceptions}
\label{bas-exc}
A\index{49.2} subset $\BasExc\subset\Exc$ of the exception names are bound to predefined
exception constructors.
These names are denoted by the identifiers to which they are bound in the
initial dynamic basis (see Appendix~\ref{init-dyn-bas-app}), 
and are as follows:
\begin{verbatim}
        Abs  Ord  Chr   Div  Mod  Quot  Prod  
        Neg  Sum  Diff  Floor  Sqrt  Exp  Ln
        Io   Match  Bind  Interrupt
\end{verbatim}
The exceptions on the first two  lines are raised by 
corresponding basic functions, where \verb+~+ {\tt /} {\tt *}
{\tt +} {\tt -} correspond respectively to {\tt Neg} {\tt Quot}
{\tt Prod} {\tt Sum} {\tt Diff}. The details are given
in Appendix~\ref{init-dyn-bas-app}. The exception $(\mbox{{\tt Io}},s)$,
where $s$ is a string, is raised
by certain of the basic input/output functions,
as detailed in Appendix~\ref{init-dyn-bas-app}.  
The exceptions ~\ml{Match}~ and
~\ml{Bind}~
are raised upon failure of pattern-matching in evaluating a 
function {\fnexp} or a
$\valbind$, as detailed in the rules to follow.   Finally, ~\ml{Interrupt}~
is raised by external intervention.

Recall from Section~\ref{further-restrictions-sec} 
that in the context {\fnexp}, the $\match$ 
must be irredundant and exhaustive and that the compiler should flag
the {\match} if it violates these restrictions. The exception~\ml{Match}
can only be raised for a match which is not exhaustive, and has therefore 
been flagged by the compiler.

%In a match of the form $\pat_1\ \ml{=>}\ \exp_1\ \ml{|}\ \ldots\ \ml{|}\ 
%            \pat_n\ \ml{=>}\ \exp_n$
%the pattern sequence $\pat_1,\ldots,\pat_n$ should be {\sl irredundant};
%that is, each $\pat_j$ must match some value
%(of the right type) which is not matched by $\pat_i$ for any $i<j$.
%In the context {\fnexp}, the $\match$ must also be {\sl exhaustive}; that is,
%every value (of the right type) must be matched by some $\pat_i$.
%The compiler must give warning on violation of these restrictions, 
%but should still compile the match. 
%The \ml{match} exception
%can only be raised for a match which is not exhaustive, and has therefore
%been flagged by the compiler.
%The restrictions are inherited by derived forms; in particular,
%this means that in the function binding 
% $\var\ \atpat_1\ \cdots\ \atpat_n\langle : \ty\rangle\ \ml{=}\ \exp$
%(consisting of one clause only), each separate $\atpat_i$ should be
%exhaustive by itself.

For each value binding $\pat\ \mbox{\ml{=}}\ \exp$ the compiler must issue a 
report (but still compile) if {\it either} pat is not exhaustive {\it or}
pat contains no variable. This will (on both counts) detect a mistaken
declaration like $\VAL\ \ml{nil}\ \mbox{\ml{=}}\ \exp$ in which the user
expects to declare a new variable \ml{nil} (whereas the language dictates
that \ml{nil} is here a constant pattern, so no variable gets declared).
However, these warnings should not be given when the binding is a component
of a top-level declaration $\valdec$; e.g. 
 $\VAL\ \mbox{\ml{x::l = }}\exp_1\ \mbox{\ml{\AND\ y = }}\exp_2$ 
is not faulted by the compiler at top level, but may of course generate
a \ml{Bind} exception.

\subsection{Closures}
The\index{50.1} informal understanding of a {\sl closure} $(\match,\E,\VE)$ is as follows:
when the closure is applied to a value $\V$,
$\match$ will be evaluated against $\V$, in the environment $\E$ modified in
a special sense by
$\VE$.  The domain $\Dom\VE$ of this third component contains those function
identifiers to be treated recursively in the evaluation.  To achieve this
effect, the evaluation of $\match$ will take place not in $\plusmap{\E}{\VE}$
but in $\plusmap{\E}{\Rec\VE}$, where
\[ \Rec\ :\ \VarEnv\to\VarEnv \]
is defined as follows:
\begin{itemize}
\item $\Dom(\Rec\VE)=\Dom\VE $
\item If $\VE(\var)\notin\Closure$, then $(\Rec\VE)(\var)=\VE(\var)$
\item If $\VE(\var)=(\match',\E',\VE')$
      then $(\Rec\VE)(\var)=(\match',\E',\VE)$
\end{itemize}
The effect is that, before application of $(\match,\E,\VE)$ to $\V$, the
closure values in $\Ran\VE$ are ``unrolled'' once, to prepare for their possible
recursive application during the evaluation of $\match$ upon $\V$.

This device is adopted to ensure that all semantic objects are finite (by
controlling the unrolling of recursion).  The operator $\Rec$ is invoked in
just two places in the semantic rules: in the rule for recursive value
bindings of the form ``$\REC\ \valbind$'', and in the rule for evaluating
an application expression ``$\exp\ atexp$'' in the case that $\exp$
evaluates to a closure.

\subsection{Inference Rules}
\label{dyncor-inf-rules-sec}
The\index{50.2} semantic rules allow sentences  of the form
\[ \s,A\ts\phrase\ra A',\s' \]
to be inferred, where $A$ is usually an environment, $A'$ is some semantic
object and $\s$,$\s'$ are the states before and after the evaluation
represented by the sentence.  Some hypotheses in rules are not of this form;
they are called {\sl side-conditions}.  The convention for options is
the same as for the Core static semantics.  

In most rules the states $\s$ and $\s'$ are omitted from sentences; they
are only included for those rules which are directly concerned with the state
-- either referring to its contents or changing it.  When omitted, the
convention for restoring them is as follows.  If the rule is presented in the
form
\[ \frac{ \begin{array}{r}
          A_1\ts\phrase_1\ra A_1'\qquad
          A_2\ts\phrase_2\ra A_2'\quad\cdots\\
          \cdots\quad A_n\ts\phrase_n\ra A_n'
          \end{array}
        }
        { A\ts\phrase\ra A'} \]
then the full form is intended to be
\[ \frac{ \begin{array}{r}
          \s_0,A_1\ts\phrase_1\ra A_1',\s_1\qquad
          \s_1,A_2\ts\phrase_2\ra A_2',\s_2\quad\cdots\\
          \cdots\quad\s_{n-1},A_n\ts\phrase_n\ra A_n',\s_n
          \end{array}
        }
        { \s_0,A\ts\phrase\ra A',\s_n} \]
(Any side-conditions are left unaltered).
Thus the left-to-right order of the hypotheses indicates the order of
evaluation.  Note that in the case $\n=0$, when there are no hypotheses
(except possibly side-conditions), we have $\s_n=\s_0$; this implies that the
rule causes no side effect.
The convention is called the {\sl state convention},\index{51.1} and
must be applied to each version of a rule obtained by inclusion or
omission of its options.

A second convention, the {\sl exception convention}, is adopted to deal
with the propagation of exception packets $\p$.
For each rule whose full form (ignoring side-conditions) is
\[ \frac{ \s_1,A_1\ts\phrase_1\ra A_1',\s_1'\qquad\cdots\qquad
          \s_n,A_n\ts\phrase_n\ra A_n',\s_n' }
        { \s,A\ts\phrase\ra A',\s'} \]
and for each $k$, $1\leq k\leq n$, for which the result $A_k'$ is not a
packet $\p$, an extra rule is added of the form
\[ \frac{ \s_1,A_1\ts\phrase_1\ra A_1',\s_1'\qquad\cdots\qquad
          \s_k,A_k\ts\phrase_k\ra \p',\s' }
        { \s,A\ts\phrase\ra \p',\s'} \]
where $\p'$ does not occur in the original rule.\footnote{There is one
exception to the exception convention; no extra rule is added for
rule~\ref{handlexp-dyn-rule1} which deals with handlers, 
since a handler is the only
means by which propagation of an exception can be arrested.}
This indicates that evaluation of phrases in the hypothesis terminates with the
first whose result is a packet (other than one already treated in the rule),
and this packet is the result of the phrase in the conclusion.

A third convention is that we allow compound variables (variables built
from the variables in Figure~\ref{comp-dyn-obj} and the symbol ``/'')
to range over unions of semantic objects. For instance 
the compound variable $\V/\p$ ranges
over $\Val\cup\Pack$. 
We also allow $x/\FAIL$ to range over $X\cup\{\FAIL\}$ where $x$ 
ranges over $X$;
furthermore, we extend environment modification to allow for failure
as follows:
\[\VE+\FAIL=\FAIL.\]
%
%                       Atomic Expressions
%
\rulesec{Atomic Expressions}{\E\ts\atexp\ra\V/\p}
\begin{equation}	% special constant
%\label{sconexp-dyn-rule}
\frac{}
     {\E\ts\scon\ra\sconval(\scon)}\index{51.15}
\end{equation}

\begin{equation}	% value variable
%\label{varexp-dyn-rule}
\frac{\E(\longvar)=\V}
     {\E\ts\longvar\ra\V}\index{51.2}
\end{equation}

\begin{equation}	% value constructor
\label{conexp-dyn-rule}
\frac{\longcon=\strid_1.\cdots.\strid_k.\con}
     {\E\ts\longcon\ra\con}\index{52.1}
\end{equation}

\begin{equation}       %  exception constant
\label{exconexp-dyn-rule}
\frac{\E(\longexn)=\e}
     {\E\ts\longexn\ra\e}
\end{equation}


\begin{equation}	% record expression
%\label{recexp-dyn-rule}
\frac{\langle\E\ts\labexps\ra\r\rangle}
     {\E\ts\lttbrace\ \recexp\ \rttbrace\ra\emptymap
                                  \langle +\ \r\rangle\ \In\ \Val}
\end{equation}

\begin{equation}        % local declaration
%\label{let-dyn-rule}
\frac{\E\ts\dec\ra\E'\qquad\E+\E'\ts\exp\ra\V}
     {\E\ts\letexp\ra\V}
\end{equation}

\begin{equation}	% paren expression
%\label{parexp-dyn-rule}
\frac{\E\ts\exp\ra\V}
     {\E\ts\parexp\ra\V}
\end{equation}
\comments
\begin{description}
\item{(\ref{conexp-dyn-rule})}
   Value constructors denote themselves.
\item{(\ref{exconexp-dyn-rule})}
   Exception constructors are looked up in the exception environment
   component of $\E$.
\end{description}

\rulesec{Expression Rows}{\E\ts\labexps\ra\r/\p}
\begin{equation}	% labelled expressions
%\label{labexps-dyn-rule}
\frac{\E\ts\exp\ra\V\qquad\langle\E\ts\labexps\ra\r\rangle}
     {\E\ts\longlabexps\ra\{\lab\mapsto\V\}\langle +\ \r\rangle}\index{52.2}
\end{equation}
\comment We may think of components as being evaluated from left to right,
because of the state and exception conventions.
%
%                        Expressions
%
\rulesec{Expressions}{\E\ts\exp\ra\V/\p}
\begin{equation}	% atomic
%\label{atexp-dyn-rule}
\frac{\E\ts\atexp\ra\V}
     {\E\ts\atexp\ra\V}\index{52.3}
\end{equation}

\begin{equation}	% constructor application
\label{conapp-dyn-rule}
\frac{\E\ts\exp\ra\con\qquad\con\neq\REF\qquad\E\ts\atexp\ra\V}
     {\E\ts\appexp\ra(\con,\V)}
\end{equation}

\begin{equation}        % exception constructor application
\frac{\E\ts\exp\ra\e\qquad\E\ts\atexp\ra\V}
     {\E\ts\appexp\ra(\e,\V)}
\end{equation}

\begin{equation}	% reference application
\label{refapp-dyn-rule}
\frac{\s,\E\ts\exp\ra~\ml{ref}~,\s'\qquad
      \s',\E\ts\atexp\ra\V,\s''\qquad
      \A\notin\Dom(\of{\mem}{\s''})}
     {\s,\E\ts\appexp\ra\A,\ \s''+\{\A\mapsto\V\} }
\end{equation}

%%%%%%%%%%%%%%%%%%%%%%%%%%%%%%%%%%%%%%%%%%%%%%%%%%%%%%%%%%%%%%%%%%%%%%
%\begin{equation}	% contents application
%\label{contapp-dyn-rule}
%\frac{\s,\E\ts\exp\ra~\mbox{\tt !}~,\s'\qquad\s',\E\ts\atexp\ra\A,\s''
%      \qquad\s''(\A)=\V}
%     {\s,\E\ts\appexp\ra\V,\s''}
%\end{equation}
%%%%%%%%%%%%%%%%%%%%%%%%%%%%%%%%%%%%%%%%%%%%%%%%%%%%%%%%%%%%%%%%%%%%%%

\begin{equation}	% assignment application
\label{assapp-dyn-rule}
\frac{\s,\E\ts\exp\ra~\mbox{\tt :=}~,\s'\qquad
      \s',\E\ts\atexp\ra\{{\tt 1}\mapsto\A,\ {\tt 2}\mapsto\V\},\s''}
     {\s,\E\ts\appexp\ra\emptymap\ \In\ \Val,\ \s''+\{\A\mapsto\V\} }
\end{equation}

\begin{equation}	% basic function application
%\label{basapp-dyn-rule}
\frac{\E\ts\exp\ra b
      \qquad\E\ts\atexp\ra\V\qquad\APPLY(b,\V)=\V'}
     {\E\ts\appexp\ra\V'}\index{53.1}
\end{equation}

\begin{equation}	% closure application
\label{closapp-dyn-rule}
\frac{\begin{array}{c}
      \E\ts\exp\ra(\match,\E',\VE)\qquad\E\ts\atexp\ra\V\\
      \E'+\Rec\VE,\ \V\ts\match\ra\V'
      \end{array}
     }
     {\E\ts\appexp\ra\V'}
\end{equation}

\begin{equation}        % failing closure application
\label{closapp-dyn-rule1}
\frac{\begin{array}{c}
      \E\ts\exp\ra(\match,\E',\VE)\qquad\E\ts\atexp\ra\V\\
      \E'+\Rec\VE,\ \V\ts\match\ra\FAIL
      \end{array}
     }
     {\E\ts\appexp\ra[{\tt Match}]}
\end{equation}



\begin{equation}        % handle exception 1
\label{handlexp-dyn-rule1}
\frac{\E\ts\exp\ra\V}
     {\E\ts\handlexp\ra\V}
\end{equation}

\begin{equation}        % handle exception 2
\label{handlexp-dyn-rule2}
\frac{\E\ts\exp\ra[\exval]\qquad\E,\exval\ts\match\ra\V}
     {\E\ts\handlexp\ra\V}
\end{equation}

\begin{equation}        % handle exception 3
\label{handlexp-dyn-rule3}
\frac{\E\ts\exp\ra[\exval]\qquad\E,\exval\ts\match\ra\FAIL}
     {\E\ts\handlexp\ra[\exval]}
\end{equation}

\begin{equation}        % raise exception
%\label{raisexp-dyn-rule}
\frac{\E\ts\exp\ra\exval}
     {\E\ts\raisexp\ra[\exval]}
\end{equation}

\begin{equation}        % function
\label{fnexp-dyn-rule}
\frac{}
     {\E\ts\fnexp\ra(\match,\E,\emptymap)}
\end{equation}
\comments
\begin{description}
\item{(\ref{refapp-dyn-rule})}
  The side condition ensures that a new address is chosen. There are
no rules concerning disposal of inaccessible addresses (``garbage
collection'').
%
\item{(\ref{conapp-dyn-rule})--(\ref{closapp-dyn-rule1})}
  Note that none of the rules for function application has a
premise in which the operator evaluates to a constructed
value, a record or an address. This is because we are interested
in the evaluation of well-typed programs only, and in such programs $\exp$
will always have a functional type.
% so $\V$ will be either a closure,
%a constructor, a basic value or \ml{:=}.
%
\item{(\ref{handlexp-dyn-rule1})}
  This is the only rule to which the exception convention does not apply.
If the operator evaluates to a packet then rule~\ref{handlexp-dyn-rule2}
or rule~\ref{handlexp-dyn-rule3} must be used.
%
\item{(\ref{handlexp-dyn-rule3})}
 Packets that are not handled by the $\match$ propagate.
%
\item{(\ref{fnexp-dyn-rule})}
  The third component of the closure is empty because the match does not
introduce new recursively defined values.
\end{description}
%
%                        Matches
%
\rulesec{Matches}{\E,\V\ts\match\ra\V'/\p/\FAIL}
\begin{equation}	% match 1
%\label{match-dyn-rule1}
\frac{\E,\V\ts\mrule\ra\V'}
     {\E,\V\ts\longmatch\ra\V'}\index{54.1}
\end{equation}

\begin{equation}	% match 2
%\label{match-dyn-rule2}
\frac{\E,\V\ts\mrule\ra\FAIL}
     {\E,\V\ts\mrule\ra\FAIL}
\end{equation}

\begin{equation}	% match 3
%\label{match-dyn-rule}
\frac{\E,\V\ts\mrule\ra\FAIL\qquad\E,\V\ts\match\ra\V'/\FAIL}
     {\E\ts\longmatcha\ra\V'/\FAIL}
\end{equation}
\comment A value $\V$ occurs on the left of the turnstile, in evaluating
a $\match$. We may think of a $\match$ as being evaluated {\sl against}
a value; similarly, we may think of a pattern as being evaluated {\sl
against} a value.
Alternative match rules are tried from left to right.

\rulesec{Match Rules}{\E,\V\ts\mrule\ra\V'/\p/\fail}
\begin{equation}	% mrule 1
%\label{mrule-dyn-rule1}
\frac{\E,\V\ts\pat\ra\VE\qquad\E+\VE\ts\exp\ra\V'}
     {\E,\V\ts\longmrule\ \ra\V'}\index{54.2}
\end{equation}

\begin{equation}	% mrule 2
%\label{mrule-dyn-rule2}
\frac{\E,\V\ts\pat\ra\FAIL}
     {\E,\V\ts\longmrule\ \ra\FAIL}
\end{equation}

%
%                        Declarations
%
\rulesec{Declarations}{\E\ts\dec\ra\E'/\p}
\begin{equation}	% value declaration
%\label{valdec-dyn-rule}
\frac{\E\ts\valbind\ra\VE}
     {\E\ts\valdec\ra\VE\ \In\ \Env}\index{54.3}
\end{equation}

\begin{equation}	% exception declaration
%\label{exceptiondec-dyn-rule}
\frac{\E\ts\exnbind\ra\EE }
     {\E\ts\exceptiondec\ra\EE\ \In\ \Env }
\end{equation}

\begin{equation}	% local declaration
%\label{localdec-dyn-rule}
\frac{\E\ts\dec_1\ra\E_1\qquad\E+\E_1\ts\dec_2\ra\E_2}
     {\E\ts\localdec\ra\E_2}
\end{equation}

\begin{equation}                % open declaration
%\label{open-strdec-dyn-rule}
\frac{ \E(\longstrid_1)=\E_1
            \quad\cdots\quad
       \E(\longstrid_k)=\E_k }
     { \E\ts\openstrdec\ra \E_1 + \cdots + \E_k }
\end{equation}

\vspace{6pt}
\begin{equation}	% empty declaration
%\label{emptydec-dyn-rule}
\frac{}
     {\E\ts\emptydec\ra\emptymap\ \In\ \Env}
\end{equation}

\begin{equation}	% sequential declaration
%\label{seqdec-dyn-rule}
\frac{\E\ts\dec_1\ra\E_1\qquad\E+\E_1\ts\dec_2\ra\E_2}
     {\E\ts\seqdec\ra\plusmap{E_1}{E_2}}
\end{equation}
%
%                        Bindings
%
\rulesec{Value Bindings}{\E\ts\valbind\ra\VE/\p}
\begin{equation}	% value binding 1
%\label{valbind-dyn-rule1}
\frac{\E\ts\exp\ra\V\qquad\E,\V\ts\pat\ra\VE\qquad
      \langle\E\ts\valbind\ra\VE'\rangle }
     {\E\ts\longvalbind\ra\VE\ \langle +\ \VE'\rangle}\index{55.1}
\end{equation}

\begin{equation}	% value binding 2
%\label{valbind-dyn-rule2}
\frac{\E\ts\exp\ra\V\qquad\E,\V\ts\pat\ra\FAIL}
     {\E\ts\longvalbind\ra[\ml{Bind}]}
\end{equation}

\begin{equation}	% recursive value binding
%\label{recvalbind-dyn-rule}
\frac{\E\ts\valbind\ra\VE}
     {\E\ts\recvalbind\ra\Rec\VE}
\end{equation}

\rulesec{Exception Bindings}{\E\ts\exnbind\ra\EE/\p}
\begin{equation}	% exception binding 1
\label{exnbind-dyn-rule1}
\frac{\e\notin\of{\excs}{\s}\qquad\s'=\s+\{\e\}\qquad
      \langle\s',\E\ts\exnbind\ra\EE,\s''\rangle }
     {\s,\E\ts\longexnbindaa\ra\{\exn\mapsto\e\}\langle +\ \EE\rangle,\
                               \s'\langle'\rangle}\index{55.2}
\end{equation}

\begin{equation}	% exception binding 2
%\label{exnbind-dyn-rule2}
\frac{\E(\longexn)=\e\qquad
      \langle\E\ts\exnbind\ra\EE\rangle }
     {\E\ts\longexnbindb\ra\{\exn\mapsto\e\}\langle +\ \EE\rangle}
\end{equation}
\comments
\begin{description}
\item{(\ref{exnbind-dyn-rule1})}
  The two side conditions ensure that a new exception name is generated and 
recorded as ``used'' in subsequent states.
\end{description}
%
%                        Atomic Patterns
%
\rulesec{Atomic Patterns}{\E,\V\ts\atpat\ra\VE/\fail}
\begin{equation}	% wildcard pattern
%\label{wildcard-dyn-rule}
\frac{}
     {\E,\V\ts\wildpat\ra \emptymap}\index{55.3}
\end{equation}

\begin{equation}	% special constant in pattern (1)
%\label{sconpat-dyn-rule1}
\frac{\V=\sconval(\scon)}
     {\E,\V\ts\scon\ra \emptymap}\index{55.35}
\end{equation}

\begin{equation}	% special constant in pattern (2)
\label{sconpat-dyn-rule2}
\frac{\V\neq\sconval(\scon)}
     {\E,\V\ts\scon\ra \FAIL}\index{55.36}
\end{equation}

\begin{equation}	% variable pattern
%\label{varpat-dyn-rule}
\frac{}
     {\E,\V\ts\var\ra \{\var\mapsto\V\} }
\end{equation}

\begin{equation}	% constant pattern
%\label{conapat-dyn-rule1}
\frac{\longcon=\strid_1.\cdots.\strid_k.\con\qquad\V=\con }
     {\E,\V\ts\longcon\ra \emptymap}
\end{equation}

\begin{equation}
\label{conapat-dyn-rule2}
\frac{\longcon=\strid_1.\cdots.\strid_k.\con\qquad\V\neq\con}
     {\E,\V\ts\longcon\ra\FAIL}
\end{equation}
%\begin{equation}	% constant pattern
%\label{conpat-dyn-rule}
%\frac{\longcon=\strid_1.\cdots.\strid_k.\con }
%     {\con\ts\longcon\ra \emptymap}
%\end{equation}

\begin{equation}        % exception constant
%\label{exconapat-dyn-rule1}
\frac{\E(\longexn)=\V}
     {\E,\V\ts\longexn\ra\emptymap}
\end{equation}

\begin{equation}	
\label{exconapat-dyn-rule2}
\frac{\E(\longexn)\neq\V}
     {\E,\V\ts\longexn\ra\FAIL}\index{56.0}
\end{equation}


\begin{equation}	% record pattern
%\label{recpat-dyn-rule}
\frac{\V=\emptymap\langle+\r\rangle\ \In\ \Val\qquad
      \langle\E,\r\ts\labpats\ra\VE/\fail\rangle}
     {\E,\V\ts\lttbrace\ \langle\labpats\rangle\ \rttbrace\ra\emptymap\langle+\VE/\fail\rangle}
\end{equation}
%\begin{equation}	% record pattern
%\label{recpat-dyn-rule}
%\frac{\langle\r\ts\labpats\ra\VE\rangle}
%     {\emptymap\langle +\ \r\rangle\ \In\ \Val
%      \ts\{\ \recpat\ \}\ra\emptymap\langle +\ \VE\rangle}
%\end{equation}

\begin{equation}	% parenthesised pattern
%\label{parpat-dyn-rule}
\frac{\E,\V\ts\pat\ra\VE/\fail}
     {\E,\V\ts\parpat\ra\VE/\fail}\index{56.1}
\end{equation}

%\begin{equation}	% failure of atomic pattern
%\label{failatpat-dyn-rule}
%\frac{\forall\VE\ (\V\ts\atpat\not\Rightarrow\VE)}
%     {\V\ts\atpat\ra\FAIL}
%\end{equation}
\comments
\begin{description}
\item{(\ref{sconpat-dyn-rule2}),(\ref{conapat-dyn-rule2}),(\ref{exconapat-dyn-rule2})}
  Any evaluation resulting in $\FAIL$ must do so because 
rule~\ref{sconpat-dyn-rule2},
rule~\ref{conapat-dyn-rule2},
rule~\ref{exconapat-dyn-rule2},
rule~\ref{conpat-dyn-rule2},
or rule~\ref{exconpat-dyn-rule2} has been
applied.
\end{description}

\rulesec{Pattern Rows}{\E,\r\ts\labpats\ra\VE/\fail}
\begin{equation}	% wildcard record
%\label{wildrec-dyn-rule}
\frac{}
     {\E,\r\ts\wildrec\ra\emptymap}\index{56.2}
\end{equation}
 

\begin{equation}	% record component with inherited FAIL
\label{longlab-dyn-rule1}
\frac{\E,\r(\lab)\ts\pat\ra\FAIL}
     {\E,\r\ts\longlabpats\ra\FAIL}
\end{equation}

\begin{equation}	% record component
\label{longlab-dyn-rule2}
\frac{\E,\r(\lab)\ts\pat\ra\VE\qquad
      \langle\E,\r\ts\labpats\ra\VE'/\fail\rangle }
     {\E,\r\ts\longlabpats\ra
      \VE\langle +\ \VE'/\fail\rangle}
\end{equation}
\comments
\begin{description}
\item{(\ref{longlab-dyn-rule1}),(\ref{longlab-dyn-rule2})}
For well-typed programs $\lab$ will be in the domain of $\r$.
\end{description}
%\begin{equation}	% record component
%\label{longlab-dyn-rule}
%\frac{\V\ts\pat\ra\VE\qquad
%      \langle\r\ts\labpats\ra\VE'\qquad\VE\sim\VE'\rangle }
%     {\{\lab\mapsto\V\}\langle +\r\rangle\ts\longlabpats\ra
%      \VE\langle +\ \VE'\rangle}
%\end{equation}

%\begin{equation}	% failure of labelled patterns
%\label{faillabpats-dyn-rule}
%\frac{\forall\VE\ (\r\ts\labpats\not\ra\VE)}
%     {\r\ts\labpats\ra\FAIL}
%\end{equation}
%
%                        Patterns
%

\rulesec{Patterns}{\E,\V\ts\pat\ra\VE/\fail}
\begin{equation}	% atomic pattern
%\label{atpat-dyn-rule}
\frac{\E,\V\ts\atpat\ra \VE/\fail}
     {\E,\V\ts\atpat\ra \VE/\fail}\index{56.3}
\end{equation}

%\begin{equation}	% atomic pattern
%%\label{atpat-dyn-rule}
%\frac{\V\ts\atpat\ra \VE}
%     {\V\ts\atpat\ra \VE}
%\end{equation}

\begin{equation}	% construction pattern
%\label{conpat-dyn-rule1}
\frac{\begin{array}{c}
       \longcon=\strid_1.\cdots.\strid_k.\con\neq\REF\qquad
      \V=(\con,\V')\\
      \E,\V'\ts\atpat\ra\VE/\fail
      \end{array}}
     {\E,\V\ts\conpat\ra \VE/\fail}
\end{equation}

\begin{equation}	% construction pattern
\label{conpat-dyn-rule2}
\frac{\longcon=\strid_1.\cdots.\strid_k.\con\neq\REF\qquad
      \V\notin\{\con\}\times\Val}
     {\E,\V\ts\conpat\ra \FAIL}
\end{equation}

%\begin{equation}	% construction pattern
%\label{conpat-dyn-rule}
%\frac{\longcon=\strid_1.\cdots.\strid_k.\con\neq\REF\qquad\V\ts\atpat\ra\VE}
%     {(\con,\V)\ts\conpat\ra \VE}
%\end{equation}

\begin{equation}        % exception construction
%\label{exconpat-dyn-rule1}
\frac{\begin{array}{c}
      \E(\longexn)=\e\qquad\V=(\e,\V')\\
      \E,\V'\ts\atpat\ra\VE/\FAIL
      \end{array}
     }
     {\E,\V\ts\exconpat\ra\VE/\FAIL}
\end{equation}

\begin{equation} 
\label{exconpat-dyn-rule2}
\frac{\E(\longexn)=\e\qquad\V\notin\{\e\}\times\Val}
     {\E,\V\ts\exconpat\ra\FAIL}
\end{equation}

\begin{equation}	% reference pattern
%\label{refpat-dyn-rule}
\frac{\s(\A)=\V\qquad\s,\E,\V\ts\atpat\ra\VE/\fail,\s}
     {\s,\E,\A\ts\REF\ \atpat\ra \VE/\fail,\s}\index{57.0}
\end{equation}

%\begin{equation}	% reference pattern
%\label{refpat-dyn-rule}
%\frac{\s(\A)=\V\qquad\s,\V\ts\atpat\ra\VE,\s}
%     {\s,\A\ts\REF\ \atpat\ra \VE,\s}
%\end{equation}

\begin{equation}	% layered pattern
%\label{layeredpat-dyn-rule}
\frac{\E,\V\ts\pat\ra\VE/\fail}
     {\E,\V\ts\layeredpata\ra\{\var\mapsto\V\}+\VE/\fail}
\end{equation}
%
%\begin{equation}	% layered pattern
%\label{layeredpat-dyn-rule}
%\frac{\V\ts\pat\ra\VE\qquad\{\var\mapsto\V\}\sim\VE}
%     {\V\ts\layeredpat\ra\VE}
%\end{equation}
%
%\begin{equation}	% failure of pattern
%\label{failpat-dyn-rule}
%\frac{\forall\VE\ (\V\ts\pat\not\ra\VE)}
%     {\V\ts\pat\ra\FAIL}
%\end{equation}
\comments
\begin{description}
\item{(\ref{conpat-dyn-rule2}),(\ref{exconpat-dyn-rule2})}
  Any evaluation resulting in $\FAIL$ must do so because 
rule~\ref{sconpat-dyn-rule2},
rule~\ref{conapat-dyn-rule2},
rule~\ref{exconapat-dyn-rule2},
rule~\ref{conpat-dyn-rule2},
or rule~\ref{exconpat-dyn-rule2} has been
applied.\index{57.1}
\end{description}
