\section{Syntax of the Core}
\label{syn-core-sec}

\subsection{Reserved Words}
The\index{6.1}
following are the {\sl reserved words} used in the Core.   They
may not (except ~{\tt =}~) be used as identifiers.   
%In this document the alphabetic reserved words are 
%always shown in typewriter font.

\vspace*{-6pt}
\begin{verbatim}
     abstype   and   andalso   as   case   do   datatype   else
     end    exception    fn    fun    handle    if   in   infix
     infixr   let     local    nonfix   of   op   open   orelse
     raise   rec   then   type   val   with   withtype    while
     (  )   [  ]   {  }  ,  :  ;  ...    _   |  =   =>   ->   #   
\end{verbatim}
\vspace*{-6pt}

\subsection{Special constants}
\label{cr:speccon}
An\index{6.2} {\sl integer constant} 
is any non-empty sequence of digits, possibly preceded
by a negation symbol (\verb+~+).
A {\sl real constant} is an integer constant,
possibly followed by a point ({\tt .}) and one or
more digits, possibly followed by an exponent symbol ~{\tt E}~ and an integer
constant; at least one of the optional parts must occur, hence no integer
constant is a real constant.
Examples: ~~{\tt 0.7}~~~{\tt 3.32E5}~~~\verb(3E~7(~~.  Non-examples:
%Examples: ~~{\tt 0.7}~~~{\tt +3.32E5}~~~\verb(3E~7(~~.  Non-examples:
%Examples: ~~{\tt 0.7}~~~{\tt +3.32E5}~~~{\tt 3E~}~~.  Non-examples:
~~{\tt 23}~~~{\tt .3}~~~{\tt 4.E5}~~~{\tt 1E2.0}~~.

We assume an underlying alphabet of 256 characters 
(numbered 0 to 255) such that the characters with numbers 
0 to 127 coincide with the ASCII character set.
A {\sl string constant} is a sequence, between quotes ({\tt "}), of zero or
more printable characters (i.e., numbered 33--126), spaces or escape 
sequences.
Each escape sequence starts with the
escape character ~\verb+\+~, and stands for a character sequence. The
escape sequences are:
\smallskip

%\begin{quote}
%\begin{tabular}{ll}
\halign{\indent#\hfil&\quad\parbox[t]{11cm}{\strut#\strut}\cr
\verb+\n+   & A single character interpreted by the system as
end-of-line.\cr
\verb+\t+   & Tab.\cr
\verb+\^+$c$  & The control character $c$, where $c$ may
                be any character with number 64--95. The number
                of ~{\tt\char'134\char'136}$c$~ is 64 less than the 
                number of $c$.\cr
\verb+\+$ddd$ & The single character with number $ddd$ (3 decimal digits
denoting an integer in the interval $[0,255]$).\cr
\verb+\"+   & {\tt "}\cr
\verb+\\+   & {\tt\char'134}\cr
\verb+\+$f\cdot\cdot f$\verb+\+
            & This sequence is ignored,
              where $f\cdot\cdot f$ stands for a sequence 
             of one or more formatting characters.\cr
}
%\end{tabular}
%\end{quote}
\smallskip

The {\sl formatting characters}\index{6.3} are a subset of the non-printable
            characters including at least space, tab, newline, formfeed.
The last form allows long strings to be written on more than one line, by
            writing ~\verb+\+~ at the end of one line and at the start of the
            next.\nopagebreak

We denote by {\SCon} the class of {\sl special constants}, i.e., the integer,
real, and string constants; we shall use {\scon}
to range over \SCon.\index{6.4}

\subsection{Comments}
A\index{7.1} {\sl comment} 
is any character sequence within comment brackets ~{\tt (* *)}~
in which
comment brackets are properly nested. An unmatched comment bracket should be
detected by the compiler.

%
\subsection{Identifiers}
\label{cyn-core-identifiers-sec}
The classes of {\sl identifiers}\index{7.2} for the Core are shown in
Figure~\ref{identifiers}.
\begin{figure}
\vspace{4pt}
\makeatletter{}
\tabskip\@centering
\halign to\textwidth
{#\hfil\tabskip1em&(#)\hfil\tabskip1em&#\hfil\tabskip\@centering\cr
\Var	& value variables	& long\cr
\Con	& value constructors	& long\cr
\Exn    & exception constructors& long\cr
\TyVar	& type variables	& \cr
\TyCon  & type constructors     & long\cr
\Lab    & record labels         & \cr
\StrId  & structure identifiers & long\cr
}
\makeatother
\caption{Identifiers}
\label{identifiers}
\vspace*{-3mm}
\end{figure}
We use $\var$, $\tyvar$ to range over Var, TyVar etc.  For each class
X marked ``long'' there is a class longX of {\sl long identifiers}; if
$x$ ranges over X then {\it longx} ranges over longX.  The syntax of
these long identifiers is given by the following: 
\vspace*{-6pt}
\begin{quote}
\begin{tabular}{rcll} {\it longx} & $::=$ & $x$ & identifier\\
& &$\strid_1.\cdots.\strid_n.x$ & qualified identifier ($n\geq 1$)
\end{tabular} 
\end{quote}
\vspace*{-6pt}
The qualified identifiers constitute a link between the Core and the
Modules. Throughout this document, the term ``identifier'', occurring 
without an adjective, refers to non-qualified identifiers only.
%version 2: For each class X marked
%``long'' there is also a class
%\[ {\rm LongX} = \StrId^\ast \times {\rm X} \]
%If $x$ ranges over X, then {\it longx}, or
%$\strid_1.\cdots.\strid_k.x$, $k\geq 0$, ranges over LongX.
%These long identifiers constitute the only link between the Core
%and the language constructs for Modules; by omitting them, and the $\OPEN$
%declaration,
%we obtain the Core as a complete programming language in
%its own right. (The corresponding adjustment to the Core static and
%dynamic semantics is simply to omit structure environments $\SE$.).

An identifier is either {\sl alphanumeric}: any sequence of
letters, digits, primes ({\tt '}) and underbars (\wildpat) starting
with a letter or prime, or {\sl symbolic}: any non-empty sequence of the
following {\sl symbols}\index{7.3}
\vspace*{-6pt}
\begin{center}
%\verb(!  %  &  $  +  -  /  :  <  =  >  ?  (@\verb(  \  ~  `  ^  |  *(
\verb(!  %  &  $  #  +  -  /  :  <  =  >  ?  @  \  ~  `  ^  |  *(
\end{center}
\vspace*{-6pt}
In either case, however, reserved words are excluded.   This means that for
example ~\verb+#+~ and ~{\tt |}~ are not identifiers, but  ~\verb+##+~ and
~{\tt |=|}~ are identifiers.
The only exception to this rule is that the symbol ~{\tt =}~, which is
a reserved word, is also allowed as an identifier to stand for
the equality predicate.
The identifier ~{\tt =}~ may not be re-bound;
this precludes any syntactic ambiguity.

A type variable $\tyvar$\index{7.4}\label{etyvar-lab} may be any
alphanumeric identifier starting with a prime; the subclass EtyVar of
TyVar, the {\sl equality} type variables, consists of those which
start with two or more primes.  
%poly 
The subclass $\ImpTyVar$ of
$\TyVar$, the {\sl imperative} type variables, consists of those which
start with one or two primes followed by an underbar.  The complement
$\AppTyVar=\TyVar\setminus\ImpTyVar$\index{8.1} consists of the {\sl
applicative} type variables.  The other six classes ({\Var}, {\Con},
{\Exn}, {\TyCon}, {\Lab} and {\StrId}) are represented by identifiers
not starting with a prime. However,\index{7.5} {\tt *} is excluded from {\TyCon},
to avoid confusion with the derived form of tuple type (see
Figure~\ref{typ-gram}). The class Lab\index{8.2} is extended to
include the {\em numeric} labels ~{\tt 1}~~{\tt 2}~~{\tt 3}~ $\cdots$,
i.e. any numeral not starting with~{\tt 0}.

TyVar is therefore disjoint
from the other six classes.   Otherwise, the syntax class of an occurrence of
identifier $\id$ in a Core phrase (ignoring derived forms, 
Section~\ref{cor-der-form-sec}) is determined thus:
\begin{enumerate}
  \item Immediately before ``.'' -- i.e. in a long identifier -- or in an
        $\OPEN$ declaration, $\id$ is a structure
        identifier.  The following rules assume that all occurrences of
        structure identifiers have been removed.
  \item At the start of a component in a record type, record pattern or record
        expression,  $\id$ is a record label.
  \item Elsewhere in types $\id$ is a type constructor, and must be within the
        scope\index{8.3} of the type binding or datatype binding which introduced it.
%  \item Elsewhere $\id$ is an exception name if it occurs immediately after
%        $\RAISE$, at the start of a handler rule $\hanrule$, or within an
%        $\EXCEPTION$ declaration or specification.
  \item Elsewhere, $\id$ is an exception constructor if it occurs in
        the scope of an exception binding which introduces it as such, 
        or a value constructor if it occurs in the
        scope of a datatype binding which introduced it as such;
        otherwise it is a value variable.
\end{enumerate}
It follows from the last rule that no value declaration can make a
``hole'' in the scope of a value or exception constructor 
by introducing the same identifier as a variable; this
is because, in the scope of the declaration which introduces $\id$ as a value
or exception constructor, any occurrence of $\id$ in a pattern 
is interpreted as the
constructor and not as the binding occurrence of a new variable.

By means of the above rules a compiler can determine the class to which each
identifier occurrence belongs; for the remainder of this document we shall
therefore assume that the classes are all disjoint.

\subsection{Lexical analysis}
Each\index{8.4} item of lexical analysis is either a reserved word, a numeric label, a
special constant or a long identifier.
Comments and formatting characters
separate items (except within string constants; see Section~\ref{cr:speccon})
and are otherwise
ignored.   At each stage the longest next item is taken.

\subsection{Infixed operators}
An\index{8.5} identifier may be given {\sl infix status} by the
~$\INFIX$~ or ~$\INFIXR$~ directive , which may occur as a
declaration; this status only pertains to its use as a $\var$, a
$\con$ or an $\exn$ within the scope (see below) of the directive.
(Note that qualified identifiers never have infix status.)  If $\id$
has infix status, then ``$\exp_1\ \id\ \exp_2$'' (resp. ``$\pat_1\
\id\ \pat_2$'') may occur -- in parentheses if necessary -- wherever
the application ``$\id$\verb+{+{\tt 1=}$\exp_1$\verb+,+{\tt
2=}$\exp_2$\verb+}+'' or its derived form
``$\id$\verb+(+$\exp_1$\verb+,+$\exp_2$\verb+)+'' (resp
``$\id$\verb+(+$\pat_1$\verb+,+$\pat_2$\verb+)+'') would otherwise
occur.  On the other hand, an occurrence of any long identifier (qualified
or not) prefixed by {\OP} is treated as non-infixed. The only required
use of {\OP} is in prefixing a non-infixed occurrence of an
identifier $\id$ which has infix status; elsewhere {\OP}, where
permitted, has no effect.\index{9.1}
%version 2: On the other 
%hand, non-infixed occurrences of $\id$ must be prefixed by the reserved word
%~$\OP$~.\index{9.1} 
Infix status is cancelled by the ~$\NONFIX$~
directive.  We refer to the three directives collectively as {\sl
fixity directives}.

The form of the fixity directives is as follows ($n\geq 1$):
\[ \longinfix \]
\[ \longinfixr \]
\[ \longnonfix \]
where $\langle d\rangle$ is an optional decimal digit $d$ indicating
binding precedence. A higher value of $d$ indicates tighter binding;
the default is $0$.  ~$\INFIX$~ and ~$\INFIXR$~ dictate left and right
associativity respectively; association is always to the left for different
operators of the same precedence.  The precedence of infix operators relative
to other expression and pattern constructions is given in
Appendix~\ref{core-gram-app}.

The {\sl scope}\index{9.2} of a fixity directive $\dir$ is the ensuing program text,
except that if $\dir$ occurs in a declaration $\dec$ in either of the phrases
\[ \LET\ \dec\ \IN\ \cdots\ \END \]
\[ \LOCAL\ \dec\ \IN\ \cdots\ \END \]
then the scope of $\dir$ does not extend beyond the phrase. Further scope
limitations are imposed for Modules.

These directives and ~$\OP$~ are omitted from the semantic rules, since they
affect only parsing.

\subsection{Derived Forms}
\label{cor-der-form-sec}
There\index{9.3} are many standard syntactic forms in ML whose meaning can be expressed
in terms of a smaller number of syntactic forms, called the {\sl bare} language.
These derived forms, and their equivalent forms in the bare language, are
given in
Appendix~\ref{derived-forms-app}.

%With one exception, these derived forms use no new lexical items.  The
%exception is that the symbol \verb+#+ prefixed to an identifier of the
%class Lab constitutes a
%lexical item;  \verb+#+{\it lab} denotes a selector function on records, cf. page~\pageref{der-exp}.

\subsection{Grammar}

The phrase classes for the Core are shown in Figure~\ref{cor-phr}.
We use the variable $\atexp$ to range over AtExp, etc.

The grammatical rules for the Core are shown in Figures~\ref{exp-syn}
and \ref{pat-syn}.

\clearpage
\begin{figure}[t]
\vspace{4pt}
\makeatletter{}
\tabskip\@centering
\halign to\textwidth
{#\hfil\tabskip1em&#\hfil\tabskip\@centering\cr
AtExp	& atomic expressions \cr
ExpRow  & expression rows \cr
Exp     & expressions \cr
Match   & matches \cr
Mrule   & match rules \cr
\noalign{\vspace{2mm}}
%\cr
Dec     & declarations \cr
ValBind & value bindings \cr
TypBind & type bindings \cr
DatBind & datatype bindings \cr
ConBind & constructor bindings \cr
%version 1: Constrs & datatype constructions \cr
ExBind  & exception bindings \cr
\noalign{\vspace{2mm}}
%\cr
AtPat   & atomic patterns \cr
PatRow  & pattern rows \cr
Pat     & patterns \cr
\noalign{\vspace{2mm}}
%\cr
Ty      & type expressions \cr
TyRow   & type-expression rows \cr
}
\makeatother
\caption{Core Phrase Classes}
\label{cor-phr}
\end{figure}

The following\index{10.1} conventions are adopted in presenting the grammatical rules,
and in their interpretation:
\begin{itemize}
  \item The brackets\index{10.2} ~$\langle\ \rangle$~ enclose optional phrases.
  \item For any syntax class X (over which $x$ ranges)
we define the syntax class Xseq (over which {\it xseq} ranges) as follows:
    \begin{quote}
    \begin{tabular}{rcll}
       {\it xseq} & $::=$ & $x$ & (singleton sequence)\\
                  &       &     & (empty sequence)\\
                  &       & \ml{(}$x_1$\ml{,}$\cdots$\ml{,}$x_n$\ml{)}
                                & (sequence,~$n\geq 1$) \\
    \end{tabular}
    \end{quote}
(Note that the ``$\cdots$'' used here, meaning syntactic iteration, must not be
confused with ``$\wildrec$'' which is a reserved word of the language.)
  \item Alternative forms for each phrase class are in order of decreasing
        precedence; this resolves ambiguity in parsing, as explained
        in Appendix~\ref{core-gram-app}.\index{10.3}
  \item L (resp. R)\index{10.4} means left (resp. right) association.

\item The syntax of types binds more tightly than that of expressions.

\item Each\index{10.6} iterated construct (e.g. $\match$, $\cdots$)
extends as far
right as possible; thus, parentheses may be needed around an expression which
terminates with a match, e.g. ``$\FN\ \match$'', if this occurs within a
larger
match.
\end{itemize}


\begin{figure}[t]
\vspace{4pt}
\makeatletter{}
\tabskip\@centering
\halign to\textwidth
{#\hfil\tabskip1em&\hfil$#$\hfil&#\hfil&#\hfil\tabskip\@centering\cr
  \atexp& ::=	& \scon 	& special constant\cr
        & 	& \opp\longvar	& value variable\cr
	&	& \opp\longcon	& value constructor\cr
        &       & \opp\longexn  & exception constructor\cr
	&	& \verb+{ +\recexp\verb+ }+
	                	& record\cr
	&	& \letexp	& local declaration\cr
	&	& \parexp	& \cr
\noalign{\vspace{6pt}}
\labexps& ::=	& \longlabexps	& expression row\cr
\noalign{\vspace{6pt}}
  \exp  & ::=	& \atexp 	& atomic\cr
	&	& \appexp	& application (L)\cr
	&	& \infexp       & infixed application\cr
	&	& \typedexp	& typed (L)\cr
	&	& \handlexp	& handle exception\cr
	&	& \raisexp	& raise exception\cr
	&	& \fnexp        & function\cr
\noalign{\vspace{6pt}}
\match  & ::=	& \longmatch    & \cr
\noalign{\vspace{6pt}}
\mrule	& ::=	& \longmrule	& \cr
\noalign{\vspace{6pt}}
  \dec  & ::=	& \valdec	& value declaration\cr
	&	& \typedec	& type declaration\cr
	&	& \datatypedec  & datatype declaration\cr
	&	& \abstypedec   & abstype declaration\cr
	&	& \exceptiondec & exception declaration\cr
	&	& \localdec	& local declaration\cr
        &       & \openstrdec   & open declaration ($n\geq 1$) \cr
	&	& \emptydec	& empty declaration\cr
	&	& \seqdec	& sequential declaration\cr
        &       & \longinfix    & infix (L) directive\cr
        &       & \longinfixr   & infix (R) directive\cr
        &       & \longnonfix   & nonfix directive\cr
\noalign{\vspace{6pt}}
\valbind& ::=   & \longvalbind   & \cr
	&	& \recvalbind	& \cr
\noalign{\vspace{6pt}}
\typbind& ::=	& \longtypbind	& \cr
\noalign{\vspace{6pt}}
\datbind& ::=	& \longdatbind	& \cr
\noalign{\vspace{6pt}}
\constrs& ::=	& \opp\longconstrs & \cr
\noalign{\vspace{6pt}}
\exnbind& ::=	& \generativeexnbind	& \cr
        &       & \eqexnbind   & \cr
\noalign{\vspace{6pt}}
}
\makeatother
\vspace{-2mm}
\caption{Grammar: Expressions, Matches, Declarations and Bindings\index{11}\index{12.1}}
\label{exp-syn}
\end{figure}
\clearpage % 1 August
\begin{figure}[h]
%\vspace{4pt}
\makeatletter{}
\tabskip\@centering
\halign to\textwidth
{#\hfil\tabskip1em&\hfil$#$\hfil&#\hfil&#\hfil\tabskip\@centering\cr
  \atpat& ::=	& \wildpat	& wildcard\cr
  	&	& \scon  	& special constant\cr
  	&	& \opp\var  	& variable\cr
	&	& \opp\longcon  & constant\cr
        &       & \opp\longexn  & exception constant\cr
	&	& \verb+{ +\recpat\verb+ }+
	                        & record\cr
	&	& \parpat       & \cr
\noalign{\vspace{6pt}}
\labpats& ::=	& \wildrec	& wildcard\cr
  	&	& \longlabpats 	& pattern row\cr
\noalign{\vspace{6pt}}
  \pat	&	& \atpat	& atomic\cr
	&	& \opp\conpat	& value construction\cr
        &       & \opp\exconpat  & exception construction\cr
	&	& \infpat       & infixed value construction\cr
        &       & \infexpat     & infixed exception construction\cr
	&	& \typedpat	& typed\cr
	&	& \opp\layeredpat	& layered\cr
\noalign{\vspace{6pt}}
  \ty   & ::=	& \tyvar        & type variable\cr
	&	& \verb+{ +\rectype\verb+ }+
	                        & record type expression\cr
	&	& \constype 	& type construction\cr
	&	& \funtype      & function type expression (R)\cr
	&	& \partype      & \cr
\noalign{\vspace{6pt}}
\labtys & ::=	& \longlabtys   & type-expression row\cr
\noalign{\vspace{6pt}}
}
\makeatother
\vspace{-2mm}
\caption{Grammar: Patterns and Type expressions\index{12.2}\index{13.1}}
\label{pat-syn}
\end{figure}
\nopagebreak[4]
\subsection{Syntactic Restrictions}\index{13.2}
\begin{itemize}
\item No pattern may contain the same $\var$ twice. No expression row,
      pattern row or type row may bind the same $\lab$ twice.
\item No binding $\valbind$, $\typbind$, $\datbind$ or $\exnbind$ may bind
      the same identifier twice; this applies also to value constructors within
      a $\datbind$.
\item In the left side $\tyvarseq\ \tycon$ of any $\typbind$ or $\datbind$,
      $\tyvarseq$ must not contain the same $\tyvar$ twice. Any $\tyvar$
      occurring within the right side must occur in $\tyvarseq$. 
\item For each value binding \pat\ \ml{=} \exp\ within $\REC$,
      $\exp$ must be of the form \fnexp, possibly constrained by one
      or more type expressions. The derived form
      of function-value binding given in Appendix~\ref{derived-forms-app},
      page~\pageref{der-dec}, necessarily obeys this restriction.
%from version 1:
%\item Every non-local exception binding -- that is, not localised by $\LET$
%      or $\LOCAL$ -- must be explicitly constrained by a type containing
%      no type variables.
%not needed, covered by polymorphic references and exceptions:
%\item Any type expression $\ty$ occurring in a non-local 
%      exception binding -- that is,
%      one not localised by $\LET$ or $\LOCAL$ -- must contain no type
%      variables.
\end{itemize}
