\section{Appendix: The Initial Dynamic Basis}
\label{init-dyn-bas-app}
We\index{76.1} shall indicate components of the initial basis by the subscript 0.
\insertion{\thelibrary}
The initial dynamic basis is $\B_0 = \F_0,\G_0,\E_0$, 
where $\F_0 = \emptymap$, $\G_0 = \emptymap$ and $\E_0 = (\SE_0, \TE_0,\VE_0)$,
where $\SE_0 = \emptymap$, $\TE_0$ is shown in Figure~\ref{dynTE0.fig} and
\medskip

$\VE_0 = \{\boxml{=}\mapsto(\boxml{=},\isv),\,\boxml{:=}\mapsto(\boxml{:=},\isv),\,\boxml{Match}\mapsto(\boxml{Match},\ise), \,\boxml{Bind}\mapsto(\boxml{Bind},\ise),$\\
       \vrule height0pt width21mm depth 0pt$\boxml{true}\mapsto(\boxml{true},\isc),\,\boxml{false}\mapsto(\boxml{false},\isc),\,$\\
       \vrule height0pt width21mm depth 0pt$\boxml{nil}\mapsto(\boxml{nil},\isc),\,
\hbox{\boxml{::}}\mapsto(\hbox{\boxml{::}},\isc),\,
\hbox{\boxml{ref}}\mapsto(\hbox{\boxml{ref}},\isc)\}$.

\deletion{\thelibrary}{
The initial dynamic basis is
\[ \B_0\ =\ \F_0,\G_0,\E_0\]
where
\begin{itemize}
\item $\F_0\ =\ \emptymap$
\item $\G_0\ =\ \emptymap$
\item $\E_0\ =\ \E_0'+\E_0''$
\end{itemize}
$\E_0'$ contains bindings of identifiers to the basic values BasVal and
basic exception names \BasExc; in fact
~$\E_0'\ =\ \SE_0',\VE_0',\EE_0'$~, where:
\begin{itemize}
\item $\SE_0'\ =\ \emptymap$
\item $\VE_0' = \{{\it id}\mapsto{\it id}\ ;\ {\it id}\in\hbox{BasVal}\}
       +\{\hbox{\boxml{:=}}\mapsto\hbox{\boxml{:=}}\} +\EE_0'$\\
       \vrule height0pt width3mm depth 0pt$+\;\{\boxml{true}\mapsto\boxml{true},\,\boxml{false}\mapsto\boxml{false},\,\boxml{nil}\mapsto\boxml{nil},\,
\hbox{\boxml{::}}\mapsto\hbox{\boxml{::}},\,
\hbox{\boxml{ref}}\mapsto\hbox{\boxml{ref}}\}$
\item $\EE_0'\ =\ \{\id\mapsto\id\ ;\ \id\in$ \BasExc$\}$
\end{itemize}
}

\deletion{\thelibrary}{Note that $\VE_0'$ is the identity function on BasVal; this is because
we have chosen to denote these values by the names of variables 
to which they are initially bound.
The semantics of these basic values (most of which are functions)
lies principally in their behaviour under APPLY, which we describe below.
On the other hand the semantics of \boxml{:=} is provided by a special
semantic rule, rule~\ref{assapp-dyn-rule}.
Similarly, $\EE_0'$ is the identity function on \BasExc, the set of
basic exception names, because we have also chosen
these names to be just those exception constructors to which they
are initially bound.
These exceptions are raised by APPLY as described below.

 $\E_0''$ contains initial variable bindings which, unlike BasVal, are
 definable in ML; it is the result of evaluating
 the following declaration in the basis $\F_0,\G_0,\E_0'$.  For convenience,
 we have also included all basic infix directives in this declaration.
 \begin{verbatim}
                infix  3  o
                infix  4  = <> < > <= >=
                infix  5  @
                infixr 5  ::
                infix  6  + - ^
                infix  7  div mod / *

                fun (F o G)x = F(G x)

                fun nil @ M = M
                  | (x::L) @ M = x::(L @ M)



                fun s ^ s' = implode((explode s) @ (explode s'))

                fun map F nil = nil
                  | map F (x::L) = (F x)::(map F L)

                fun rev nil = nil
                  | rev (x::L) = (rev L) @ [x]

                fun not true = false
                  | not false = true

                fun ! (ref x) = x
 \end{verbatim}
}
% We now\index{77.1} describe the effect of APPLY upon each value
% $b\in\BasVal$.  For special values, we shall normally use $i$, $r$,
% $n$, $s$ to range over integers, reals, numbers (integer or real),
% strings respectively.  We also take the liberty of abbreviating
% ``APPLY(\mbox{{\tt abs}}, $r$)'' to ``\mbox{{\tt abs}}($r$)'',
% ``APPLY({\tt mod}, $\{\boxml{1}\mapsto i,\boxml{2}\mapsto d\}$)'' to
% ``$i\ {\tt mod}\ d$'', etc. .
% \begin{itemize}
% \item \verb+~+($n$)~ returns the negation of $n$, or the 
%       packet ~[{\tt Neg}]~ if the result is out of range.
% \item \mbox{{\tt abs}}($n$)~ returns the absolute value of $n$, or
%       the packet ~[{\tt Abs}]~ if the result is out of range.
% \item {\tt floor}($r$)~ returns the largest integer $i$ not greater than $r$;
%       it returns the packet ~[{\tt Floor}]~ if $i$ is out of range.
% \item {\tt real}($i$)~ returns the real value equal to $i$.
% \item {\tt sqrt}($r$)~ returns the square root of $r$, or the packet
%       ~[{\tt Sqrt}]~ if r is negative.
% \item {\tt sin}($r$)~,  {\tt cos}($r$)~ return the result of the appropriate
%       trigonometric functions.
% \item {\tt arctan}($r$)~ returns the result of the appropriate
%       trigonometric function in the range $\underline{+}\pi/2$.
% \item {\tt exp}($r$)~, {\tt ln}($r$)~ return respectively the exponential
%       and the natural logarithm of $r$, or an exception packet
%       ~[{\tt Exp}]~ or ~[{\tt Ln}]~ if the result is out of range.
% \item {\tt size}($s$)~ returns the number of characters in $s$.
% \item {\tt chr}($i$)~ returns the character numbered $i$ (see Section~\ref{cr:speccon}) if $i$ is in the interval $[0,255]$, and the packet
%       ~[{\tt Chr}]~ otherwise.
% \pagebreak
% \item {\tt ord}($s$)~ returns\index{78.0} the number of the first character
%       in $s$ (an integer in the interval $[0,255]$, see Section~\ref{cr:speccon}), 
%       or the packet ~[{\tt Ord}]~ if $s$ is empty.
% \item {\tt explode}($s$)~ returns the list of characters (as single-character
%       strings) of which $s$ consists.
% \item {\tt implode}($L$)~ returns the string formed by concatenating all members
%       of the list $L$ of strings.
% \item The arithmetic\index{78.1} functions ~\verb+/+,\verb+*+,\verb-+-,\verb+-+~ all
%       return the results of the usual
%       arithmetic operations, or exception packets respectively
%       [{\tt Quot}], [{\tt Prod}], [{\tt Sum}], [{\tt Diff}]
%       if the result is undefined or out of range.
% \item $i\ {\tt mod}\ d$~,~$i\ {\tt div}\ d$~ return integers $r,q$
%       (remainder, quotient) determined by the equation $d\times q +r=i$,
%       where either $0\leq r<d$ or $d<r\leq 0$.  Thus the remainder has the
%       same sign as the divisor $d$. The packet [{\tt Mod}] or
%       [{\tt Div}] is returned if $d=0$.
% \item The order relations ~\verb+<+,\verb+>+,\verb+<=+,\verb+>=+~ return
%       boolean values in accord with their usual meanings.
% \item $v_1$\verb+ = +$v_2$~ returns {\TRUE} or {\FALSE} according as
%       the values $v_1$ and $v_2$ are, or are not, identical.
%       The type discipline (in particular, the fact that function types
%       do not admit equality) ensures that equality is only ever applied
%       to special values, nullary constructors, addresses, and values
%       built out of such by record formation and constructor application.
% %version 2:\item $v_1$\verb+ = +$v_2$~ returns the boolean value of $v_1=v_2$,
% %      where the equality of values (=) is defined recursively as follows:
% %      \begin{itemize}
% %      \item If $v_1,v_2$ are constants (including nullary constructors) or
% %      addresses, then $v_1=v_2$ iff $v_1$ and $v_2$ are identical.
% %      \item $(\con_1,v_1)=(\con_2,v_2)$ iff $\con_1,\con_2$ are identical and
% %            $v_1=v_2$.
% %      \item $r_1=r_2$ (for records $r_1,r_2$) iff $\Dom r_1=\Dom r_2$ and, for
% %            each $\lab\in\Dom r_1$, $r_1(\lab)=r_2(\lab)$.
% %      \end{itemize}
% %      The type discipline (in particular, the fact that function types
% %      do not admit equality) makes it unnecessary to specify equality in
% %      any other cases.
% \item $v_1$\verb+ <> +$v_2$ returns the opposite boolean value to
%       $v_1$\verb+ = +$v_2$.
% \end{itemize}
% It remains to define the effect of APPLY upon basic values concerned with
% input/output; we therefore proceed to describe the ML input/output system.
%  
% Input/Output in ML uses the concept of a {\sl stream}. A stream is a finite or
% infinite sequence of characters; if finite, it may or may not be terminated.
% (It may be convenient to think of a special end-of-stream character
% signifying termination, provided one realises that this ``character'' is
% never treated as data).
% Input streams -- or {\sl instreams} --
% are of type ~\INSTREAM~ and will be denoted by ~$is$~;
% output streams -- or {\sl outstreams} -- are of type ~\OUTSTREAM~ and will
% be denoted by ~$os$~. Both these types of stream are {\sl abstract}, in the
% sense that streams may only be manipulated by the functions provided in
% BasVal.
% 
% Associated with an instream is a {\sl producer}, normally an I/O device or
% file; similarly an outstream is associated with a {\sl consumer}.  After this
% association has been established -- either initially or by the ~{\tt open\_in}~
% or ~{\tt open\_out}~ function -- the stream acts as a vehicle for character
% transmission from producer to program, or from program to consumer.
% The association can be broken by the ~{\tt close\_in}~ or ~{\tt close\_out}~
% function.\index{79.1}
% A closed stream permits no further character transmission; a closed
% instream is equivalent to one which is empty and terminated.
% \pagebreak
% 
% There\index{79.1.1} are two streams in BasVal:
% \begin{itemize}
% \item  {\tt std\_in}: an instream whose producer is the terminal.
% \item  {\tt std\_out}: an outstream whose consumer is the terminal.
% \end{itemize}
% The other basic values concerned with Input/Output are all functional, and
% the effect of APPLY upon each of them given below. We take the
% liberty of abbreviating ``APPLY({\tt open\_in}, $s$)'' to 
% ``{\tt open\_in}($s$)''
% etc., and
% we shall use ~$s$~ and ~$n$~ to range over strings and integers
% respectively.
% \begin{itemize}
% \item  {\tt open\_in}($s$)~ returns a new instream ~$is$~, whose producer is
%        the external file named ~$s$~. It returns exception packet
%        \begin{quote}
%        [$(${\tt Io},\verb+"Cannot open +$s$\verb+"+$)$]
%        \end{quote}
%        if file ~$s$~ does not exist or does not provide read access.
% \item  {\tt open\_out}($s$)~ returns a new outstream ~$os$~, whose consumer is
%        the
%        external file named ~$s$~. If file $s$ is non-existent, it is taken to
%        be initially empty. \ insertion{\thecommentary}{Any existing contents of the file $s$ are 
%            lost. The exception packet
%        \begin{quote}
%        [$(${\tt Io},\boxml{"Cannot open }$s$\boxml{"}$)$]
%        \end{quote}
%        is returned if write access to the file ~$s$~ is not provided.}
% \item  {\tt input}($is,n$)~ returns a string ~$s$~ containing the first $n$
%        characters of ~$is$~, also removing them from ~$is$~.  If only $k<n$
%        characters are available on ~$is$~, then
%        \begin{itemize}
%        \item If ~$is$~ is terminated after these $k$ characters, the
%              returned string
%             ~$s$~ contains them alone, and they are removed from ~$is$~.
%        \item Otherwise no result is returned until the producer of ~$is$~
%              either supplies $n$ characters or terminates the stream.
%        \end{itemize}
% \item  {\tt lookahead}($is$)~ returns a single-character string ~$s$~ containing
%        the next character of ~$is$~, without removing it. If no character is
%        available on ~$is$~ then
%        \begin{itemize}
%        \item If ~$is$~ is closed, the empty string is returned.
%        \item Otherwise no result is returned until the producer of ~$is$~
%              either supplies a character or closes the stream.
%        \end{itemize}
% \item  {\tt close\_in}($is$)~ empties and terminates the instream ~$is$~ .
% \item  {\tt end\_of\_stream}($is$)~ returns {\TRUE} if 
%        ~{\tt lookahead}($is$)~ returns the empty string, {\FALSE} otherwise; 
%         it detects the end of the instream ~$is$~.
% \item {\tt output}($os,s$)~ writes the characters of ~$s$~  to the outstream
%        ~$os$~, unless ~$os$~ is closed, in which case it returns the exception
%        packet
%        \begin{quote}
%        [$(${\tt Io},\verb+"Output stream is closed"+$)$]
%        \end{quote}
% \item  {\tt close\_out}($os$)~ terminates the outstream ~$os$~.\index{79.2}
% \end{itemize}
% %\end{document}
% 
% %              Constructors
% %\newcommand{\FALSE}{\mbox{\tt false}}
% %\newcommand{\TRUE}{\mbox{\tt true}}
% %\newcommand{\NIL}{\mbox{\tt nil}}
% %\newcommand{\REF}{\mbox{\tt ref}}
% %\newcommand{\UNIT}{\mbox{\tt unit}}
% 
% %              Basic Values BasVal
% %\newcommand{\STDIN}{\mbox{\tt std\_in}}
% %\newcommand{\STDOUT}{\mbox{\tt std\_out}}

\insertion{\thedatatyperepl}{
\begin{figure}[h]
\begin{center}
\begin{tabular}{|rll|}
\hline
$\tycon$   & $\mapsto$  & $\{\vid_1\mapsto(\V_1,\is_1),\ldots,\vid_n\mapsto(\V_n,\is_n)\}\quad (n\geq0)$\\
\hline
\UNIT      & $\mapsto $ &  $\emptymap$ \\
\BOOL      & $\mapsto $ & $\{\TRUE\mapsto(\TRUE,\isc),
                                         \ \FALSE\mapsto(\FALSE,\isc)\}$\\
\INT       & $\mapsto $ & $\{\}$\\
\WORD      & $\mapsto $ & $\{\}$\\
\REAL      & $\mapsto $ & $\{\}$\\
\STRING    & $\mapsto $ & $\{\}$\\
%\UNISTRING & $\mapsto $ & $\{\}$\\
\CHAR      & $\mapsto $ & $\{\}$\\
%\UNICHAR   & $\mapsto $ & $\{\}$\\
\LIST      & $\mapsto $ & $\{\NIL\mapsto(\NIL,\isc),\ml{::}\mapsto(\ml{::},\isc)\}$\\
\REF       & $\mapsto $ & $\{\REF\mapsto(\REF,\isc)\}$\\
\EXCN      & $\mapsto $ & $\emptymap$\\
\hline
\end{tabular}
\end{center}
\caption{Dynamic $\TE_0$}
\label{dynTE0.fig}
\end{figure}}
